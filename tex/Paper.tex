%&latex2e


\documentclass[twoside,final]{article}
\usepackage{latexsym,e-journal}



\usepackage[letterpaper]{geometry}
%\usepackage[utf8]{inputenc}
\usepackage[english]{babel}

\usepackage{amssymb,amsfonts,amsmath}

\usepackage[perpage,symbol*]{footmisc}
\usepackage[final]{graphicx}
%\usepackage{pstricks}
\usepackage{cite}

\usepackage[varg]{txfonts}


\oddsidemargin=-0.20in
\evensidemargin=-0.20in
\topmargin=-30pt

\textwidth=498pt
\textheight=646pt

\usepackage[colorlinks=true, urlcolor=blue, citecolor=blue]{hyperref}
%\usepackage[colorlinks=false]{hyperref}

\begin{document}

%\sloppy

\renewcommand{\refname}{References}
\renewcommand{\tablename}{\small Table}
\renewcommand{\figurename}{\small Fig.}
\renewcommand{\contentsname}{Contents}


\twocolumn[%
\begin{center}
\renewcommand{\baselinestretch}{0.93}
{\Large\bfseries 

Revision and integration of Maxwell's and Navier-Stokes' Equations and the
origin of quantization in Superfluids and Spacetime itself



}\par
\renewcommand{\baselinestretch}{1.0}
\bigskip
Arend Lammertink\\ 
{\footnotesize  Schoolstraat 107, 7471 WV, Goor, The Netherlands.\rule{0pt}{12pt}
E-mail: lamare@gmail.com\\
}\par
\medskip
{\small\parbox{11cm}{%

\textbf{Abstract.} 

It is well known that the Maxwell equations predict the behavior of the
electromagnetic field very well. However, they predict only one wave equation
while there are significant differences between the "near" and "far" fields and
various anomalies have been observed involving the detection of super luminous
signals in experiments with electrically short coaxial cables 
\cite{Kuehn2019, Kuehn2020}, microwaves \cite{Ranfagni2004, Allaria2011, Agresti2015,
Mojahedi2000, Musha2019, Wang2003, Walker2006, Solemino2014, Ranfagni2006,
Walker2000}, optical fibers \cite{Gonzalez2005, Stenner2016, Thevenaz2008,
Wang2000}, as well as other methods \cite{Ardavan2004, Niang2018,
Wheatstone1834, Tesla1900}. 

We show that the mathematical Laplace operator defines a complete set of vector
fields consisting of two potential fields and two fields of force, which form a
Helmholtz decomposition of any given vector field $\mathbf{F}$. We found that
neither in Maxwell's equations nor in fluid dynamics vector theory this result
has been recognized, which causes the potential fields to not be uniquely
defined and also makes the Navier-Stokes equations unnecessarily complicated and
introduces undesirable redundancy as well. We show that equivalents to both the
Maxwell equations as well as the Navier-Stokes equations can be directly derived
from a single diffusion equation describing Newton's second law in 3D. We found
that the diffusion constant $k$ in this equation has the same value as the speed
of light squared, but has a unit of measurement in meters squared per second
thus uncovering problems with time derivatives in current theories, showing
among others that the mass-energy equivalence principle is untenable. Finally,
we show that the diffusion equation we found can be divided by mass density
$\rho$, resulting in a velocity diffusion equation that only has units of
measurement in meters and seconds, thus decoupling the dynamics of the medium
from it's substance, mass density $\rho$. This reveals the quantized nature of
spacetime itself, whereby the quantum circulation constant k is found to govern
the dynamics of physical reality, leading to the conclusion that at the
fundamental quantum level only dynamic viscous forces exist while static elastic
forces are an illusion created by problems with a number of time derivatives in
current theories.

With our equivalents for the Maxwell equations three types of wave phenomena can
be described, including super luminous longitudinal sound-like waves that can
explain the mentioned anomalies. This paper contributes to the growing body of
work revisiting Maxwell's equations \cite{vanVlaenderen2001, Barrett1993,
Pinheiro2005, Behera2018, Nedic2017, Atsukovsky2011, Arbab2014, Onoochin2019,
Gray2019, Shaw2014} by deriving all of the fields from a single equation, so the
result is known to be mathematically consistent and free of singularities and
uniquely defines the potential fields thus eliminating gauge freedom. Unlike
Maxwell's equations, which are the result of the entanglement of Faraday's
circuit level law with the more fundamental medium arguably creating most of the
problems in current theoretical physics, these revisions describe the three
different electromagnetic waves observed in practice and so enable a better
mathematical representation.


Keywords: Classical Electrodynamics, Superfluid medium, Fluid Dynamics,
Theoretical Physics, Vector Calculus.


}}\smallskip
\end{center}]{%


\setcounter{section}{0}
\setcounter{equation}{0}
\setcounter{figure}{0}
\setcounter{table}{0}
\setcounter{page}{1}


\markboth{Arend Lammertink. Revision of Maxwell's equations}{\thepage}
\markright{Arend Lammertink. Revision of Maxwell's equations}


\section{Introduction}
\markright{Arend Lammertink. Revision of Maxwell's equations}


In 1861, James Clerk Maxwell published his paper "On Physical Lines of Force"
\cite{Maxwell1861}, wherein he theoretically derived a set of twenty equations
which accurately described the electromagnetic field insofar as known at that
time. He modeled the magnetic field using a molecular vortex model of Michael
Faraday's "lines of force" in conjunction with the experimental result of Weber
and Kohlrausch, who determined in 1855 that there was a quantity related to
electricity and magnetism, the ratio of the absolute electromagnetic unit of
charge to the absolute electrostatic unit of charge, and determined that it
should have units of velocity. In an experiment, which involved charging and
discharging a Leyden jar and measuring the magnetic force from the discharge
current, they found a value 3.107e8 m/s, remarkably close to the speed of light.

In 1884, Oliver Heaviside, concurrently with similar work by Josiah Willard 
Gibbs and Heinrich Hertz, grouped Max\-well's twenty equations together into a set
of only four, via vector notation. This group of four equations was known
variously as the Hertz-Heaviside equations and the Maxwell-Hertz equations but
are now universally known as Maxwell's equations. 

The Maxwell equations predict the existence of just one type of electromagnetic
wave, even though it is now known that at least two electro\-magnetic wave
phenomena exist, name\-ly the "near" and the "far" fields. The "near" field has
been shown to be a non-radiating surface wave that is guidable along a
completely unshielded single conductor \cite{Elmore2019} and can be applied for
wide band low loss communication systems. The Maxwell equations have not been
revised to incorporate this new knowledge.

Given the above, the following questions should be asked: 

\begin{itemize}
\item What is charge?
\item Why is it a property of certain particles?
\end{itemize}

As long as we insist that charge is an elemental quantity that is a property of
certain particles, we cannot answer these questions. Also, when the wave
particle duality principle is considered in relation to what is considered to be
the cause for electromagnetic radiation, charged particles, in Maxwell's
equations electromagnetic radiation is essentially considered to be caused by
(quanta of) electromagnetic radiation, an obvious case of circular logic which
is not desirable. 

In the area of vector calculus, Helmholtz's theorem, also known as the
fundamental theorem of vector calculus, states that any sufficiently smooth,
rapidly decaying vector field in three dimensions can be resolved into the sum
of an irrotational (curl-free) vector field and a solenoidal (divergence-free)
vector field; this is known as the Helmholtz decomposition. A terminology often
used in physics refers to the curl-free component of a vector field as the
longitudinal component and the divergence-free component as the transverse
component. This theorem is also of great importance in electromagnetic (EM) and
microwave engineering, especially for solving the low-frequency breakdown issues
caused by the decoupling of electric and magnetic fields \cite{Xiong2014}.
Further, a vector field can be uniquely specified by a prescribed divergence and
curl and it can be shown that the Helmholtz theorem holds for arbitrary vector
fields, both static and time-dependent \cite{Oleinik2003}.

In potential theory, the study of harmonic functions, the Laplace equation is
very important, amongst other with regards to consideration of the symmetries of
the Laplace equation. The symmetries of the n-dimensional Laplace equation are
exactly the conformal symmetries of the n-dimensional Euclidean space, which has
several implications. One can systematically obtain the solutions of the Laplace
equation which arise from separation of variables such as spherical harmonic
solutions and Fourier series. By taking linear superpositions of these
solutions, one can produce large classes of harmonic functions which can be
shown to be dense in the space of all harmonic functions under suitable
topologies. 

The Laplace equation as well as the more general Poisson equation are 2nd order
differential equations, in both of which the Laplacian represents the flux
density of the gradient flow of a function. In one dimension, the Laplacian
simply is ${\partial}${\texttwosuperior}/${\partial}$x{\texttwosuperior},
representing the curvature of a given function $f$. For scalar functions in 3D,
the Laplacian is a common generalization of the second derivative and is the
differential operator defined by:

\begin{equation}
\Delta f = \frac{\partial ^2f}{\partial x^2}
         + \frac{\partial ^2f}{\partial y^2}
         + \frac{\partial ^2f}{\partial z^2}
\end{equation}

The Laplacian of a scalar function is equal to the divergence of the gradient
and the trace of the Hessian matrix. The vector Laplacian is a further
generalization in three dimensions and defines the second order spatial
derivative of any given vector field F, the "3D curvature" if you will, and is
given by the identity:

\begin{equation}
\Delta \boldsymbol{F}= \nabla (\nabla \cdot \boldsymbol{F})
                     - \nabla \times (\nabla \times \boldsymbol{F})
\end{equation}

Whereas the scalar Laplacian applies to a scalar field and returns a scalar
quantity, the vector Laplacian applies to a vector field, returning a vector
quantity. When computed in orthonormal Cartesian coordinates, the returned
vector field is equal to the vector field of the scalar Laplacian applied to
each vector component. 

With this identity, a full 3D generalization of the Poisson equation can also be
defined, the vector wave equation, which has three independent
solutions \cite{Stratton1941}, the vector spherical harmonics:

\begin{equation}\label{seq:refText3}
 \Delta \boldsymbol{F}  + k^2 \boldsymbol{F}
 = \nabla (\nabla \cdot \boldsymbol{F})  
 - \nabla \times (\nabla \times \boldsymbol{F})
 + k^2 \boldsymbol{F} = 0.
\end{equation}

\section{Methods}
\markright{Arend Lammertink. Revision of Maxwell's equations}

The terms in the definition for the vector Laplacian can be negated:

\begin{equation}
-\nabla ^2\boldsymbol F=-\nabla (\nabla \cdot \boldsymbol F)+\nabla \times (\nabla \times \boldsymbol F)
\end{equation}

and then the terms in this identity can be written out to define a vector field
for each of these terms:  

\begin{equation}\label{seq:refText5}
\begin{matrix}\hfill \boldsymbol A&=&\nabla \times \boldsymbol F\hfill\null \\\hfill \varphi &=&\nabla \cdot
\boldsymbol F\hfill\null \\\hfill \boldsymbol B&=&\nabla \times \boldsymbol A\text{ = }\nabla \times
(\nabla \times \boldsymbol F)\hfill\null \\\hfill \boldsymbol E&=&-\nabla \varphi \text{ = }-\nabla (\nabla
\cdot \boldsymbol F)\hfill\null \end{matrix}
\end{equation}

And, since the curl of the gradient of any twice-differentiable scalar field
$\varphi $ is always the zero vector (${\nabla}${\texttimes}(${\nabla}$$\varphi
$)=0), and the divergence of the curl of any vector field A is always zero as
well (${\nabla}{\cdot}$(${\nabla}${\texttimes}A)=0), we can establish that E is
curl-free and B is divergence-free, and we can write: 

\begin{equation}
\begin{matrix}\hfill \nabla \times \boldsymbol E&=&0\hfill\null \\\hfill \nabla \cdot \boldsymbol
B&=&0\hfill\null \end{matrix}
\end{equation}

As can be seen from this, the vector Laplacian establishes a Helmholtz
decomposition of the vector field $\boldsymbol F$ into an irrotational or curl
free component $\boldsymbol E$ and a divergence free component $\boldsymbol B$,
along with associated potential fields $\varphi$ and $\boldsymbol A$, all from a
single equation c.q. operator. 

Thus we have shown that the mathematical definitions for potential fields are
hidden within the Laplace operator c.q. the fundamental theorem of vector
calculus c.q. the second order spatial derivative, which has tremendous
consequences for both the analytical analysis of the electromagnetic field as
well as fluid dynamics vector theory. The symmetry between the fields thus
defined is fundamental and has been mathematically proven to be correct, so it
is vital to maintain this fundamental symmetry in our physics equations. 

So far, we have considered the general case, which is valid for any given vector
field F. In the following, we will use the m subscript to refer to the
electromagnetic domain along Maxwell's equations, while the f subscript is used
for the fluid dynamics domain. 

In Maxwell's equations, the curl of the electric field Em is defined by the
Maxwell-Faraday equation:

\begin{equation}
\nabla \times \boldsymbol E_m=-\frac{\partial \boldsymbol B_m}{\partial t},
\end{equation}

which is obvious not equal to zero for electromagnetic fields varying with time
and therefore the dynamic Maxwell equations cannot be second order spatial
derivatives of any vector field Fm as defined by the Laplacian. 

Herewith, we have shown that no vector field Fm exists for which Maxwell's
equations are the second order spatial derivative and therefore Maxwell's
equations do not satisfy the vector Laplace equation. The end result of this is
that while the solutions of Laplace's equation are all possible harmonic wave
functions, with Maxwell's equations there is only one resulting wave equation
which defines a ``transverse'' wave, whereby the Em and Bm components are always
perpendicular with respect to one another. This is also the reason why no
separate wave equations can be derived for the ``near'' and ``far'' fields.

Furthermore, in Maxwell's equations, the two potential fields which are used
with Helmholtz's theorem are the electrical potential $\varphi $m and the
magnetic vector potential Am, which are defined by the equations
\cite{Feynman1964}:


\begin{equation}
\begin{matrix}\hfill \boldsymbol B_m&=&\nabla \times \boldsymbol A_m\hfill\null \\\hfill \boldsymbol
E_m&=&-\nabla \varphi _m-\frac{\partial \boldsymbol A_m}{\partial t}\hfill\null \end{matrix}
\end{equation}
where Bm is the magnetic field and Em is the electric field.

The Helmholtz theorem can also be described as follows. Let A be a solenoidal
vector field and $\varphi $ a scalar field on R3 which are sufficiently smooth
and which vanish faster than 1/r2 at infinity. Then there exists a vector field
F such that:

\begin{equation}
\nabla \boldsymbol F=\varphi \text{and}\nabla \times \boldsymbol F=\boldsymbol A
\end{equation}
and if additionally, the vector field F vanishes as r $\rightarrow $ ${\infty}$,
then F is unique

\footnote{ David J. Griffiths, Introduction to Electrodynamics,
Prentice-Hall, 1999, p. 556.\par }.

Now let us consider the units of measurement involved in these fields, whereby
the three vector operators used all have a unit of measurement in per meter
[/m]. The magnetic field Bm has a unit of measurement in Tesla [T], which is
defined in SI units as [kg/s2{}-A]. So, for the magnetic vector potential Am we
obtain a unit of [kg-m/s2{}-A] and for dAm/dt we obtain a unit of [kg-m/s3{}-A].
The electric field Em has a unit of measurement in volt per meter, which is
defined in SI units as [kg-m/s3{}-A], which matches that for dAm/dt. So, for the
electric scalar potential $\varphi $m we obtain a unit of [kg-m2/s3{}-A].

However, neither the units of measurement for Em and Bm are the same, nor are
the units of measurements for $\varphi $m and Am. This is in contradiction with
Helmholtz's theorem, which states that a vector field Fm exists that should have
a unit of measurement equal to that of $\varphi $m and Am times meters or that
of Em and Bm times meters squared.

Thus, we have shown that Maxwell's equations are in contradiction with
Helmholtz's theorem as well, which means that the potential fields defined by
Maxwell are mathematically inconsistent and should therefore be revised.

It can be shown

\footnote{ Stowe, P. and Mingst, B., The Atomic Vortex Hypothesis, a Forgotten
Path to Unification, http://vixra.org/abs/1310.0237 \par } 

that by using the 19th Century's atomic vortex postulate in combination with a
superfluid model for the medium, it is possible to construct a single simple
integrated model which covers all major branches of physics including kinetic,
fluid, gravitation, relativity, electromagnetism, thermal, and quantum theory.
With this method, it can also be shown that anomalous observations such as
Pioneer's drag and the electron's magnetic moment can be directly accounted for
by the model. Furthermore, with this model all units of measurements are defined
in terms of just three fundamental units of measurement: mass, length, and time. 

It should be noted that there are two distinct levels in this model, with each
playing their own role. The first consists of basic media quanta, which forms a
superfluid model for the medium itself. The second describes vortices within the
fluid, which forms a particle model on top of the medium model. The lower base
level is assumed to be an (if not ideal, nearly so) in-viscous superfluid system
obeying the defined rules of basic kinetic theory and that is the model this
paper is originally based on, which means that the equations presented in this
paper do not depend on the higher level Atomic Vortex Hypothesis based model.
However, during the course of this work it became clear that viscosity plays a
crucial role in our model, which has as consequence that an in-viscous
superfluid model is insufficient to describe the behavior of the medium.

Of course, a (viscous) superfluid model can also be described in vector notation
and since this model essentially describes a fluid/gas like medium, we can apply
continuum mechanics fluid dynamics vector calculus methods to re-derive the
Maxwell equations from the basic model. As is common practice in continuum
mechanics fluid dynamics vector theory, we can describe its dynamic behavior by
working with the medium's flow velocity vector field

\footnote{ Prieve, D.C., A
Course in Fluid Mechanics with Vector Field Theory. Department of Chemical
Engineering, Carnegie Mellon University,
https://archive.org/details/ACourseInFluidMechanicsWithVectorFieldTheory\par }
v, with v representing the local average bulk flow velocity. 

It should be noted that because we use continuum mechanics, the equations
presented in this paper are independent on the detailed description of the
constituents of the medium itself and that there is a lower limit with respect
to scale below which the medium can no longer be considered as a continuum. In
that case, the model is no longer applicable, which is a well-known limitation
of continuum mechanics. The Knudson number can be used to estimate this limit. 

Within the fluid dynamics domain, a scalar potential field $\varphi $f and a
vector potential field [D835?][DC00?]f are generally described for an
incompressible fluid (${\nabla}{\cdot}$vf = 0) with a flow velocity field vf as
follows

\footnote{ Richardson, S., \& Cornish, A. (1977). Solution of three-dimensional
incompressible flow problems. Journal of Fluid Mechanics, 82(2), 309-319.
doi:10.1017/S0022112077000688
https://pdfs.semanticscholar.org/9344/48b028a3a51a7567c2b441b5ca3e49ebb85c.pdf
\par } (eq. 17-19):

\begin{equation}
\boldsymbol v_f=\nabla \varphi _f+\nabla \times \boldsymbol A_f
\end{equation}
where the velocity potential $\varphi $f is a scalar potential field, satisfying
the Laplace equation:

\begin{equation}
\nabla ^2\varphi _f=0
\end{equation}
and the vorticity potential [D835?][DC00?]f is a solenoidal (i.e.
${\nabla}$.${\cdot}$Af = 0) vector potential field satisfying the Poisson
equation:

\begin{equation}
\nabla ^2\boldsymbol A_f=-\nabla \times (\nabla \times \boldsymbol A_f)=-\boldsymbol{\omega }_v,
\end{equation}
where $\omega $v=${\nabla}${\texttimes}vf  is the velocity vorticity field.

However, with this definition, the potential fields are not uniquely defined and
the boundary conditions on $\varphi $f and [D835?][DC00?]f depend on the nature
of the flow at the boundary of the flow domain and on the topological properties
of the flow domain, respectively.  

We can can attempt to resolve this problem for the general case of a fluid that
is both compressible and rotational by defining a compressible irrotational
velocity field Ef for the scalar potential $\varphi $f and an incompressible
solenoidal velocity field Bf and associated vorticity field $\omega $ for the
vector potential [D835?][DC00?]f using the Helmholtz decomposition and negating
the commonly used definition for the velocity potential $\varphi $f:

\begin{equation}
\boldsymbol v_f=-\nabla \varphi _f+\nabla \times \boldsymbol A_f=\boldsymbol E+\boldsymbol B
\end{equation}
\begin{equation}
\begin{matrix}\hfill \boldsymbol E_f&=&-\nabla \varphi _f\hfill\null \\\hfill \boldsymbol B_f&=&\nabla \times
\boldsymbol A_f\hfill\null \\\hfill \boldsymbol{\omega }&=&\nabla \times \boldsymbol B_f\hfill\null
\end{matrix}
\end{equation}
This way, the Ef and Bf fields describe flow velocity fields with a unit of
measurement in [m/s] and both the velocity potential and the velocity vorticity
potential describe fields with a unit of measurement in meters squared per
second [m2/s]. However, the primary vector field Ff thus has a unit of
measurement in [m3/s], which describes a vector field for a volumetric flow rate
or volume velocity. This can be considered as the flow velocity vector field vf
times a surface S perpendicular to vf with a surface area proportional to h2
square meters [m2], with h the physical length scale in meters [m]. This results
in the zero vector when taking the limit for the length scale h to zero, which
is obviously problematic.

So far, we have considered the general mathematical case for the Helmholtz
decomposition of any given vector field F as well as its common use in both the
electrodynamics and the fluid dynamics domains, whereby we encountered a number
of problems.  In order to resolve these problems and avoid confusion with the
various fields used thus far, let us first introduce a new set of fields along
equation (5): 

\begin{equation}\label{seq:refText15}
\begin{matrix}\hfill \Pi &=&\nabla \cdot \boldsymbol C\hfill\null \\\hfill \boldsymbol{\Omega }&=&\nabla \times
\boldsymbol C\hfill\null \\\hfill \boldsymbol L&=&-\nabla \Pi \text{ = }-\nabla (\nabla \cdot \boldsymbol
C)\hfill\null \\\hfill \boldsymbol R&=&\nabla \times \boldsymbol{\Omega }\text{ = }\nabla \times (\nabla \times
\boldsymbol C),\hfill\null \end{matrix}
\end{equation}
where C is our primary vector field, $\Pi $ is the scalar potential or pressure,
$\Omega $ is the vector potential or angular pressure, L is the longitudinal or
translational force density and R is the rotational or angular force density.
Hereby, $\Pi $ and $\Omega $ have a unit of measurement in Pascal [Pa] or
Newtons per square meter [N/m2] and L and R are in Newtons per cubic meter
[N/m3]. C is in Newtons per meter [N/m] or kilograms per second squared [kg/s2],
thus representing an as of yet undefined quantity. Further down, we will see
that for the medium this unit corresponds to the Ampere, hence the choice for
using the symbol C for ``current''.

Let us now consider the 3D generalization of Newton's second law for a substance
with a certain mass density, expressed in densities or per unit volume:

\begin{equation}
\boldsymbol f_n=\rho \boldsymbol a=\rho \frac{d\boldsymbol v}{\mathit{dt}}=-{\nabla}\Pi ,
\end{equation}
with fn the force density in [N/m3], $\rho $ the mass density of the substance,
v the velocity field, a the acceleration field and $\Pi $ the pressure or scalar
potential field in Pascal [Pa], defined as the divergence of some primary field
C. Since C should exist according to the Helmholtz decomposition and should have
a unit of measurement in [kg/s2] or [N/m], we can define C as follows:  

\begin{equation}
\boldsymbol C=\eta \boldsymbol v,
\end{equation}
with $\eta $ the viscosity of the substance in [kg/m-s]. This way, we obtain a
full 3D generalization of Newton's second law per unit volume, describing not
only a longitudinal force density field L but also a rotational or angular force
density field R: 

\begin{equation}\label{seq:refText18}
\rho \frac{d\boldsymbol v}{\mathit{dt}}=-{\nabla}^2\boldsymbol C=-{\nabla}^2\eta \boldsymbol
v=-(\boldsymbol L+\boldsymbol R).
\end{equation}
This definition also allows us to work with the vector wave equation(3):

\begin{equation}
{\nabla}^2\boldsymbol C\text{ + }k^2\boldsymbol C=\nabla \nabla \cdot \boldsymbol C\text{ - }\nabla \times
\nabla \times \boldsymbol C\text{ + }k^2\boldsymbol C=0.
\end{equation}
This is a full 3D vector wave equation, in contrast to the complex wave function
that is often used in Quantum Mechanics. With wave functions there are only two
axis, the real and the imaginary, which is simply insufficient to fully describe
phenomena in three dimensions. In other words: current Quantum Mechanics
theories lack the required dimensionality in order to be capable of fully
describing the phenomena and are therefore incomplete. 

When we divide equation (18) by mass density $\rho $, we obtain the velocity
diffusion equation:

\begin{equation}\label{seq:refText20}
\boldsymbol a=\frac{d\boldsymbol v}{\mathit{dt}}=-{\nabla}^2k\boldsymbol v=-{\nabla}^2\Lambda ,
\end{equation}
with a the acceleration field in [m/s2], k the diffusivity or kinematic
viscosity, defined by:

\begin{equation}
k=\frac{\eta }{\rho },
\end{equation}
and $\Lambda $ the volumetric acceleration field, defined by: 

\begin{equation}
\Lambda =k\boldsymbol{\mathit{v.}}
\end{equation}
This results in the diffusivity for the medium k having a unit of measurement in
meters squared per second [m2/s] and a value equal to light speed c squared
($c^2$), so there is a per second [/s] difference in the unit of measurement,
suggesting that in our current models the dimensionality of certain quantities
is off by a per second. As we shall see, this has profound consequences for our
understanding of physical reality including the mass-energy equivalence
principle. 

This per second difference in units of measurement suggests that in our current
models there are a number of problems involving time derivatives that have not
been properly accounted for. When we consider that the solutions of the vector
wave equation are harmonic functions, characterized by sine and cosine functions
of time, it becomes clear how these problems could have arisen. Since the cosine
is the time derivative of the sine function and vice versa, there is only a
phase difference between the two. When we consider that all known particles
adhere to the wave-particle duality principle and have characteristic
oscillation frequencies that are very high, it becomes clear that the quantum
scale phase differentials between a force acting upon a particle and the
resulting (time delayed) acceleration of that particle are virtually
undetectable at the macroscopic level.  

Note that with this diffusion equation, the only units of measurement are the
meter and the second, which means that we have succeeded in separating the
dynamics over space and time from the substance (mass density) that's being
diffused over space and time. In other words: with this diffusion equation we
have described the quantum characteristics of spacetime itself. 

Analogous to equation (20), we can also define a second order diffusion equation:

\begin{equation}
\boldsymbol j=\frac{d\boldsymbol a}{\mathit{dt}}=-k{\nabla}^2\boldsymbol a=-k\nabla ^2(-k\nabla
^2\boldsymbol v),
\end{equation}
which we can work out further by multiplying by mass density $\rho $ to define
the radiosity or intensity field I in Watts per square meter [W/m2],
representing a heat flux density:

\begin{equation}
\rho \boldsymbol j=\rho \frac{d\boldsymbol a}{\mathit{dt}}=-\rho k\nabla ^2\boldsymbol a=-\nabla
^2\boldsymbol I=-\eta \nabla ^2\boldsymbol a=-k\nabla ^2(\boldsymbol L+\boldsymbol R),
\end{equation}
or:

\begin{equation}
\boldsymbol I=k(\boldsymbol L+\boldsymbol R)
\end{equation}
From this, we can derive additional fields analogous to equation (15), which
results in fields representing power density in Watts per cubic meter [W/m3] for
the first spatial derivatives and jerk j times mass density in [N/m3{}-s3] for
the second spatial derivatives and thus we find that the spatial derivatives of
the intensity field I are the time derivatives of the corresponding spatial
derivatives of our primary field C.

The process of taking higher order derivatives can be continued indefinitely,
whereby for harmonic solutions we end up with the same results over and over
again, resulting in only a phase differential between subsequent results.  

An interesting detail is that the intensity field I can also be defined as:

\begin{equation}\label{seq:refText26}
\boldsymbol I=-\kappa \boldsymbol v,
\end{equation}
with $\kappa $ the modulus or elasticity in [Pa] or [kg/m-s2], which has a unit
of measurement that differs by a per second [/s] from the unit of measurement
for viscosity $\eta $ in [Pa-s] or [kg/m-s].

This reflects the difference between elastic forces and viscous (shear) forces,
namely that the  elastic force is proportional to the amount of deformation,
while the viscous one is proportional to the rate of deformation. 


In our work on the implementation of a finite time difference method for the
new model, 

Ref: \url{https://github.com/l4m4re/fdtd/blob/master/fdtd/operators.py}

we found that such a method is implemented on a "Staggered Yee Grid",
essentially a computable discrete grid of vectors and scalars, located at
specific geometrical locations on the grid. This learns that the div, grad and
curl operators work on vectors and scalars, located at specific geometric
locations on the grid, which implies that there are actually two, fundamentally
different curl operators.

As GitHub Copilot puts it:

\begin{quote}
  
Q: are these correct?

Yes, these functions are correct and consistent with a staggered Yee grid. The
curl\_point function computes the point curl of a vector field F located on the
grid points, which are the edges of the grid cells and are stored at integer
grid points in the arrays. The output vector field curl\_F is located on the grid
points and is therefore a point field.

The `curl\_surface` function computes the surface curl of a vector field F
located on the faces of the grid cells that are perpendicular to the x, y, and z
axes, and are stored at half-integer grid points in the arrays, and is therefore
a surface field.

These functions are consistent with the staggered Yee grid, which is commonly
used in electromagnetics and fluid dynamics simulations.

\end{quote}


This reflects the difference between elastic forces and viscous (shear) forces,
namely that the  elastic force is proportional to the amount of deformation,
while the viscous one is proportional to the rate of deformation. 

So, it appears
we can conclude that in physical reality there are no actual static (elastic)
forces (at the quantum level) and that deep down there are only dynamic forces
and interactions which are governed by the velocity diffusion equation(20),
whereby what we observe as static forces are in reality the time derivatives of
fundamentally viscous forces.

% This is incorrect. thus:



The elastic force is proportional to the amount of deformation, whereby the
deformation is the time integral of the velocity field over the surface of
a certain volume. This handles about the longitudinal half of the Helmholtz
decomposition, of which the resulting linear force density field is defined
as the gradient of a scalar potential field andis located at a point at the 
center of the volume, a cube in the case of a grid cell.

The viscous force is proportional to the rate of deformation, whereby the
deformation is dynamic and involves a velocity gradient around c.q. due to a
number of irrotational, hollow core superfluid vortices, each with a circulation
equal to the quantum circulation constant $k$. 

Thus, the viscous force is an angular phenomenon, characterized by a vector
potential, while the elastic force is a linear phenomenon, characterized 
by a scalar potential.






This brings us to the mass energy equivalence principle:

\begin{equation}
E=\mathit{mc}^{2,}
\end{equation}
which can now alternatively be formulated by:

\begin{equation}\label{seq:refText28}
L=mk,
\end{equation}
with L the angular momentum of a particle in [J/s] or [kg-m2/s] relative to its center of mass, m the mass of the
particle and k the diffusivity or kinematic viscosity. 

This can be related to the unusual behavior of superfluids such as 3He, which
spontaneously creates quantized vortex lines when the container holding the
liquid is put into rotation

\footnote{  Olli V. Lounasmaa and Erkki Thuneberg,
Vortices in rotating superfluid 3He. PNAS July 6, 1999 96 (14) 7760-7767;
https://doi.org/10.1073/pnas.96.14.7760 \par }, 

thus forming a quantum vortex.
This is a hollow core around which the superfluid flows along an irrotational
vortex pattern (i.e. $\nabla \times \boldsymbol v=0$). This flow is
quantized in the sense that the circulation takes on discrete values

\footnote{
Whitmore, S C, and Zimmermann, W Jr., Observation of Quantized Circulation in
Superfluid Helium. United States: N. p., 1968. Web. doi:10.1103/PhysRev.166.181.
\par }. 

The quantum unit of circulation or quantum circulation constant is h/m,
with h Planck's constant in [J-s] or [kg-m2/s] and m the mass of the superfluid
particles in [kg].  Note that Planck{}'s constant has a unit of measurement
representing angular momentum.

For the medium, we can equate this quantum circulation constant to k, the
diffusivity or kinematic viscosity, which we can now also refer to as the
quantum circulation constant, and thus we can compute the mass of an elemental
aether particle along:

\begin{equation}
m_{\mathit{elemental}}=\frac h k,
\end{equation}
which computes to approximately 7.372e-51 kg, about 20 orders of magnitude
lighter than the electron. 

When we compute the Compton wavelength for such a particle, we obtain the value
of the speed of light c, but with a unit of measurement in meters [m] rather
than velocity [m/s], while it's associated frequency computes to 1 Hertz [Hz].
Since the Compton wavelength of a particle is equal to the wavelength of a
photon whose energy is the same as the mass of that particle along the
mass-energy equivalence principle, this puts serious question marks to the
mass-energy equivalence principle in favor of our alternative in equation (28),
whereby we conclude that the quantization that is observed in physics is not a
quantization of mass/energy, but one of angular momentum. And since angular
momentum is represented by the magnetic field, we can also conclude that it's
the magnetic field that is quantized and that magnetic field lines are actually
irrotational hollow core vortices in a superfluid medium with a circulation
equal to the quantum circulation constant k. 




\section{Citations}
\markright{Arend Lammertink. Revision of Maxwell's equations}

A single citation is here: \cite{eddy}. Multiple citations are as follows
\cite{bondi,Pez,La2}. A citation containing a comment is \cite[see p.\,5]{eddy}

%%%%%%%% the \cite{eddy} command generates citation number proceeded from
%%%%%%%% the label \bibitem{eddy} in the bibliography list


\markright{Arend Lammertink. Revision of Maxwell's equations}
\section{Equations}
\markright{Arend Lammertink. Revision of Maxwell's equations}

Here is a manual-numbered equation
$$
r\,= \sqrt{dx^{2} + dy^{2} + dz^{2}}.
\eqno \mbox{(1.1)}
$$

Here is an automatic-numbered equation
\begin{equation}
r\,= \sqrt{dx^{2} + dy^{2} + dz^{2}}.
\end{equation}

Here is an unnumbered equation
$$
r\,= \sqrt{dx^{2} + dy^{2} + dz^{2}}.
$$


Here is a double-line equation, typeset to the left side
$$
\begin{array}{ll}
%
\displaystyle
ds^{2}\,= L(r)dt^{2} - M(r)(dx^{2} + dy^{2} + dz^{2}) -\\[+8pt]  % 1st row
%
\displaystyle
- N(r)(xdx + ydy + zdz)^{2}, \\% 2nd row
\end{array}
$$


Here are automatic-designed brackets
\begin{equation}
\left( \frac{\mathrm{D} N^\alpha}{ds}\right),\quad
\left[ \frac{\mathrm{D} N^\alpha}{ds}\right],\quad
\left\{ \frac{\mathrm{D} N^\alpha}{ds}\right\},
\end{equation}
where you need in an ``empty'' bracket, if you feel to insert one-side brackets.
For instance: $\left( \right.$.



Here are hand-designed brackets
\begin{equation}
\bigl( \frac{\mathrm{D} N^\alpha}{ds}\bigr),\quad
\Bigl( \frac{\mathrm{D} N^\alpha}{ds}\Bigr),\quad
\biggl( \frac{\mathrm{D} N^\alpha}{ds}\biggr) , 
\label{gensol}
\end{equation}
where is no need to insert an ``empty'' bracket, so you can mere type
\begin{equation}
\frac{\mathrm{D} N^\alpha}{ds} =
\Bigl\{ K^\alpha ; 0.
\end{equation}


%%%%%%%% [+8pt] is intendation following after the row
%%%%%%%% \displaystyle is normalsize in the fractions

%%%%%%%% this equation will be typeset to right, if use
%%%%%%%% {rr} argument istead {ll} in the preamble of the array

%%%%%%%% there is so many rows available as you feel

\markright{Arend Lammertink. Revision of Maxwell's equations}
\section{Formulae in text}
\markright{Arend Lammertink. Revision of Maxwell's equations}

Take operators in the \,{}\, brackets in the inline formulae, for compact
typing: \,{=}\, gives $w \,{=}\, c^2 $. Write down \dots \ instead of ...


\markright{Arend Lammertink. Revision of Maxwell's equations}
\section{Items}
\markright{Arend Lammertink. Revision of Maxwell's equations}


An unnumbered item containing bullets is:
\begin{itemize}
\item The most general metric
\item The most general metric
\item The most general metric
\end{itemize}


Here is an unnumbered item:
\begin{itemize}
\item [] The most general metric
\item [] The most general metric
\item [] The most general metric
\end{itemize}


An Arabic-numbered item:
\begin{enumerate}
\item The most general metric
\item The most general metric
\item The most general metric
\end{enumerate}


A your-style numbered item:
\begin{itemize}
\item [A1] The most general metric
\item [A2] The most general metric
\item [A3] The most general metric
\end{itemize}


A double-level item (it is numbered, a sample):
\begin{enumerate}
\item The most general metric
  \begin{enumerate}
  \item The most general metric
  \item The most general metric
  \item The most general metric
  \end{enumerate}
\item The most general metric
\item The most general metric
\item The most general metric
\end{enumerate}


\markright{Arend Lammertink. Revision of Maxwell's equations}
\section{References to text pages}
\markright{Arend Lammertink. Revision of Maxwell's equations}


If you like to refer a numbered formula in the {equation} environment, input
\label{nickname-of-the-formula} into the formula, so you will need to type
(\ref{nickname-of-the-formula}) in the text instead of (12), for instance. Such
reference will automatically be changed keeping the real number of the
reference, if you reorder/remove/add formulae.

It works in only the {equation} environment --- auto numbered formulae.



\markright{Arend Lammertink. Revision of Maxwell's equations}
\section{Cross-references}
\markright{Arend Lammertink. Revision of Maxwell's equations}


Insert \label{myidea} in your text, then you have that page number where your
label \pageref{myidea} appeared. For instance:

The general equation, see formula (\ref{gensol}) in page~\pageref{gensol}, is
very good.

Don't use two or more same labels in the same document!



\markright{Arend Lammertink. Revision of Maxwell's equations}
\section{Brackets, dividing paragraphs, etc.}
\markright{Arend Lammertink. Revision of Maxwell's equations}


The commands `` and '' produce open-closed brackets: ``notation''.

Instead of \par one uses empty space(s) between paragraphs, because it is more visible.

Any sequence following a formula starts new paragraph.

If a paragraph ends by a formula, the next paragraph starts from the first line indented.

Text and space in formulae:
$$
\mbox{here is a text in this formula}\quad
\mbox{small space}\qquad \mbox{big space}
$$


\markright{Arend Lammertink. Revision of Maxwell's equations}
\section{Spaces and dashes}
\markright{Arend Lammertink. Revision of Maxwell's equations}


Einstein-Infeld, space-like, Bohr-like include single dash.

Page numbers 3--27 include double dash.

Thin spaces in text: v.\,13, no.\,24.

American long dash is---like this case.

British long dash is --- like this one.

We assumed the British case in our Journal.



\markright{Arend Lammertink. Revision of Maxwell's equations}
\section{Normal size inside fractions}
\markright{Arend Lammertink. Revision of Maxwell's equations}


Use ``displaystyle'' command before every line:
\begin{equation}
\begin{array}{ll}
\displaystyle
R_{p}(r) = \sqrt{\sqrt{C(r)}\left(\sqrt{C(r)} - \alpha\right)} + \\[+12pt]
\displaystyle
+ \;\alpha\ln\left|\frac{\sqrt{\sqrt{C(r)}} + \sqrt{\sqrt{C(r)} - 
\alpha}}{\sqrt{\alpha}}\right|.
\end{array}
\end{equation}

Compare it with follows
\begin{equation}
\begin{array}{ll}
%  \displaystyle
R_{p}(r) = \sqrt{\sqrt{C(r)}\left(\sqrt{C(r)} - \alpha\right)} + \\[+12pt]
%  \displaystyle
+ \;\alpha\ln\left|\frac{\sqrt{\sqrt{C(r)}} + \sqrt{\sqrt{C(r)} - 
\alpha}}{\sqrt{\alpha}}\right|.
\end{array}
\end{equation}




\section*{Acknowledgements}

This work would not have been possible without the groundbreaking work of Paul
Stowe and Barry Mingst, who succeeded in integrating the gravitational domain
with the electromagnetic domain within a single superfluid based model. It is
this integration that resolves the classic problems associated with aether based
theories, namely that because the gravitational field was considered to be
separate from the electromagnetic domain, the movements of planetary bodies
would necessarily result in measurable disturbances in the medium. When no such
disturbances were found in the Michelson-Morley experiment, the aether
hypothesis was considered as having been disproven. But because the
gravitational force is now considered to be a force caused by longitudinal
dielectric waves, which propagate trough the medium, this argument no longer
applies. And therewith there is no longer any reason to disregard an aether
based theory as a basis for theoretical physics.

This work would also not have been possible without the work of Eric Dollard,
N6KPH, who replicated a lot of Tesla's experiments in the 1980's. It is his
demonstrations and analysis of Tesla's work that enabled the very consideration
of an aether based theory as an alternative to the current theoretical model.


%
\begin{flushright}\footnotesize
Submitted on Month Day, Year / Accepted on Month Day, Year
\end{flushright}


\begin{thebibliography}{99}\footnotesize


% electrically short coaxial cables
\bibitem{Kuehn2019} K\"uhn, Steffen. (2019) Electronic data transmission at
three times the speed of light and data rates of 2000 bits per second over long
distances in buffer amplifier chains. DOI: 10.13140/RG.2.2.33988.78721/1
https://www.researchgate.net/publication/335677198

\bibitem{Kuehn2020} K\"uhn, Steffen. (2020). General Analytic Solution of the
Telegrapher's Equations and the Resulting Consequences for Electrically Short
Transmission Lines. Journal of Electromagnetic Analysis and Applications. 12.
71-87. 10.4236/jemaa.2020.126007. 


% microwaves

\bibitem{Ranfagni2004} Ranfagni, A \& Mugnai, Daniela \& Ruggeri, Rocco. (2004).
Unexpected behavior of crossing microwave beams. Physical review. E,
Statistical, nonlinear, and soft matter physics. 69. 027601.
10.1103/PhysRevE.69.027601.

\bibitem{Allaria2011} Allaria, Enrico \& Mugnai, Daniela \& A.ranfagni, \& C.ranfagni,.
(2011). Unexpected behavior of crossing of microwave and optical beams. Modern
Physics Letters B. 19. 10.1142/S0217984905009274.
\href{https://www.researchgate.net/publication/263801213}{https://www.researchgate.net/publication/263801213}

\bibitem{Agresti2015} Agresti, Alessandro \& Cacciari, Ilaria \& Ranfagni, A \&
Mugnai, Daniela \& Mignani, Roberto \& Petrucci, Andrea. (2015). Two possible
interpretations of the near-field anomaly in microwave propagation. Results in
Physics. 5. 196. 10.1016/j.rinp.2015.08.002.\\
\url{https://www.researchgate.net/publication/282609282}

\bibitem{Mojahedi2000} Mojahedi, Mohammad \& Schamiloglu, Edl \& Hegeler, Frank
\& Malloy, Kevin. (2000). Time-domain detection of superluminal group velocity
for single microwave pulses. Physical review. E, Statistical physics, plasmas,
fluids, and related interdisciplinary topics. 62. 5758-66.
10.1103/PhysRevE.62.5758. https://www.researchgate.net/publication/12238975 

\bibitem{Musha2019} Musha. (2019). Superluminal Speed of Photons in the
Electromagnetic Near-Field. Recent Adv Photonics Opt 2(1):36-39.\\
\url{https://scholars.direct/Articles/photonics-and-optics/rapo-2-007.php}

\bibitem{Wang2003} Wang \& Xiong. (2003). Superluminal Behaviors of
Electromagnetic Near-fields. https://arxiv.org/pdf/physics/0311061.pdf 

\bibitem{Walker2006} Walker. (2006). Superluminal Electromagnetic and
Gravitational Fields Generated in the Nearfield of Dipole Sources.\\
\url{https://arxiv.org/abs/physics/0603240}

\bibitem{Solemino2014} Solemino, De Lisio \& Altucci. (2014) Superluminal
behavior in wave propagation: a famous case study in the microwave region.\\
\url{
http://www.societanazionalescienzeletterearti.it/pdf/161\_nota\_Solimeno-Altucci\_19-12-2014.pdf
}    


\bibitem{Ranfagni2006} Ranfagni, A. \& Fabeni, P. \& Pazzi, G. \& Ricci, A. \&
Trinci, R. \& Mignani, Roberto \& Ruggeri, Rocco \& Cardone, F. \& Agresti,
Alessandro. (2006). Observation of Zenneck-type waves in microwave propagation
experiments. Journal of Applied Physics. 100. 024910 - 024910.
10.1063/1.2212307. https://www.researchgate.net/publication/224453964 

\bibitem{Walker2000} Walker, William. (2000). Experimental Evidence of
Near-Field Superluminally Propagating Electromagnetic Fields.
10.1007/0-306-48052-2\_18. https://arxiv.org/pdf/physics/0009023.pdf



% optical fibers
\bibitem{Gonzalez2005} Gonzalez-Herraez, Miguel \& Song, Kwang-Yong \&
Th\'evenaz, Luc. (2005). Optically Controlled Slow and Fast Light in Optical
Fibers Using Stimulated Brillouin Scattering. Applied Physics Letters. 87.
081113 - 081113. 10.1063/1.2033147. 


\bibitem{Stenner2016} Stenner, M.D., Gauthier, D.J. \& Neifeld, M.A., The speed
of information in a `fast-light' optical medium, Nature. 425. 695-8.
10.1038/nature02016. 


\bibitem{Thevenaz2008} Th\'evenaz, Luc. (2008). ``Achievements in slow and fast
light in optical fibres''. 1. 75 - 80. 10.1109/ICTON.2008.4598375. 

\bibitem{Wang2000} Wang, L. \& Kuzmich, A \& Dogariu, Arthur. (2000). Gain-assisted
superluminal light propagation. Nature. 406. 277-9. 10.1038/35018520. 



% other methods

\bibitem{Ardavan2004} Ardavan, A, Singleton, J, Ardavan, H,  Fopma, J, Halliday,
D and Hayes, W. (2004). Experimental demonstration of a new radiation mechanism:
emission by an oscillating, accelerated, superluminal polarization current.
arXiv:physics/0405062 https://arxiv.org/abs/physics/0405062

\bibitem{Niang2018} Niang, Anna \& de Lustrac, Andr\'e \& Burokur, Shah Nawaz.
(2018). Broadband Superluminal Transmission Line with Non-Foster Negative
Capacitor. 10.1049/cp.2018.1179.\\
\url{https://www.researchgate.net/publication/328410118}

\bibitem{Wheatstone1834} Wheatstone, Charles. (1834). An Account of Some
Experiments to Measure the Velocity of Electricity, and the Duration of Electric
Light. Philosophical Transactions of the Royal Society of London. 124.
10.1098/rstl.1834.0031

\bibitem{Tesla1900} Tesla, N. (1900). Art of transmitting electrical energy
through the natural mediums. US patent No. 787,412. In this patent, Tesla
disclosed a wireless system by which he was able to transmit electric energy
along the earth's surface and he established that the current of his transmitter
passed over the earth's surface with a mean velocity 471,240 kilometers per
second, about $\pi$/2 times the speed of light.





\bibitem{vanVlaenderen2001} K. J. van Vlaenderen and A. Waser, Generalization of
Classical Electrodynamics to Admit a Scalar Field and Longitudinal Waves.
Hadronic Journal, Vol. 24, 2001, pp. 609-628. 

\bibitem{Barrett1993} Barrett, Terence. (1993). Electromagnetic phenomena not
explained by Maxwell's equations. 10.1142/9789814360005\_0002. 

\bibitem{Pinheiro2005} Pinheiro, Mario. (2005). Do Maxwell's Equations Need
Revision? A Methodological Note. Physics Essays. 20. 10.4006/1.3119404. 

\bibitem{Behera2018} Behera, Harihar \& Barik, N.. (2018). A New Set of
Maxwell-Lorentz Equations and Rediscovery of Heaviside-Maxwellian (Vector)
Gravity from Quantum Field Theory. 

\bibitem{Nedic2017} Nedi\'{c}, S., Longitudinal Waves in Electromagnetism.
Towards Consistent Theoretical Framework for Tesla's Eergy and Information
Transmission. INFOTEH-JAHORINA Vol. 16, March 2017. 


\bibitem{Atsukovsky2011} Atsukovsky, V.A., Efirodinamicheskie Osnovy
Elektromagnetizma -- Teoria, Eksperimenty, Vedrenie. Energoatomizdat, 2011, ISBN
978- 5- 283-03317-4. 

\bibitem{Arbab2014} Arbab, Arbab \& Al-Ajmi, Mudhahir. (2014). The Modified
Electromagnetism and the Emergent Longitudinal Wave. Applied Physics Research.
10. 10.5539/apr.v10n2p45. 

\bibitem{Onoochin2019} Onoochin, Vladimir. (2019). Longitudinal Electric field
and the Maxwell Equation. 10.13140/RG.2.2.26478.36163. 

\bibitem{Gray2019} Gray, Robert. (2019). Experimental Disproof of Maxwell and
Related Theories of Classical Electrodynamics. 

\bibitem{Shaw2014} Shaw, Duncan W. (2014). Reconsidering Maxwell's aether.
Physics Essays, Volume 27, Number 4, December 2014, pp. 601-607(7). 


% Introduction

\bibitem{Maxwell1861} Maxwell, J. (1861). On Physical Lines of Force. The
London. Edinburgh. 338-348. 0.1080/14786446108643067.

\bibitem{Elmore2019} Elmore, G., Introduction to the Propagating Wave on a
Single Conductor, http://www.corridor.biz/FullArticle.pdf


\bibitem{Xiong2014} Xiong, Xiaoyan \& Sha, Wei \& Jiang, Li. (2014). Helmholtz
decomposition based on integral equation method for electromagnetic analysis.
Microwave and Optical Technology Letters. 56. 10.1002/mop.28454. 

\bibitem{Oleinik2003} Oleinik, V.P., The Helmholtz Theorem and Superluminal
Signals, Department of General and Theoretical Physics, National Technical
University of Ukraine "Kiev Polytechnic Institute", Prospect Pobedy 37, Kiev,
03056, Ukraine,\\
\url{https://arxiv.org/abs/quant-ph/0311124}

\bibitem{Stratton1941} Julius Adams Stratton, Electromagnetic Theory,
McGraw-Hill Book Company, New York, 1941.

\bibitem{Feynman1964} Feynman, Richard P; Leighton, Robert B; Sands, Matthew
(1964). The Feynman Lectures on Physics Volume 2. Addison-Wesley. ISBN
0-201-02117-X. (pp. 15-15)



% template examples. REMOVE!!

\bibitem{eddy} Eddington A.\,S. The mathematical
theory of relativity. Cambridge University Press,
Cambridge, 1924. % Here is referred book

\bibitem{bondi}  Bondi~H. Negative mass in General 
Relativity. \textit{Review of Modern Physics}, 1957, 
v.\,29\,(3), 423--428. % Here is referred article

\bibitem{Pez} Pezzaglia W. Physical applications of 
generalized Clifford Calculus: Papatetrou equations 
and metamorphic curvature. arXiv: gr-qc/9710027. 
% Here is referred electronic publication

\bibitem{La2}  Lambiase G., Papini G.,  Scarpetta G. Maximal acceleration
corrections to the Lamb shift o/f one electron atoms. \textit{Nuovo Cimento},
v.\,B112, 1997, 1003. arXiv: hep-th/9702130.
% Here is double paper-electronic published article

\end{thebibliography}
\vspace*{-6pt}
\centerline{\rule{72pt}{0.4pt}}
}


\end{document}


