% This file was converted to LaTeX by Writer2LaTeX ver. 1.4
% see http://writer2latex.sourceforge.net for more info
\documentclass{article}
\usepackage[ascii]{inputenc}
\usepackage[T1]{fontenc}
\usepackage[english]{babel}
\usepackage{amsmath}
\usepackage{amssymb,amsfonts,textcomp}
\usepackage{array}
\usepackage{supertabular}
\usepackage{hhline}
\makeatletter
\newcommand\arraybslash{\let\\\@arraycr}
\makeatother
\setlength\tabcolsep{1mm}
\renewcommand\arraystretch{1.3}
\newcounter{Text}
\renewcommand\theText{\arabic{Text}}
\newcommand\boldsubformula[1]{\text{\mathversion{bold}$#1$}}
\title{}
\author{}
\date{2022-11-20}
\begin{document}
Revision and integration of Maxwell's and Navier-Stokes' Equations 

and the origin of quantization in Superfluids and Spacetime itself

Arend Lammertink

Schoolstraat 107, 7471 WV, Goor, The Netherlands,

+316 5425 6426, lamare@gmail.com

{}``I hope that someone will discover a more realistic way, 

or rather a more tangible basis than it has been my lot to find.''

Albert Einstein.

DRAFT revision 33. 





Now let us consider the Cauchy momentum equation without external forces working on the fluid:

\begin{equation}
\rho \frac{d\boldsubformula v}{\mathit{dt}}=-\nabla \cdot \boldsubformula P,
\end{equation}
with  $\boldsubformula P$  the Cauchy stress tensor, which has a unit of measurement in [N/m2] or [Pa] and is a central
concept in the linear theory of elasticity for continuum solid bodies in static equilibrium, when the resultant force
and moment on each axis is equal to zero. It can be demonstrated that the components of the Cauchy stress tensor in
every material point in a body satisfy the equilibrium equations and according to the principle of conservation of
angular momentum, equilibrium requires that the summation of moments with respect to an arbitrary point is zero, which
leads to the conclusion that the stress tensor is symmetric, thus having only six independent stress components,
instead of nine.

In our model, we have only four independent stress components, namely the scalar and vector potentials $\Pi $ and
$\Omega $.

From this momentum equation, the Navier-Stokes equations can be derived, of which the most general one without external
(gravitational) forces is:

\begin{equation}
\rho \frac{D\boldsubformula v}{\mathit{Dt}}=\rho \left(\frac{\partial \boldsubformula v}{\partial t}+\boldsubformula
v\cdot \nabla \boldsubformula v\right)=-\nabla p+\nabla \cdot \left\{\eta \left(\nabla \boldsubformula v+\left(\nabla
\boldsubformula v\right)^T-\frac 2 3\left(\nabla \cdot \boldsubformula v\right)\boldsubformula I\right)+\zeta
\left(\nabla \cdot \boldsubformula v\right)\boldsubformula I\right\},
\end{equation}
with p the pressure, I the identity tensor and  $\zeta $  the volume, bulk or second viscosity. This can be re-written
to:

\begin{equation}
\rho \frac{\mathit{\delta v}}{\mathit{\delta t}}=-{\nabla}p-\rho \left(v\cdot {\nabla}v\right)+\eta {\nabla}\cdot
\left({\nabla}v+\left({\nabla}v\right)^T\right)+\left(\zeta -\frac{2\eta } 3\right)\left({\nabla}\cdot {\nabla}\cdot
v\right)\mathit{I.}
\end{equation}
This is also a second order equation, whereby notably for the viscous term  $\eta
{\nabla}{\cdot}\left({\nabla}v+\left({\nabla}v\right)^T\right)$  the order of the differential operators is reversed
compared to the definition of the second spatial derivative, the vector Laplace operator, while for the elastic term, 
$\left(\zeta -\frac{2\eta } 3\right)\left({\nabla}{\cdot}{\nabla}{\cdot}v\right)I$, the divergence of the divergence is
taken. Also, a separate term is introduced for pressure as well as a convective term,  $\rho
\left(v{\cdot}{\nabla}v\right)$. All this not only causes the complexity of the equations to increase dramatically
while introducing redundancy in the symmetric stress tensor, it also ignores the fundamental symmetry between the
compressible, irrotational components and the incompressible, solenoidal components as prescribed by the Helmholtz
decomposition.

When we compare this with our proposal, we end up with two fundamentally different approaches:

\begin{enumerate}
\item A solution that fundamentally only has viscosity and one fundamental interaction of Nature, yields harmonic
solutions c.q. builds upon deterministic (spherical) harmonics and provides a basis for the observed quantization as
well as a 3D generalization of the currently used under-dimensioned wave functions;
\item A solution that has both viscosity as well as elasticity, the latter of which builds upon Brownian statistical
mechanics and thus requires randomness and is therefore non-deterministic.  
\end{enumerate}
However, with our solution so far, we have lost the description of elastic behavior and thus our model is incomplete.
This again brings us to the unusual behavior of superfluids, which is currently described with a two-fluid
theory\footnote{ Russell J. Donnelly, {\textquotedbl}The two-fluid theory and second sound in liquid
helium{\textquotedbl}, Physics Today 62, 34-39 (2009) https://doi.org/10.1063/1.3248499
https://sites.fas.harvard.edu/\~{}phys191r/References/e1/donnelly2009.pdf \par }. Donnely notes a/o the following: 

\begin{enumerate}
\item In superfluid state, liquid helium can flow without friction. A test tube lowered partly into a bath of helium II
will gradually fill by means of a thin film of liquid helium that flows without friction up the tube's outer wall.
\item There is a thermo-mechanical effect. If two containers are connected by a very thin tube that can block any
viscous fluid, an increase in temperature in one container will be accompanied by a rise in pressure, as seen by a
higher liquid level in that container.
\item The viscous properties of liquid helium lead to a paradox. The oscillations of a torsion pendulum in helium II
will gradually decay with an apparent viscosity about one-tenth that of air, but if liquid helium is made to flow
through a very fine tube, it will do so with no observable pressure drop---the apparent viscosity is not only small, it
is zero!
\end{enumerate}
He also describes ``second sound'', fluctuations of temperature, which according to him ``has turned out to be an
incredibly valuable tool in the study of quantum turbulence'' and provides a condensed summary:

{}``After one of his discussions with London and inspired by the recently discovered effects, Tisza had the idea that
the Bose-condensed fraction of helium II formed a superfluid that could pass through narrow tubes and thin films
without dissipation. The uncondensed atoms, in contrast, constituted a normal fluid that was responsible for phenomena
such as the damping of pendulums immersed in the fluid. That revolutionary idea demanded a ``two-fluid'' set of
equations of motion and, among other things, predicted not only the existence of ordinary sound---that is, fluctuations
in the density of the fluid---but also fluctuations in entropy or temperature, which were given the designation
``second sound'' by Russian physicist Lev Landau. By 1938 Tisza's and London's papers had at least qualitatively
explained all the experimental observations available at the time: the viscosity paradox, frictionless film flow, and
the thermo-mechanical effect.''

This leads to the question of whether or not the effects described by the current ``two-fluid'' theory can also be
described by the (spatial derivatives) of the two related fields we have defined, our primary field [C] and the
intensity field [I], since we already noted that the intensity field [I] does appear to describe elastic behavior (eq
(26)), apart from a per second difference in units of measurement, and that when dealing with harmonic functions of
time, such as those describing elemental particles, it is all too easy to get these time derivatives mixed up, because
there is only a phase differential between the [C] and [I] fields and their respective spatial derivative fields. 

Further, since electric current can be associated with both the curl of the magnetic field as well as with electric
resistance and thus dissipation, it seems clear that rather than associating the absence of dissipation/resistance with
the absence of viscosity, this absence should be associated with the absence of vorticity or turbulence. 

This brings us to the idea of the ``vortex sponge''\footnote{ Edward M. Kelly, {\textquotedbl}Maxwell's Equations as
Properties of the Vortex Sponge{\textquotedbl}, American Journal of Physics 31, 785-791 (1963)
https://doi.org/10.1119/1.1969085  \par },  devised by John Bernoulli in 1736, although in the shape of vortex tubes
rather than ring vortices, which gives rise to elastic behavior of the medium because of momentum transfer effects
arising from the fine-grained vorticity. This idea matches seamlessly to a vortex theory of atoms\footnote{ Van der
Laan, G., ``The Vortex Theory of Atoms.'' Master's thesis, Utrecht University.
https://dspace.library.uu.nl/bitstream/handle/1874/260711/Thesis.pdf \par  }, which was developed after around 1855 a
new type of vortex theory emerged, the so-called `vortex sponge theory'. Instead of viewing atoms as consisting out of
small, separate vortices, this type of theory supposed that the ether was completely filled with tiny vortices. These
tiny, close-packed vortices made up large sponge-like structures, which gave this category of models its name. 

When we consider the basic idea that particles consist of a number (quantized) vortex rings, we would then consider
these to form such a vortex sponge, especially in the case of crystalline materials such as silicon. Therefore, we
would associate a material substance to such a dynamic vortex sponge rather than considering the aether itself to be
universally filled with tiny vortices.  And since in this view there are no stiff, point-like particles that bounce
onto one another in a random manner, we would also do away with Brownian statistics and consider the interactions
between the vortices to occur along harmonic functions of time.

The question then becomes whether or not the static forces we consider at the macroscopic level really are forces along
Newton's third:

\begin{equation}
F=ma,
\end{equation}
or are actually time derivatives thereof, yank, along:

\begin{equation}
Y=mj,
\end{equation}
with j the jerk, the time derivative of acceleration. The derivative of force with respect to time does not have a
standard term in physics, but the term ``yank'' has recently been proposed in biomechanics\footnote{ David C. Lin,
Craig P. McGowan, Kyle P. Blum, Lena H. Ting; Yank: the time derivative of force is an important biomechanical variable
in sensorimotor systems. J Exp Biol 15 September 2019; 222 (18): jeb180414. doi: https://doi.org/10.1242/jeb.180414
\par }.

This would also offer further insight into what inertia, resistance to {\textquotedbl}change{\textquotedbl}, actually
is, because the dynamic viscous forces we have described thus far are proportional to a rate of deformation and
describe something dynamic, whereby there is a continuous flow of mass  along quantized irrotational vortices. In a
way, this can be seen as the opposite of resistance to ``chance'' and could perhaps rather be thought of as conductance
of ``change''.

Let's illustrate that along the rotating superfluid wherein quantum vortexes are formed. Once a certain angular speed
has been established with the rotating container, a certain number of quantum vortices have formed and the system is in
equilibrium. In that situation, the vortices are irrotational and therefore no vorticity nor turbulence and thus no
resistance nor dissipation. In other words: there is a steady-state situation, which could easily be confused with a
{\textquotedbl}static{\textquotedbl} situation, were it not that these the vortex lines are visible.

When we wish to increase the rotation speed of the rotating container, we must exert a
{\textquotedbl}force{\textquotedbl} and thus we introduce turbulence until a new equilibrium is established. This way,
we convert the energy we provide into the rotating superfluid, whereby the steady state situation becomes disturbed and
turbulence is introduced, which results in more quantum vortices forming until eventually the turbulence dies out and a
new equilibrium is established. Thus, from the outside it appears as though the rotating mass in the container resists
change, but in reality it sort of stores {\textquotedbl}change{\textquotedbl} by forming additional vortices until
there is no more turbulence.

Now let us look back at equation (20),  the velocity diffusion equation:

\begin{equation}
\boldsubformula a=\frac{d\boldsubformula v}{\mathit{dt}}=-{\nabla}^2k\boldsubformula v
\end{equation}
with a the acceleration field in [m/s2], k the diffusivity, kinematic viscosity or quantum circulation constant. This is
an equation with only meters and seconds and by dividing by velocity we can find that the time derivative operator can
be related to the second spatial derivative by a single constant:

\begin{equation}
\frac d{\mathit{dt}}=-k{\nabla}^{2,}
\end{equation}
which suggests space and time are indeed closely related.

{}-:-

Intermezzo:  my current ``to do'' list, some cut\&pastes from discussions on researchgate:

Where we are now is that we can describe both the quantum level as well as the superfluid level (quantum phenomena at a
macroscopic scale) with the same equations, only different parameters like mass density and quantum circulation
constant, whereby we find that fundamentally there are only viscous forces.

What we see with superfluids is that when temperature rises (power density increases), elastic behavior emerges, which
is currently described with a two-fluid model.

It seems that this effect can be attributed to the formation of some kind of vortex sponge which gives rise to elastic
behavior. And it also seems we can describe the effects this creates within a continuum by the definition of fields
that are derived from the intensity field [I] rather than our primary field [C].

The fields that can be defined as the second spatial derivatives of [I] have a unit of measurement describing the time
derivative of force density, which would be yank density.

What it appears to come down to is that within current physics Force and Yank have been considered as one and the same
thing, resulting in 3D equations that break the fundamental symmetry demanded by the vector Laplace operator.

So, it seems that there are actually two versions of Newton's law, which have currently been taken together into one:

1) F = m a,

2) Y = m j,

and the challenge thus comes down to figuring out which one of the two applies where.

When we put these quantities in a table:

Action:\ \ kg-m[2C6?]2/s.\ \ Momentum mv: kg-m/s.  \ \ Momentum density: \ \ kg/m[2C6?]2-s 

Energy: \ \ kg-m[2C6?]2/s[2C6?]2\ \ Force  ma: \ \   kg-m/s[2C6?]2  \ \ Force density:  \ \ kg/m[2C6?]2-s[2C6?]2\ \ 

Power:  \ \ kg-m[2C6?]2/s[2C6?]3\ \ Yank  mj: \ \   kg-m/s[2C6?]3  \ \ Yank density:  \ \ kg/m[2C6?]2-s[2C6?]3

it also seems that additional fields can be defined to describe action density and its spatial derivative, momentum
density.

{}-{}-{}-{}-

{\textquotedbl}Sorry, but the root of your special problem is not vector analysis. It is your na\"ive assumption that
you are free in selecting parts of the Navier Stokes equation to handle special problems.{\textquotedbl}

Well, I must confess I was a bit too fast by assuming that because I started from the vector Laplace operator and all
seemed to fit seamlessly, I had solved the puzzle and that the loss of a few independent stress components was nothing
to worry about. So, guilty as charged in that respect.  

However, it was not an exercise in selecting parts of Navier Stokes equations that met my needs, it was an attempt to
derive equivalents of Navier Stokes from vector potential theory and to align these with equivalents of Maxwell and to
derive both from one and the same equation, which turned out to represent Newton's third in 3D.

Since I was familiar with the scalar and vector potentials used in Maxwell and I found that the terms in the vector
Laplace operator can be written out and define fields that establish a Helmholtz decomposition, I became convinced that
this is the way it should be done. When I searched for usage of a vector potential in fluid dynamics, I found this
paper and not much more: 

https://pdfs.semanticscholar.org/9344/48b028a3a51a7567c2b441b5ca3e49ebb85c.pdf 

As I wrote in my paper, I attempted to define a primary vector field for the Laplace operator to work on for these,
since that should exist according to the Helmholtz decomposition. It seemed that all I needed to do was negate the
definition for the scalar potential, but then the unit of measurement for the primary field turned out to be in
[m[2C6?]3/s], denoting a volumetric flow velocity, which results in the null vector when taking the limit of the volume
to zero. So that didn't work out very well.

After a lot of puzzling, I found a solution that involved viscosity, whereby I found that the kinematic viscosity nu
yielded a value equal to light speed squared for the aether, but a mismatch in units of measurement by a per second,
pointing to problems here and there with time derivatives. When I realized that this constant nu can also be seen as
the quantum circulation constant, I became convinced I'm on the right track and that the thus far mysterious properties
of superfluids (quantum phenomena on a macroscopic scale) offer the key to unlocking the mysteries of quantum
mechanics.

{\textquotedbl}The Navier Stokes equations have been derived from momentum conservation. For an incompressible fluid we
get two partial differential equations for density and pressure. For a compressible fluid the energy balance must be
considered, which brings temperature and heat capacity into the game.{\textquotedbl}

It is rather interesting that the fields I derived from my primary field [C] do seem to describe an incompressible fluid
(viscous behavior), while we seem to have lost compressibility and that that should bring temperature and heat capacity
into the game.

My working hypothesis is that temperature is a measure of power density and has a unit of measurement in Watts per cubic
meter [W/m[2C6?]3], but that may not be correct since Stowe (see below) found a unit in [kg-m/s[2C6?]3].  

I found a paper regarding superfluids, wherein it is stated that {\textquotedbl}second sound{\textquotedbl} waves exist
in a superfluid, which incorporates the propagation of fluctuations in temperature:

https://sites.fas.harvard.edu/\~{}phys191r/References/e1/donnelly2009.pdf 

According to Donnely, this phenomena ``has turned out to be an incredibly valuable tool in the study of quantum
turbulence''.  

Thus, we have quite some hints suggesting that elastic behavior, or compressibility, indeed has to do with the (spatial
derivatives of) the intensity field [I] I thus far payed little attention to. I've updated my overview table and also
included another primary field [Q] of which the second spatial derivatives yield momentum density or mass flux, which I
see as another step forward.

What I think is an important detail is that the vector Laplace operator is the 3D generalization of the second spatial
derivative, which would be d[2C6?]2/dx[2C6?]2 in 1D. This means that the 3D complexity of the vector equations we can
define with these three vector fields [Q], [C] and [I], such as the vector wave equation, can be effortlessly reduced
to one dimension to describe phenomena like for instance the mechanical behavior of a long rod or a long thin tube
filled with a fluid.  

{\textquotedbl}The possible approximations are ``incompressibility'', ``ideal gas'', or even ''perfect gas'' with a
constant heat capacity. Another issue are the boundary conditions inclusive external sources and sinks, which define
the geometry of the considered problem. Finally, the initial values are important.

With your approach you stay outside of the terminology used to define Navier Stokes types of problems.{\textquotedbl}

So far, I haven't solved the problem of temperature and black body radiation, but now that I realize the importance of
the intensity field [I] and it's consequence that we have to consider yank rather than force, it seems it is only a
matter of time before we can come full circle.

First of all, it is rather interesting that the gas law also involves quantization denoted by n:

P V = n Kb T, (eq 1)

With T the temperature in Kelvin,

P the pressure,

V the volume,

n the number of quanta,

and Kb Boltzmann's constant.

Second, I found the work of Paul Stowe very interesting, but very hard to comprehend. On the one hand, he managed to
express all the major constants of nature in terms of just 5 constants and expressed all units of measurement in just
three: mass, length and time, while on the other he managed to write it all down in a manner that I found very
confusing, for instance because he refers to charge q as {\textquotedbl}divergence{\textquotedbl} while meaning
{\textquotedbl}divergence of momentum density{\textquotedbl}:

https://vixra.org/pdf/1310.0237v1.pdf 

Nonetheless, valuable insights can be obtained from his work, if only as a starting point for further considerations.
With respect to temperature and the gas law, in his eq. 20 we find a relationship between electrical charge and
Boltzmann's constant: 

Kb = h/(qc), (eq 2)

with q elemental charge and h Planck's constant, which results in the conclusion that the quantization in the gas law is
related to the quantization of the medium, which is governed by the quantum quantization constant nu. While it is nice
that this equation yields the right number, this does not necessarily mean this equation is 100\% correct as written,
but it certainly seems to point in the right direction. 

Another interesting paper on the subject of black body radiation in relation to aether theory is this one by C.K.
Thornhill, which gives a valuable starting point for deriving Planck's law:

https://etherphysics.net/CKT1.pdf 

His main argument:

{\textquotedbl}Another argument against the existence of a physical ethereal medium is that Planck's empirical formula,
for the energy distribution in a black-body radiation field, cannot be derived from the kinetic theory of a gas with
Maxwellian statistics. Indeed, it is well-known that kinetic theory and Maxwellian statistics lead to an energy
distribution which is a sum of Wien-type distributions, for a gas mixture with any number of different kinds of atoms
or molecules. But this only establishes the impossibility of so deriving Planck's distribution for a gas with a finite
variety of atoms or molecules. To assert the complete impossibility of so deriving Planck's distribution it is
essential to eliminate the case of a gas with an infinite variety of atoms or molecules, i .e . infinite in a
mathematical sense, but physically, in practice, a very large variety. The burden of the present paper is to show that
this possibility cannot be eliminated, but rather that it permits a far simpler derivation of Planck's energy
distribution than has been given anywhere heretofore.{\textquotedbl}

What is interesting, is that he found a relationship between the adiabatic index $\omega $ and the number of degrees of
freedom $\alpha $ of (aether) particles, which leads to the conclusion that $\alpha $ must be equal to 6 and he
concludes:

{\textquotedbl}Thus, the quest for a gas-like ethereal medium, satisfying Planck's form for the energy distribution, is
directed to an ideal gas formed by an infinite variety of particles, all having six degrees of freedom.{\textquotedbl}

It is this adiabatic index which provides a relationship to heat capacity, since it is also known as the heat capacity
ratio:

https://en.wikipedia.org/wiki/Heat\_capacity\_ratio 

{\textquotedbl}In thermal physics and thermodynamics, the heat capacity ratio, also known as the adiabatic index, the
ratio of specific heats, or Laplace's coefficient, is the ratio of the heat capacity at constant pressure (C\_P) to
heat capacity at constant volume (C\_V).{\textquotedbl}

So, while we clearly have not yet cracked the whole nut, it seems to me we are on the right track towards the
formulation of a {\textquotedbl}theory of everything{\textquotedbl}, that holy grail that has thus far proven to be
unreachable, which I'm sure will turn out to be attributable to ignoring the implications of the vector Laplace
operator.

Personally, I have no doubt both the weak and strong nuclear forces can be fully accounted for by our model c.q.
electromagnetic forces, once completely worked out, and that the gravitational force also propagates through the
aether, as actually confirmed by the Michelson-Morley experiment, so that we will end up with a model that is much,
much simpler and only has one fundamental interaction of nature. 

To illustrate the argument that the nuclear forces can be fully accounted for by electromagnetic forces, I
wholeheartedly recommend the experimental work of David LaPoint, who shows this in his laboratory:

https://youtu.be/siMFfNhn6dk  

[end intermezzo]

{}-:-

So far, we have shown that it is possible to derive a complete and mathematically consistent set of fields from a single
equation, the 3D generalization of Newton's second law, by using the LaPlace operator and working out the terms
thereof. With this equation, we can use the vector wave equation, which has harmonic solutions, just like the wave
function currently used in Quantum Mechanics. This makes it possible to extend the current Quantum Mechanical wave
function solutions into full 3D solutions in a manner that maintains the fundamental symmetry of the Helmholtz
decomposition within a framework of uniquely defined fields without gauge freedom. We have also shown that we can
decouple the dynamics of the medium from it's substance, mass density, with the velocity diffusion equation which
reveals that the dynamics of the medium are governed by a single constant k, the quantum circulation constant. And we
have shown that we can take higher order derivatives of these equations over and over again, resulting in only phase
differentials for the resulting vector spherical harmonic solutions.

What this comes down to is that we have come to a deeper model of physical reality, which reveals a number of intricate
relationships between various fields defined so far, whereby the quantum circulation constant k determines that at the
quantum level there is an intricate balance between translational and angular momentum. This ultimately governs the
possible harmonic solutions that can exist in the shape of particles, the oscillating dynamic structures that can be
described by the vector spherical harmonics.  

This model offers a new tangible basis for theoretical physics that may eventually very well lead to an an integrated
``theory of everything'', which is however by no means an easy task. 

So far, it has proven to be very challenging even to integrate Maxwell's equations with this basis in a manner that is
completely consistent with the current model and it's units of measurement. Maxwell's equations essentially describe a
phenomenological model that is based upon the assumption that some kind of fundamental quantity called ``charge''
exists, to which a unit of measurement in Coulombs [C] has been assigned. All of the units of measurement within the
electromagnetic domain can be derived from the Coulomb within this model, but there is no definition of what charge
actually is nor what current actually is. Also, there is no explanation for why charge is considered to be polarized. 

However, the model presented thus far has as big advantage that it describes a fluid-like medium and thus we can use
fluid dynamics phenomena as analogies in our analysis. 

Let us start with Ampere's original law to define current density J:

\begin{equation}\label{seq:refText37}
\boldsubformula J=-\nabla \times \boldsubformula{\mathit{R.}}
\end{equation}
And let us provide an overview of the fields defined thus far, along with their units of measurement:

\begin{flushleft}
\tablefirsthead{}
\tablehead{}
\tabletail{}
\tablelasttail{}
\begin{supertabular}{|m{1.737cm}|m{1.763cm}|m{3.698cm}|m{3.7cm}|m{4.0090003cm}|}
\hline
 &
$\Lambda $ =  k v &
Q =  $\tau $ k $\rho $ v &
C = $\eta $ v =  k $\rho $ v &
I = $\eta $ a = k $\rho $ a = k(L+R)\\\hline
$\Lambda $, 

Q, 

C, 

I &
[m3/s2]

 &
[kg/s], [N-s/m], [J-s/m2], [Pa-s-m], [C]

(charge) &
[kg/s2], [N/m]

[J/m2], [Pa-m], [A]

(current, action flux) &
[kg/s3], [N/m-s], [J/m2{}-s], [Pa-m/s], [W/m2] 

(radiosity Je , intensity I, 

energy flux)\\\hline
S,  $\Sigma $

$\Pi $, $\Omega $

T, $\chi $ &
[m2/s2] &
[kg/m-s], [N-s/m2], [J-s/m3], [Pa-s], [V-s]  

(action density,

momentum density flux) &
[kg/m-s2], [N/m2], [J/m3], [Pa], [A/m], [V]

(energy density,

momentum flux, 

force density flux, 

pressure) &
[kg/m-s3], [N/m2{}-s], [J/m3{}-s], [Pa/s], [W/m3], [K]

(power or heat density, 

force flux,

yank density flux, 

temperature)\\\hline
M, $\Lambda $

L, R

Y, $\Psi $ &
[m/s2]  

(a = dv/dt,

acceleration) &
[kg/m2{}-s], [N-s/m3], [Pa-s/m]  

($\rho $ v, momentum den sity, mass flux) &
[kg/m2{}-s2], [N/m3], [Pa/m], [A/m2], [C/m2{}-s]

($\rho $ a, force density, charge flux) &
[kg/m2{}-s3], [N/m3{}-s], [J/m4{}-s], [Pa/m-s], [J/m4{}-s], [W/m4]

($\rho $ j, yank density, current flux)\\\hline
J = curl R

(electric current density) &
 &
 &
[kg/m3{}-s2], [N/m4], [Pa/m2], [A/m3] 

(d2$\rho $/dt2) &
\\\hline
\end{supertabular}
\end{flushleft}
Table 1, overview of fields defined thus far.

This way, we would think of the electric field as being described by L, the translational force density field, and the
magnetic field as being described by R, the angular force density field. And thus current would represent vorticity,
which aligns pretty well with observations such as Elmore's non-radiating guided surface wave\footnote{ Elmore, G.,
Introduction to the Propagating Wave on a Single Conductor, http://www.corridor.biz/FullArticle.pdf\par }. From
equation (37), this gives us a unit of measurement in kilograms per second square [kg/s2] for the Ampere and we can
define the Ampere as well as the Coulomb by:

\begin{center}
\tablefirsthead{\centering 1 Ampere = 1 kilogram per second squared. &
\raggedleft\arraybslash (\stepcounter{Text}{\theText})\\}
\tablehead{\centering 1 Ampere = 1 kilogram per second squared. &
\raggedleft\arraybslash (\stepcounter{Text}{\theText})\\}
\tabletail{}
\tablelasttail{}
\begin{supertabular}{m{14.024cm}m{1.578cm}}

\end{supertabular}
\end{center}
\begin{center}
\tablefirsthead{\centering 1 Coulomb = 1 kilogram per second. &
\raggedleft\arraybslash (\stepcounter{Text}{\theText})\\}
\tablehead{\centering 1 Coulomb = 1 kilogram per second. &
\raggedleft\arraybslash (\stepcounter{Text}{\theText})\\}
\tabletail{}
\tablelasttail{}
\begin{supertabular}{m{14.024cm}m{1.578cm}}

\end{supertabular}
\end{center}
We can subsequently define charge density as the divergence of momentum density:

\begin{equation}
\rho _q=\nabla \cdot (\rho \boldsubformula v)=\frac 1 k\nabla \cdot (\boldsubformula C)=\frac 1 k\Pi ,
\end{equation}
resulting in a unit of measurement for charge density $\rho _q$ in kilograms per cubic meter per second [kg/m3{}-s],
which leads to the conclusion that charge density represents the time derivative of mass density.

With this definition, the charge to mass ratio of a particle results in a unit of measurement in per second or Hertz
[Hz], yielding a characteristic longitudinal oscillation frequency for such a particle. For the electron, this
frequency computes to approximately 175.88 GHz, which falls within 10\% of the calculated spectral radiance dE$\nu
$/d$\nu $ in the observed cosmic background radiation which peaks at 160.23 GHz and is calculated from a measured CMB
temperature of approximately 2.725 K\footnote{ Stowe, P. and Mingst, B., The Atomic Vortex Hypothesis, a Forgotten Path
to Unification, http://vixra.org/abs/1310.0237 \par } suggesting a possible connection. 

This suggestion leads to the idea that even though we can describe the medium itself as a superfluid, we cannot consider
even the vacuum in outer space as devoid from any particles, disturbances or (zero point) energy and thus we can
consider it to have a certain charge density $\rho $qb0, a background charge density, which would be depending on the
material or medium we are working with, just like the permeability and permittivity are. 

This way, we can define the electric field E as follows:

\begin{equation}
\boldsubformula E=\frac 1{\rho _q}\boldsubformula L=k\frac{\boldsubformula L}{\boldsubformula{\Pi }},
\end{equation}
with L as defined in equation (15) and $\rho $q the charge density, resulting in a unit of measurement for the electric
field E in meters per second [m/s]. Coulomb's law then becomes:

\begin{equation}
\boldsubformula F=q\boldsubformula E=\frac q{\rho _q}\boldsubformula L=qk\frac{\boldsubformula L}{\boldsubformula{\Pi
}}.
\end{equation}
The electric (scalar) potential $\varphi $ can subsequently be defined as:

\begin{equation}
\varphi =\frac 1{\rho _q}\Pi =k=\frac{\eta }{\rho },
\end{equation}
with $\Pi $ the scalar pressure in Pascal [Pa] as defined in equation (15), yielding a unit of measurement in meters
squared per second [m2/s] for the scalar electric potential $\varphi $ and thus we can define the Volt as:

\begin{center}
\tablefirsthead{\centering 1 Volt = 1 square meter per second. &
\raggedleft\arraybslash (\stepcounter{Text}{\theText})\\}
\tablehead{\centering 1 Volt = 1 square meter per second. &
\raggedleft\arraybslash (\stepcounter{Text}{\theText})\\}
\tabletail{}
\tablelasttail{}
\begin{supertabular}{m{14.024cm}m{1.578cm}}

\end{supertabular}
\end{center}
We can now also work out the unit of measurement for permittivity $\varepsilon $, which has an SI unit in [C2/N-m2]. By
substitution we find that this results in a unit of measurement in kilograms per cubic meter [kg/m3] and we can equate
the mass density of the medium $\rho $ to its permittivity:

\begin{equation}
\rho =\epsilon 
\end{equation}
For the magnetic field, we start out at the unit of measurement for permeability $\mu $, which is defined in SI units as
Newtons per Ampere squared [N/A2]. By substitution we find that this corresponds [m-s2/kg], the inverse of the
modulus/elasticity in [Pa] or [kg/m-s2]. The latter differs by a per second to the unit of measurement for viscosity
$\eta $ in [Pa-s] or [kg/m-s], the same difference we encountered earlier and which led us to conclude that in our
current models the dimensionality of certain quantities is off by a per second. Therefore, we define the value of
viscosity $\eta $ but not its unit of measurement by:

\begin{equation}
\eta =\frac 1{\mu }.
\end{equation}
We can now define the magnetic field strength:

\begin{equation}
\boldsubformula H=\boldsubformula R,
\end{equation}
with R the angular force density in Newton per cubic meter [N/m3], resulting in a unit of measurement for the magnetic
field strength in Ampere per meter squared [A/m2], which differs from the SI definition which is in Ampere per meter
[A/m]. 

The magnetic flux density then becomes:

\begin{equation}
\boldsubformula B=\mu \boldsubformula H=\mu \boldsubformula R,
\end{equation}
and has a unit of measurement in per meter [/m].

The magnetic (vector) potential A can subsequently be defined as:

\begin{equation}
\boldsubformula A=\Omega ,
\end{equation}
with $\Omega $ the angular vector pressure in Pascal [Pa].

This leaves us with a problem in the dimensionality of the Lorentz force, however, which is not easily resolved in a
satisfactory manner, although dimensionally, we can resolve the problem by defining the Lorentz force as:

\begin{equation}
\boldsubformula F_L=q\lambda \boldsubformula{(v\times B)}=mc\boldsubformula{(v\times \boldsubformula B)},
\end{equation}
whereby $\lambda $ is the wavelength of the particle along $\lambda $=c/f. With f=q/m we then obtain q$\lambda $=mc.

This brings us in the situation whereby we have obtained a fluid dynamics medium model that is capable of bridging the
gap between the Quantum Mechanic and macroscopic worlds in a deterministic manner, but leaves us with open questions
around the detailed nature of the Coulomb and Lorentz forces, especially in relation to the nature of charged particles
and their mass/charge ratios. 

However, it is clear that the irrotational vortex plays a dominant role in magnetics and these can also form closed loop
rings, which explains why magnetic field lines are always closed. This suggests that toroidal ring models like
Parson's\footnote{Alfred L. Parson, {\textquotedbl}A Magneton Theory of the Structure of the Atom{\textquotedbl},
Smithsonian Miscellaneous Collection, Pub 2371, 80pp (Nov 1915).  \par
https://ia802702.us.archive.org/35/items/amagnetontheory00parsgoog/amagnetontheory00parsgoog.pdf\par }  can be
integrated with our model, especially because solid spherical harmonics can be expressed as series of toroidal
harmonics and vice versa\footnote{ Matt Majic, Eric C. Le Ru, ``Relationships between solid spherical and toroidal
harmonics'', arXiv:1802.03484. https://arxiv.org/abs/1802.03484\par } and it is known that the solutions to the vector
wave equation are the spherical harmonics.

When we assume that particles can indeed be considered as consisting of a number of closed loop hollow core vortex
rings, then the physics of the vortex ring can also be expected to provide further insight in the nature of the Lorentz
force working on charged particles. It is for example known that a vortex ring moves forward with its own self-induced
velocity v\footnote{ SULLIVAN, I., NIEMELA, J., HERSHBERGER, R., BOLSTER, D., \& DONNELLY, R. (2008). Dynamics of thin
vortex rings. Journal of Fluid Mechanics, 609, 319-347. doi:10.1017/S0022112008002292 
https://www.researchgate.net/publication/232025984 \par }. And since a vortex ring has two axis of rotation, poloidal
and toroidal, this could also offer an explanation for the existence of the polarization currently attributed to
charge.  

Either way, since all our fields are uniquely defined as solutions of the vector Laplace equation, we can establish that
with deriving all fields from equation (18), we have eliminated ``gauge freedom'' and since we know these equations can
be transformed using the Galilean coordinate transform, we have also eliminated the need for the Lorentz transform and
are thus no longer bound to the universal speed limit.

With this application of the fundamental theorem of vector calculus, we have thus come to a revised version of the
Maxwell equations that not only promises to resolve all of the problems that have been found over the years, we also
obtain a model that is easy to interpret and can be easily simulated and visualized with finite-difference time-domain
methods (FTDT) as well. 

Now let us consider the difference between the definition we found for E and the corresponding definition in Maxwell's
equations:

\begin{equation}
\boldsubformula E_m=-\nabla \varphi _m-\frac{\partial \boldsubformula A_m}{\partial t},
\end{equation}
When considered from the presented perspective, this is what breaks the fundamental result of Helmholtz' decomposition,
namely the decomposition into a rotation free translational component and a divergence free rotational component, since
Am is not rotation free and therefore neither is its time derivative.  

When taking the curl on both sides of this equation, we obtain the Maxwell-Faraday equation, representing Faraday's law
of induction:

\begin{equation}
\nabla \times \boldsubformula E_m=-\frac{\partial \boldsubformula B_m}{\partial t},
\end{equation}
Faraday's law of induction is a basic law of electromagnetism predicting how a magnetic field will interact with an
electric circuit to produce an electromotive force (EMF), which is thus a law that applies at the macroscopic level. It
is clear that this law should not be entangled with a model for the medium and therefore our revision should be
preferred.  

Discussion and Conclusions

We have shown that the terms in the Laplace operator can be written out to define a complete and mathematically
consistent whole of four closely related vector fields which by definition form solutions to the vector Laplace
equation, a result that has tremendous consequences for both the analytical analysis of the electromagnetic field as
well as fluid dynamics vector theory, such as weather forecasting, oceanography and mechanical engineering. The
symmetry between the fields thus defined is fundamental and has been mathematically proven to be correct, so it is
vital to maintain this fundamental symmetry in our physics equations.  

We have also shown that we can decouple the dynamics of the medium from it's substance, mass density, with the velocity
diffusion equation which reveals that the dynamics of the medium are governed by a single constant k, the quantum
circulation constant. And we have shown that we can take higher order derivatives of these equations over and over
again, resulting in only phase differentials for the resulting vector spherical harmonic solutions.

Revising Maxwell equations by deriving directly from a superfluid medium model using the Laplace operator, we have
called upon vector theory for an ideal, compressible, viscous Newtonian superfluid that has led to equations which are
known to be mathematically consistent, are known to be free of singularities and are invariant to the Galilean
transform as well. This results in an integrated model which has only three fundamental units of measurement: mass,
length and time and also explains what ``charge'' is: a compression/decompression oscillation of ``charged'' particles.


As is known from fluid dynamics, these revised Maxwell equations predict three types of wave phenomena, which we can
easily relate to the observed phenomena:

\begin{enumerate}
\item Longitudinal pressure waves, Tesla's superluminal waves\footnote{ Tesla, N. (1900). Art of transmitting electrical
energy through the natural mediums. US patent No. 787,412. In this patent, Tesla disclosed a wireless system by which
he was able to transmit electric energy along the earth's surface and he established that the current of his
transmitter passed over the earth's surface with a mean velocity 471,240 kilometers per second, about $\pi $/2 times
the speed of light.\par } c.q. the super luminal longitudinal dielectric mode, which he found to propagate at a speed
of 471,240 kilometers per second, within 0.1\% of $\pi $/2 times the speed of light. The factor $\pi $/2 coincides with
the situation whereby the theoretical reactance of a shorted lossless transmission line goes to infinity\footnote{
Knight, David. (2016). The self-resonance and self-capacitance of solenoid coils: applicable theory, models and
calculation methods.. 10.13140/RG.2.1.1472.0887.  https://www.researchgate.net/publication/301824613 \par } (eq 1.2)
and thus does not support an electromagnetic wave propagation mode; 
\end{enumerate}
\begin{enumerate}
\item {}``Transverse'' ``water'' surface waves, occurring at the boundary of two media with different densities such as
the metal surfaces of an antenna and air, aka the ``near field'', Elmore's non-radiating surface waves that have been
shown to be guidable along a completely unshielded conductor\footnote{ Elmore, G., Introduction to the Propagating Wave
on a Single Conductor, http://www.corridor.biz/FullArticle.pdf\par };
\end{enumerate}
\begin{enumerate}
\item Vortices and/or vortex rings, the ``far field'', which is known to be quantized and to incorporate a thus far
mysterious mixture of ``particle'' and ``wave'' properties aka ``photons'', the so called ``wave particle duality''
principle. 
\end{enumerate}
Even though the actual wave equations for these three wave types still need to be derived, we can already conclude these
to exist and predict a number of their characteristics, because of the integration of the electromagnetic domain with
the fluid dynamics domain. The latter has a tremendous advantage, namely that dynamic phenomena known to occur in
fluids and gasses can be considered to also occur in the medium.

Further Research

Theoretical

While the revised Maxwell equations presented in this paper describe the motions of the medium accurately in principle,
the actual wave equations for the three predicted wave types still need to be derived and worked out. This is
particularly complicated for the ``transverse'' ``water'' surface wave, because of the fact that in current fluid
dynamics theory the potential fields have not been defined along the Helmholtz decomposition defined by the vector
Laplacian as we proposed, which leads to non-uniquely defined fields and associated problems with boundary conditions.
In order to derive a wave equation for the ``transverse'' surface wave, the incompressibility constraint would have to
be removed from the Saint-Venant equations\footnote{ Farge, M., \& Sadourny, R. (1989). Wave-vortex dynamics in
rotating shallow water.~Journal of Fluid Mechanics,~206, 433-462. doi:10.1017/S0022112089002351 \par } and these would
subsequently need to be fully worked out using vector calculus methods. 

Furthermore, we have also argued that Faraday's law should not be entangled with the model for the medium, which leaves
us without revised equations for Faraday's law of induction. This leads to the question of why a DC current trough a
wire loop results in a magnetic field, but the magnetic field of a permanent magnet does not induce a current in a wire
wound around it. A similar question arises when a (neodymium) magnet is used as an electrode in an electrolysis
experiment, which results in a vortex becoming visible in the electrolyte above the magnet.

It is expected the answers to these questions as well as Faraday's law of induction can be worked out by considering the
physics of the irrotational vortex, given that we found that the current density is actually one and the same thing as
the vorticity of the medium, apart from a constant. In the absence of external forces, a vortex evolves fairly quickly
toward the irrotational flow pattern, where the flow velocity v is inversely proportional to the distance r. The fluid
motion in a vortex creates a dynamic pressure that is lowest in the core region, closest to the axis, and increases as
one moves away from it. It is the gradient of this pressure that forces the fluid to follow a curved path around the
axis and it is this pressure gradient that is directly related to the velocity potential $\Phi $fd c.q. the velocity
field component Efd.

Practical

The revised Maxwell equations presented in this paper open the possibilities of further considerations and research into
the properties of the dielectric and gravitational fields and associated wave phenomena. Because both of these fields
are considered as one and the same within the above presented revised Maxwell paradigm, a wide range of possible
applications become conceivable, some of which are hardly imaginable from within the current paradigm and/or are highly
speculative while others are more straightforward.

Superluminal communication

This is the most direct application of the theory presented in this paper, which is supported by a number of sources
mentioned in the abstract, the oldest of which dates back to 1834, some theoretical methods\footnote{ Shore, G.M..
(1998). `Faster than light' photons in gravitational fields --- Causality, anomalies and horizons. Nuclear Physics B.
460. 379-394. 10.1016/0550-3213(95)00646-X. https://www.researchgate.net/publication/223392267 \par },\footnote{ Nanni,
Luca. (2019). ``Evanescent waves and superluminal behavior of matter''. International Journal of Modern Physics A. 34.
1950141. 10.1142/S0217751X19501410. https://www.researchgate.net/publication/335719399 \par },\footnote{
Pav\v{s}i\v{c}, Matej. (2011). Extra Time Like Dimensions, Superluminal Motion, and Dark Matter.
https://www.researchgate.net/publication/51946961 \par },\footnote{ Aginian, M.A. \& Arutunian, S. \& Lazareva, E.G. \&
Margaryan, A.V. (2018). Superluminal Synchotron Radiation. Resource-Efficient Technologies. 19-25.
10.18799/24056537/2018/4/216. https://www.researchgate.net/publication/329985374  \par } as well as some preliminary
experimental work by the author\footnote{ Lammertink, A.H..  Did I actually measure a superluminous signal thus
disproving the relativity theory? \par
https://www.researchgate.net/post/Did-I-actually-measure-a-superluminous-signal-thus-disproving-the-relativity-theory
\par  }. There is active and ongoing experimental research in this area.

Experiments regarding gravitational effects, such as aimed at obtaining thrust.

The Biefeld-Brown effect is an electrical phenomenon that has been the subject of extensive research involving charging
an asymmetric capacitor to high voltages and the effect is commonly attributed to corona discharges which occur only at
the sharp electrode, which causes an imbalance in the number of positive and negative ions created in comparison to
when a symmetric capacitor is used. 

However, according to a report\footnote{ Bahder, Thomas; Fazi, Christian (June 2003). Force on an Asymmetric Capacitor.
U.S. Army Research Laboratory -- via Defense Technical Information Center. http://www.dtic.mil/docs/citations/ADA416740
 \par } by researchers from the Army Research Laboratory (ARL), the effects of ion wind was at least three orders of
magnitude too small to account for the observed force on the asymmetric capacitor in the air. Instead, they proposed
that the Biefeld--Brown effect may be better explained using ion drift instead of ion wind. This was later confirmed by
researchers from the Technical University of Liberec\footnote{ Mal\'ik, M. \& Primas, J. \& Kopeck\'y, V\'aclav \&
Svoboda, M.. (2013). Calculation and measurement of a neutral air flow velocity impacting a high voltage capacitor with
asymmetrical electrodes. AIP Advances. 4. 10.1063/1.4864181. \par }. 

If this is correct, then the need for an asymmetric capacitor raises the question if the resulting diverging electric
field can indeed be used to obtain thrust by working on an electrically neutral dielectric, in this case a dielectric
consisting of air and net neutral ions, and how this results in a net force acting upon the capacitor plates. It is
known that a dielectric is always drawn from a region of weak field toward a region of stronger field. It can be shown
that for small objects the force is proportional to the gradient of the square of the electric field, because the
induced polarization charges are proportional to the fields and for given charges the forces are proportional to the
field as well. There will be a net force only if the square of the field is changing from point to point, so the force
is proportional to the gradient of the square of the field\footnote{ Feynman, R. (1963, 2006, 2010). The Feynman
Lectures on Physics Vol. II Ch. 10: Dielectrics. (Fig. 10--8.)
https://www.feynmanlectures.caltech.edu/II\_10.html\#Ch10-F8  \par }.

Another line of research in this regard has to do with the gravitational force itself, which can be speculated to be
caused by longitudinal dielectric flux, which causes a pushing and not a pulling force. This is supported by Van
Flandern\footnote{ Van Flandern, T (1998). The speed of gravity ? What the experiments say. Physics Letters A. 250
(1--3): 1--11. http://www.ldolphin.org/vanFlandern/gravityspeed.html \par }, who determined that with a purely central
pulling force and a finite speed of gravity, the forces in a two-body system no longer point toward the center of mass,
which would make orbits unstable. The fact alone that a central pulling gravity force requires a practically infinite
speed makes clear that pulling gravity models are untenable and recourse must be taken to a Lesagian type of pushing
gravity model. The longitudinal dielectric flux which would thus describe gravity is probably caused by cosmic
(microwave) background radiation. If this naturally occurring flux had an arbitrary frequency spectrum, superconductors
would reflect this flux and would thus shield gravity, which does not happen.

However, acceleration fields outside a rotating superconductor were found\footnote{ Tajmar, Martin \& Plesescu, Florin
\& Marhold, Klaus \& de Matos, C.. (2006). Experimental Detection of the Gravitomagnetic London Moment. \par
},\footnote{ Tajmar, Martin \& Plesescu, Florin \& Seifert, Bernhard \& Marhold, Klaus. (2007). Measurement of
Gravitomagnetic and Acceleration Fields Around Rotating Superconductors. ChemInform. 38. 10.1002/chin.200734232. \par
}, which are referred to as Gravitomagnetic effects, and also anomalous acceleration signals, anomalous gyroscope
signals and Cooper pair mass excess were found in experiments with rotating superconductors\footnote{ De Matos, C. \&
Christian, Beck. (2009). Possible Measurable Effects of Dark Energy in Rotating Superconductors. Advances in Astronomy.
2009. 10.1155/2009/931920. }. 

It can be speculated that the relation Stowe and Mingst found between the characteristic oscillation frequency of the
electron and the cosmic microwave background radiation is what causes the spectrum of the gravitational flux and that
this is related to the characteristic oscillation frequencies of the electron, neutron and proton as well. If that is
the case, then the incoming flux would resonate with the oscillating particles within the material at these specific
frequencies, which would therefore not be blocked/reflected but would be absorbed/re-emitted along Huygens' principle.

It can further be speculated that when objects are rotated, their ``clock'', the characteristic oscillation frequency of
the elemental particles making up the material, would be influenced, causing them to deviate from the specific
frequencies they otherwise operate at. It is conceivable that this would result in a condition whereby superconductors
would indeed reflect the naturally occurring gravitational flux, which could explain this anomaly.


\end{document}
