% This file was converted to LaTeX by Writer2LaTeX ver. 1.4
% see http://writer2latex.sourceforge.net for more info
\documentclass[a4paper]{article}
\usepackage[ascii]{inputenc}
\usepackage[T1]{fontenc}
\usepackage[english]{babel}
\usepackage{amsmath}
\usepackage{amssymb,amsfonts,textcomp}
\usepackage{color}
\usepackage[top=1.249cm,bottom=1.249cm,left=2.499cm,right=2.499cm,nohead,includefoot,foot=1.251cm,footskip=2.401cm]{geometry}
\usepackage{array}
\usepackage{supertabular}
\usepackage{hhline}
\usepackage{hyperref}
\hypersetup{colorlinks=true, linkcolor=blue, citecolor=blue, filecolor=blue, urlcolor=blue}

% Text styles
\newcommand\textstyleNone[1]{#1}
\newcommand\textstyleInternetlink[1]{#1}
\newcommand\textstyletexhtml[1]{#1}
\newcommand\textstyleappleconvertedspace[1]{#1}
\newcommand\textstyleHyperlinkii[1]{#1}
\makeatletter
\newcommand\arraybslash{\let\\\@arraycr}
\makeatother
% Footnote rule
\setlength{\skip\footins}{0.119cm}
\renewcommand\footnoterule{\vspace*{-0.018cm}\setlength\leftskip{0pt}\setlength\rightskip{0pt plus 1fil}\noindent\textcolor{black}{\rule{0.25\columnwidth}{0.018cm}}\vspace*{0.101cm}}
\setlength\tabcolsep{1mm}
\renewcommand\arraystretch{1.3}
\newcounter{Text}
\renewcommand\theText{\arabic{Text}}
\newcommand\boldsubformula[1]{\text{\mathversion{bold}$#1$}}
\title{}
\author{}
\date{2022-11-20}
\begin{document}
\clearpage\setcounter{page}{1}{\centering\bfseries\color[rgb]{0.101960786,0.101960786,0.101960786}
Revision and integration of Maxwell's and Navier-Stokes' Equations 
\par}

{\centering\bfseries\color[rgb]{0.101960786,0.101960786,0.101960786}
and the origin of quantization in Superfluids and Spacetime itself
\par}

{\centering\color[rgb]{0.101960786,0.101960786,0.101960786}
Arend Lammertink
\par}

{\centering\itshape\color[rgb]{0.101960786,0.101960786,0.101960786}
Schoolstraat 107, 7471 WV, Goor, The Netherlands,
\par}

{\centering\itshape\color[rgb]{0.101960786,0.101960786,0.101960786}
+316 5425 6426, \href{mailto:lamare@gmail.com}{lamare@gmail.com}
\par}

{\centering\color[rgb]{0.101960786,0.101960786,0.101960786}
{}``I hope that someone will discover a more realistic way, 
\par}

{\centering\color[rgb]{0.101960786,0.101960786,0.101960786}
or rather a more tangible basis than it has been my lot to find.''
\par}

{\centering\bfseries\color[rgb]{0.101960786,0.101960786,0.101960786}
\textmd{Albert Einstein}.
\par}

{
\textbf{{DRAFT revision 33. }}}

{

\textbf{{Abstract}}{.
It is well known that the Maxwell equations predict the behavior of the
electromagnetic field very well. However, they predict only one wave equation
while there are significant differences between the
{\textquotedbl}near{\textquotedbl} and {\textquotedbl}far{\textquotedbl} fields
and various anomalies have been observed involving the detection of super
luminous signals in experiments with electrically short coaxial
cables}\footnote{
\href{https://www.researchgate.net/profile/Steffen-Kuehn-3?_sg\%5B0\%5D=zlLZ0Gbl1GNlaEJvpgGlVutlqzhUsFGEl-EZkXLnMKvKykRtWeKzdlCZ2I4o0ixYK73oKKY.8mfcbtOw1v3oZSHsL8aIQtbCwx_4U6Di5vqQ84O5mnn8TnI7pcEwkxjCffJUXI-C5Kre7wRGShKiT7EGeFY9qg&_sg\%5B1\%5D=WUlVJagcaa-si2q-5nMYe4qf09LJiCKvtcIrx5Iyn0639PGnwxYpSZoBHKhgJEBMRVUxBy0.WqE4nE0I582CEhQgcvCR_67us9bpt94MFGuqsz2FgFJjxM1_QQ8YD0ErlNpA3_vvRvmJlhhX60C8i0OUaTLTEw}{K\"uhn},
Steffen. (2019) Electronic data transmission at three times the speed of light
and data rates of 2000 bits per second over long distances in buffer amplifier
chains. DOI: 10.13140/RG.2.2.33988.78721/1
\url{https://www.researchgate.net/publication/335677198} \par
}{\textsuperscript{,}}\footnote{
K\"uhn, Steffen. (2020). General Analytic Solution of the Telegrapher's
Equations and the Resulting Consequences for Electrically Short Transmission
Lines. Journal of Electromagnetic Analysis and Applications. 12. 71-87.
10.4236/jemaa.2020.126007. \par
}{, microwaves}\footnote{
Ranfagni, A \& Mugnai, Daniela \& Ruggeri, Rocco. (2004). Unexpected behavior of
crossing microwave beams. Physical review. E, Statistical, nonlinear, and soft
matter physics. 69. 027601. 10.1103/PhysRevE.69.027601.\par
}\textstyleNone{{\textsuperscript{,}}}\footnote{
Allaria, Enrico \& Mugnai, Daniela \& A.ranfagni, \& C.ranfagni,. (2011).
Unexpected behavior of crossing of microwave and optical beams. Modern Physics
Letters B. 19. 10.1142/S0217984905009274.
\url{https://www.researchgate.net/publication/263801213} \par
}\textstyleNone{{\textsuperscript{,}}}\footnote{
Agresti, Alessandro \& Cacciari, Ilaria \& Ranfagni, A \& Mugnai, Daniela \&
Mignani, Roberto \& Petrucci, Andrea. (2015). Two possible interpretations of
the near-field anomaly in microwave propagation. Results in Physics. 5. 196.
10.1016/j.rinp.2015.08.002.
\url{https://www.researchgate.net/publication/282609282} \par
}\textstyleNone{{\textsuperscript{,}}}\footnote{
Mojahedi, Mohammad \& Schamiloglu, Edl \& Hegeler, Frank \& Malloy, Kevin.
(2000). Time-domain detection of superluminal group velocity for single
microwave pulses. Physical review. E, Statistical physics, plasmas, fluids, and
related interdisciplinary topics. 62. 5758-66. 10.1103/PhysRevE.62.5758.
\url{https://www.researchgate.net/publication/12238975} \par
}\textstyleNone{{\textsuperscript{,}}}\footnote{
Musha. (2019). Superluminal Speed of Photons in the Electromagnetic Near-Field.
Recent Adv Photonics Opt 2(1):36-39.
\url{https://scholars.direct/Articles/photonics-and-optics/rapo-2-007.php} \par
}\textstyleNone{{\textsuperscript{,}}}\footnote{
Wang \& Xiong. (2003). Superluminal Behaviors of Electromagnetic Near-fields.
\url{https://arxiv.org/pdf/physics/0311061.pdf} \par
}\textstyleNone{{\textsuperscript{,}}}\footnote{
Walker. (2006). Superluminal Electromagnetic and Gravitational Fields Generated
in the Nearfield of Dipole Sources. \url{https://arxiv.org/abs/physics/0603240}
\par
}\textstyleNone{{\textsuperscript{,}}}\footnote{
Solemino, De Lisio \& Altucci. (2014) Superluminal behavior in wave propagation:
a famous case study in the microwave region.
\url{http://www.societanazionalescienzeletterearti.it/pdf/161_nota_Solimeno-Altucci_19-12-2014.pdf}
\par
}\textstyleNone{{\textsuperscript{,}}}\footnote{
Ranfagni, A. \& Fabeni, P. \& Pazzi, G. \& Ricci, A. \& Trinci, R. \& Mignani,
Roberto \& Ruggeri, Rocco \& Cardone, F. \& Agresti, Alessandro. (2006).
Observation of Zenneck-type waves in microwave propagation experiments. Journal
of Applied Physics. 100. 024910 - 024910. 10.1063/1.2212307.
\url{https://www.researchgate.net/publication/224453964} \par
}\textstyleNone{{\textsuperscript{,}}}\footnote{
Walker, William. (2000). Experimental Evidence of Near-Field Superluminally
Propagating Electromagnetic Fields. 10.1007/0-306-48052-2\_18.
\url{https://arxiv.org/pdf/physics/0009023.pdf} \par
}{, optical
fibers}\footnote{\textstyleNone{ Gonzalez-Herraez, Miguel \& Song, Kwang-Yong \&
Th\'evenaz, Luc. (2005). Optically Controlled Slow and Fast Light in Optical
Fibers Using Stimulated Brillouin Scattering. Applied Physics Letters. 87.
081113 - 081113. 10.1063/1.2033147. }\par
}\textstyleNone{{\textsuperscript{,}}}\footnote{\textstyleNone{
Stenner, M.D., Gauthier, D.J. \& Neifeld, M.A., The speed of information in a
`fast-light' optical medium, Nature. 425. 695-8. 10.1038/nature02016. }\par
}\textstyleNone{{\textsuperscript{,}}}\footnote{\textstyleNone{
Th\'evenaz, Luc. (2008). ``Achievements in slow and fast light in optical
fibres''. 1. 75 - 80. 10.1109/ICTON.2008.4598375. }\par
}\textstyleNone{{\textsuperscript{,}}}\footnote{\textstyleNone{
Wang, L. \& Kuzmich, A \& Dogariu, Arthur. (2000). Gain-assisted superluminal
light propagation. Nature. 406. 277-9. 10.1038/35018520. }\par
}{ as well as other
methods}\footnote{ Ardavan, A, Singleton, J, Ardavan, H,  Fopma, J, Halliday, D
and Hayes, W. (2004). Experimental demonstration of a new radiation mechanism:
emission by an oscillating, accelerated, superluminal polarization current.
arXiv:physics/0405062 \url{https://arxiv.org/abs/physics/0405062} \par
}\textstyleNone{{\textsuperscript{,}}}\footnote{
Niang, Anna \& de Lustrac, Andr\'e \& Burokur, Shah Nawaz. (2018). Broadband
Superluminal Transmission Line with Non-Foster Negative Capacitor.
10.1049/cp.2018.1179. \url{https://www.researchgate.net/publication/328410118}
\par
}\textstyleNone{{\textsuperscript{,}}}\footnote{\textstyleNone{
Wheatstone, Charles. (1834). An Account of Some Experiments to Measure the
Velocity of Electricity, and the Duration of Electric Light.  Philosophical
Transactions of the Royal Society of London. 124.  10.1098/rstl.1834.0031 }\par
}{\textsuperscript{,}}\footnote{\textstyleNone{
Tesla, N. (1900). Art of transmitting electrical energy through the natural
mediums. US patent No. 787,412. In this patent, Tesla disclosed a wireless
system by which he was able to transmit electric energy along the earth's
surface and he established that the current of his transmitter passed over the
earth's surface with a mean velocity 471,240 kilometers per second, about $\pi
$/2 times the speed of light.}\par
}{.  

We show that the mathematical Laplace operator defines a complete set of vector
fields consisting of two potential fields and two fields of force, which form a
Helmholtz decomposition of any given vector field }\textbf{{F}}{.

We found that neither in Maxwell's equations nor in fluid dynamics vector theory
this result has been recognized, which causes the potential fields to not be
uniquely defined and also makes the Navier-Stokes equations unnecessarily
complicated and introduces undesirable redundancy as well. We show that
equivalents to both the Maxwell equations as well as the Navier-Stokes equations
can be directly derived from a single diffusion equation describing Newton's
second law in 3D. We found that the diffusion constant \textit{{k}} in this
equation has the same value as the speed of light squared, but has a unit of
measurement in meters squared per second thus uncovering problems with time
derivatives in current theories, showing among others that the mass-energy
equivalence principle is untenable. Finally, we show that the diffusion equation
we found can be divided by mass density $\rho $, resulting in a velocity
diffusion equation that only has units of measurement in meters and seconds,
thus decoupling the dynamics of the medium from it's substance, mass density
$\rho $. This reveals the quantized nature of spacetime itself, whereby the
quantum circulation constant }\textit{{k}}{ is found to govern the dynamics of
physical reality, leading to the conclusion that at the fundamental quantum
level only dynamic viscous forces exist while static elastic forces are an
illusion created by problems with a number of time derivatives in current
theories.}}

{
{With our equivalents for the Maxwell equations three types of wave
phenomena can be described, including super luminous longitudinal sound-like waves that can explain the  mentioned
anomalies. This paper contributes to the growing body of work revisiting Maxwell's equations}\footnote{ K. J. van
Vlaenderen and A. Waser, Generalization of Classical Electrodynamics to Admit a Scalar Field and Longitudinal Waves.
Hadronic Journal, Vol. 24, 2001, pp. 609-628. \url{https://www.researchgate.net/publication/267480932} \par
}{\textsuperscript{,}}\footnote{ Barrett, Terence. (1993).
Electromagnetic phenomena not explained by Maxwell's equations. 10.1142/9789814360005\_0002, 
\url{https://www.researchgate.net/publication/288491661} \par
}{\textsuperscript{,}}\footnote{ Pinheiro, Mario. (2005). Do
Maxwell's Equations Need Revision? A Methodological Note. Physics Essays. 20. 10.4006/1.3119404. 
\textstyleInternetlink{https://www.researchgate.net/publication/2174426} \par
}{\textsuperscript{,}}\footnote{ Behera, Harihar \& Barik, N..
(2018). A New Set of Maxwell-Lorentz Equations and Rediscovery of Heaviside-Maxwellian (Vector) Gravity from Quantum
Field Theory. \url{https://www.researchgate.net/publication/328275054}\textstyleInternetlink{ }\par
}{\textsuperscript{,}}\footnote{ Nedi\'{c}, S., Longitudinal Waves
in Electromagnetism. Towards Consistent Theoretical Framework for Tesla's Eergy and Information Transmission.
INFOTEH-JAHORINA Vol. 16, March 2017, \par \url{https://infoteh.etf.ues.rs.ba/zbornik/2017/radovi/KST-2/KST-2-8.pdf}
}{\textsuperscript{,}}\footnote{ Atsukovsky, V.A.,
Efirodinamicheskie Osnovy Elektromagnetizma -- Teoria, Eksperimenty, Vedrenie. Energoatomizdat, 2011, ISBN 978- 5-
283-03317-4; an English translation of the Section 6 `Electromagnetic field' is available on request at
nedic.slbdn@gmail.com, or as Appendix I in his paper, avail. at: \par
\url{https://www.researchgate.net/publication/297715588} \par
}{\textsuperscript{,}}\footnote{ Arbab, Arbab \& Al-Ajmi, Mudhahir.
(2014). The Modified Electromagnetism and the Emergent Longitudinal Wave. Applied Physics Research. 10.
10.5539/apr.v10n2p45. \url{https://www.researchgate.net/publication/260716325} \par
}{\textsuperscript{,}}\footnote{ Onoochin, Vladimir. (2019).
Longitudinal Electric field and the Maxwell Equation. 10.13140/RG.2.2.26478.36163. 
\url{https://www.researchgate.net/publication/335686655}  \par
}{\textsuperscript{,}}\footnote{ Gray, Robert. (2019). Experimental
Disproof of Maxwell and Related Theories of Classical Electrodynamics.
\url{https://www.researchgate.net/publication/335777516}\textstyleInternetlink{ }\par \textstyleInternetlink{
}}{\textsuperscript{,}}\footnote{ 
\href{https://www.ingentaconnect.com/search;jsessionid=9s75nnff84d8s.x-ic-live-02?option2=author&value2=Shaw,+Duncan+W.}{\textstyleInternetlink{Shaw,
Duncan W.}} (2014). Reconsidering Maxwell's aether. Physics Essays, Volume 27, Number 4, December 2014, pp. 601-607(7).
 \url{https://doi.org/10.4006/0836-1398-27.4.601} \par \url{http://www.duncanshaw.ca/ReconsideringMaxwellsAether.pdf}
}{ by deriving all of the fields from a single equation, so the
result is known to be mathematically consistent and free of singularities and uniquely defines the potential fields
thus eliminating gauge freedom. Unlike Maxwell's equations, which are the result of the entanglement of Faraday's
circuit level law with the more fundamental medium arguably creating most of the problems in current theoretical
physics, these revisions describe the three different electromagnetic waves observed in practice and so enable a better
mathematical representation.}}

{
\textbf{{Keywords}}{:
Classical Electrodynamics, Superfluid medium, Fluid Dynamics, Theoretical Physics, Vector Calculus.}}

{\centering\bfseries\color[rgb]{0.101960786,0.101960786,0.101960786}
Introduction
\par}

{
{In 1861, James Clerk Maxwell published his paper ``On Physical
Lines of Force''}\footnote{ Maxwell, J.. (1861). On Physical Lines of Force. The London. Edinburgh. 338-348.
0.1080/14786446108643067. \par }{, wherein he theoretically derived
a set of twenty equations which accurately described the electro-magnetic field insofar as known at that time. He
modeled the magnetic field using a molecular vortex model of Michael Faraday's {\textquotedbl}lines of
force{\textquotedbl} in conjunction with the experimental result
}{of Weber and Kohlrausch, who determined in 1855 that there was a
quantity related to electricity and magnetism, the ratio of the absolute electromagnetic unit of charge to the absolute
electrostatic unit of charge, and determined that it should have units of velocity. In an experiment, which involved
charging and discharging a Leyden jar and measuring the magnetic force from the discharge current, they found a value
3.107e8 m/s, remarkably close to the speed of light.}}

{\color[rgb]{0.101960786,0.101960786,0.101960786}
In 1884, Oliver Heaviside, concurrently with similar work by Josiah Willard Gibbs and Heinrich Hertz, grouped Maxwell's
twenty equations together into a set of only four, via vector notation. This group of four equations was known
variously as the Hertz-Heaviside equations and the Maxwell-Hertz equations but are now universally known as Maxwell's
equations. }

{
{The Maxwell equations predict the existence of just one type of
electromagnetic wave, even though it is now known that at least two electromagnetic wave phenomena exist, namely the
``near'' and the ``far'' fields. The ``near'' field has been shown to be a non-radiating surface wave that is guidable
along a completely unshielded single conductor}\footnote{ Elmore, G., Introduction to the Propagating Wave on a Single
Conductor, \url{http://www.corridor.biz/FullArticle.pdf}\par }{ and
can be applied for wide band low loss communication
systems}\textstyleNone{{. The Maxwell equations have not been
revised to incorporate this new knowledge.  }}}

{
\textstyleNone{{Given the above, the following questions should be
asked: }}}

\begin{itemize}
\item {
\textstyleNone{{What is charge?}}}
\item {
\textstyleNone{{Why is it a property of certain particles?}}}
\end{itemize}
{
\textstyleNone{{As long as we insist that charge is an elemental
quantity that is a property of certain particles, we cannot answer these questions. Also, when the wave particle
duality principle is considered in relation to what is considered to be the cause for electromagnetic radiation,
charged particles, in Maxwell's equations electromagnetic radiation is essentially considered to be caused by (quanta
of) electromagnetic radiation, an obvious case of circular logic which is not desirable. }}}

{
\textstyleNone{{In the area of vector calculus, Helmholtz's theorem,
also known as the fundamental theorem of vector calculus, states that any sufficiently smooth, rapidly decaying vector
field in three dimensions can be resolved into the sum of an irrotational (curl-free) vector field and a solenoidal
(divergence-free) vector field; this is known as the Helmholtz decomposition. A terminology often used in physics
refers to the curl-free component of a vector field as the longitudinal component and the divergence-free component as
the transverse component. This theorem is also of great importance in electromagnetic (EM) and microwave engineering,
especially for solving the low-frequency breakdown issues caused by the decoupling of electric and magnetic
fields.}}\footnote{ Xiong, Xiaoyan \& Sha, Wei \& Jiang, Li. (2014). Helmholtz decomposition based on integral equation
method for electromagnetic analysis. Microwave and Optical Technology Letters. 56. 10.1002/mop.28454.\par
}\textstyleNone{{\textsuperscript{
}}{Further, a vector field can be uniquely specified by a
prescribed divergence and curl and it can be shown that the Helmholtz theorem holds for arbitrary vector fields, both
static and time-dependent}}\footnote{\textstyleNone{ Oleinik, V.P., The Helmholtz Theorem and Superluminal Signals,
Department of General and Theoretical Physics, National Technical University of Ukraine {\textquotedbl}Kiev Polytechnic
Institute{\textquotedbl}, Prospect Pobedy 37, Kiev, 03056, Ukraine,
}\url{https://arxiv.org/abs/quant-ph/0311124}\textstyleNone{ }\par
}\textstyleNone{{.}}}

{
\textstyleNone{In potential theory, the study of harmonic functions, the Laplace equation is very important, amongst
other with regards to consideration of the symmetries of the Laplace equation. The symmetries of the n-dimensional
Laplace equation are exactly the conformal symmetries of the n-dimensional Euclidean space, which has several
implications. One can systematically obtain the solutions of the Laplace equation which arise from separation of
variables such as spherical harmonic solutions and Fourier series. By taking linear superpositions of these solutions,
one can produce large classes of harmonic functions which can be shown to be dense in the space of all harmonic
functions under suitable topologies. }}

{
\textstyleNone{The Laplace equation as well as the more general Poisson equation are 2nd order differential equations,
in both of which the Laplacian represents the flux density of the gradient flow of a function. In one dimension, the
Laplacian simply is ${\partial}${\texttwosuperior}/${\partial}$x{\texttwosuperior}, representing the curvature of a
given function \textit{f}. For scalar functions in 3D, the Laplacian is a }common generalization of the second
derivative and is the differential operator defined by:}

\begin{equation}
\nabla ^2f=\frac{\partial ^2f}{\partial x^2}\text{ + }\frac{\partial ^2f}{\partial y^2}\text{ + }\frac{\partial
^2f}{\partial z^2}
\end{equation}
{
\textstyleNone{The Laplacian of a scalar function is equal to the divergence of the gradient and the trace of the
Hessian matrix. The vector Laplacian is a further generalization in three dimensions and defines the second order
spatial derivative of any given vector field \textbf{\textit{F}}, the ``3D curvature'' if you will, and is given by the
identity:}}

\begin{equation}
\nabla ^2\boldsubformula F=\nabla (\nabla \cdot \boldsubformula F)-\nabla \times (\nabla \times \boldsubformula F)
\end{equation}
{
\textstyleNone{Whereas the scalar Laplacian applies to a scalar field and returns a scalar quantity, the vector
Laplacian applies to a vector field, returning a vector quantity. When computed in orthonormal Cartesian coordinates,
the returned vector field is equal to the vector field of the scalar Laplacian applied to each vector component. }}

{
\textstyleNone{With this identity, a full 3D generalization of the Poisson equation can also be defined, }the vector
wave equation, which has three independent solutions\footnote{ Julius Adams Stratton, Electromagnetic Theory,
McGraw-Hill Book Company, New York, 1941.\par  }, the vector spherical harmonics:}

\begin{equation}\label{seq:refText3}
{\nabla}^2\boldsubformula F\text{ + }k^2\boldsubformula F=\nabla \nabla \cdot \boldsubformula F\text{ - }\nabla \times
\nabla \times \boldsubformula F\text{ + }k^2\boldsubformula F=0.
\end{equation}
{\centering
\textstyleNone{\textbf{{Methods}}}
\par}

{
\textstyleNone{The terms in the definition for the vector Laplacian can be negated:}}

\begin{equation}
-\nabla ^2\boldsubformula F=-\nabla (\nabla \cdot \boldsubformula F)+\nabla \times (\nabla \times \boldsubformula F)
\end{equation}
{
\textstyleNone{and then the terms in this identity can be written out to define a vector field for each of these terms: 
}}

\begin{equation}\label{seq:refText5}
\begin{matrix}\hfill \boldsubformula A&=&\nabla \times \boldsubformula F\hfill\null \\\hfill \varphi &=&\nabla \cdot
\boldsubformula F\hfill\null \\\hfill \boldsubformula B&=&\nabla \times \boldsubformula A\text{ = }\nabla \times
(\nabla \times \boldsubformula F)\hfill\null \\\hfill \boldsubformula E&=&-\nabla \varphi \text{ = }-\nabla (\nabla
\cdot \boldsubformula F)\hfill\null \end{matrix}
\end{equation}
{
\textstyleNone{{And, since the curl of the gradient of any
twice-differentiable scalar field }{$\varphi $
}{is always the zero vector
}{(}{${\nabla}$}{{\texttimes}(}{${\nabla}$}{$\varphi
$}{)=}{0)}{,
and the divergence of the curl of any vector field
}\textbf{\textit{{A
}}}{is always zero as well
(}{${\nabla}{\cdot}$}{(}{${\nabla}$}{{\texttimes}}\textbf{\textit{{A}}}{)=0),
we can establish that }\textbf{\textit{{E
}}}{is
}{curl-free
}{and
}\textbf{\textit{{B
}}}{is
}{divergence-free}{,
and we can write}: }}

\begin{equation}
\begin{matrix}\hfill \nabla \times \boldsubformula E&=&0\hfill\null \\\hfill \nabla \cdot \boldsubformula
B&=&0\hfill\null \end{matrix}
\end{equation}
{
\textstyleNone{As can be seen from this, the vector Laplacian establishes a Helmholtz decomposition of the vector field
\textbf{\textit{F}} into an {irrotational or }curl free component
\textbf{\textit{E}} and a divergence free component \textbf{\textit{B}}, along with associated potential fields
{$\varphi $} and \textbf{\textit{A}}, all from a single equation
c.q. operator. }}

{
\textstyleNone{Thus we have shown that the mathematical definitions for potential fields are hidden within the Laplace
operator c.q. the fundamental theorem of vector calculus c.q. the second order spatial derivative, which has tremendous
consequences for both the analytical analysis of the electromagnetic field as well as fluid dynamics vector theory. The
symmetry between the fields thus defined is fundamental and has been mathematically proven to be correct, so it is
vital to maintain this fundamental symmetry in our physics equations. }}

{
\textstyleNone{So far, we have considered the general case, which is valid for any given vector field \textbf{F}. In the
following, we will use the \textsubscript{m} subscript to refer to the electromagnetic domain along Maxwell's
equations, while the \textsubscript{f} subscript is used for the fluid dynamics domain. }}

{
\textstyleNone{In Maxwell's equations, the curl of the electric field \textbf{\textit{E}}\textsubscript{m} is defined by
{the Maxwell-Faraday equation}:}}

\begin{equation}
\nabla \times \boldsubformula E_m=-\frac{\partial \boldsubformula B_m}{\partial t},
\end{equation}
{
\textstyleNone{which is obvious not equal to zero for electromagnetic fields varying with time and therefore the dynamic
Maxwell equations cannot be second order spatial derivatives of any vector field \textbf{\textit{F}}\textsubscript{m}
as defined by the Laplacian. }}

{
\textstyleNone{Herewith, we have shown that no vector field \textbf{\textit{F}}\textsubscript{m} exists for which
Maxwell's equations are the second order spatial derivative and therefore Maxwell's equations do not satisfy the vector
Laplace equation. The end result of this is that while the solutions of Laplace's equation are all possible harmonic
wave functions, with Maxwell's equations there is only one resulting wave equation which defines a ``transverse'' wave,
whereby the \textbf{\textit{E}}\textsubscript{m} {and
}\textbf{\textit{{B}}}{\textsubscript{m}}{
components are always perpendicular with respect to one another. This is also the reason why no separate wave equations
can be derived for the ``near'' and ``far'' fields.}}}

{
\textstyleNone{{Furthermore, in Maxwell's equations, the two
potential fields which are used with Helmholtz's theorem are the electrical potential
}{$\varphi
$}{\textsubscript{m}}{
and the magnetic vector potential
}\textbf{\textit{{A}}}{\textsubscript{m}}{,
which are defined by the equations}}\footnote{ Feynman, Richard P; Leighton, Robert B; Sands, Matthew (1964). The
Feynman Lectures on Physics Volume 2. Addison-Wesley. ISBN 0-201-02117-X. (pp. 15-15)\par
}\textstyleNone{{:}}}

\begin{equation}
\begin{matrix}\hfill \boldsubformula B_m&=&\nabla \times \boldsubformula A_m\hfill\null \\\hfill \boldsubformula
E_m&=&-\nabla \varphi _m-\frac{\partial \boldsubformula A_m}{\partial t}\hfill\null \end{matrix}
\end{equation}
{
\textstyleNone{{where
}\textbf{\textit{{B}}}{\textsubscript{m}}{
is the magnetic field and
}\textbf{\textit{{E}}}{\textsubscript{m}}{
is the electric field.}}}

{
\textstyleNone{{The Helmholtz theorem can also be described as
follows. Let
}\textbf{{A}}{
be a solenoidal vector field and }{$\varphi
$}{ }{a scalar
field on
}\textbf{{R}}{\textsuperscript{3}}{
which are sufficiently smooth and which vanish faster than
}{1/r}{\textsuperscript{2}}{
at infinity. Then there exists a vector field
}\textbf{{F}}{
such that:}}}

\begin{equation}
\nabla \boldsubformula F=\varphi \text{and}\nabla \times \boldsubformula F=\boldsubformula A
\end{equation}
{
\textstyleNone{{and if additionally, the vector field
}\textbf{{F}}{
vanishes as r $\rightarrow $ ${\infty}$, then
}\textbf{{F}}{
is unique}}\footnote{ David J. Griffiths, Introduction to Electrodynamics, Prentice-Hall, 1999, p. 556.\par
}\textstyleNone{{.}}}

{
\textstyleNone{{Now let us consider the units of measurement
involved in these fields, whereby the three vector operators used all have a unit of measurement in per meter [/m]. The
magnetic field
}\textbf{{B}}{\textsubscript{m}}{
has a unit of measurement in Tesla [T], which is defined in SI units as
[kg/s}{\textsuperscript{2}}{{}-A].
So, for the magnetic vector potential
}\textbf{{A}}{\textsubscript{m}}{
we obtain a unit of
[kg-m/s}{\textsuperscript{2}}{{}-A]
and for
}{d}\textbf{{A}}{\textsubscript{m}}{/dt}{
we obtain a unit of
[kg-m/s}{\textsuperscript{3}}{{}-A].
The electric field
}\textbf{{E}}{\textsubscript{m}}{
has a unit of measurement in volt per meter, which is defined in SI units as
[kg-m/s}{\textsuperscript{3}}{{}-A],
which matches that for
}{d}\textbf{{A}}{\textsubscript{m}}{/dt}{.
So, for the electric scalar potential }{$\varphi
$}{\textsubscript{m}}{
we obtain a unit of
}{[kg-m}{\textsuperscript{2}}{/s}{\textsuperscript{3}}{{}-A].}}}

{
\textstyleNone{{However, neither the units of measurement for
}\textbf{{E}}{\textsubscript{m}}{
and
}\textbf{{B}}{\textsubscript{m}}{
are the same, nor are the units of measurements for }{$\varphi
$}{\textsubscript{m}}{
and
}\textbf{{A}}{\textsubscript{m}}{.
This is in contradiction with Helmholtz's theorem, which states that a vector field
}\textbf{{F}}{\textsubscript{m}}{
exists that should have a unit of measurement equal to that of
}{$\varphi
$}{\textsubscript{m}}{
and
}\textbf{{A}}{\textsubscript{m}}{
times meters or that of
}\textbf{{E}}{\textsubscript{m}}{
and
}\textbf{{B}}{\textsubscript{m}}{
times meters squared.}}}

{
\textstyleNone{{Thus, we have shown that Maxwell's equations are in
contradiction with Helmholtz's theorem as well, which means that the potential fields defined by Maxwell are
mathematically inconsistent and should therefore be revised.}}}

{
\textstyleNone{{It can be shown}}\footnote{\textstyleNone{ Stowe, P.
and Mingst, B., The Atomic Vortex Hypothesis, a Forgotten Path to Unification,
}\url{http://vixra.org/abs/1310.0237}\textstyleNone{ }\par
}\textstyleNone{{ that by using the 19th Century's atomic vortex
postulate in combination with a superfluid model for the medium, it is possible to construct a single simple integrated
model which covers all major branches of physics including kinetic, fluid, gravitation, relativity, electromagnetism,
thermal, and quantum theory. With this method, it can also be shown that anomalous observations such as Pioneer's drag
and the electron's magnetic moment can be directly accounted for by the model. Furthermore, with this model all units
of measurements are defined in terms of just three fundamental units of measurement: mass, length, and time.}} }

{
It should be noted that there are two distinct levels in this model, with each playing their own role. The first
consists of basic media quanta, which forms a superfluid model for the medium itself. The second describes vortices
within the fluid, which forms a particle model on top of the medium model. The lower base level is assumed to be an (if
not ideal, nearly so) in-viscous superfluid system obeying the defined rules of basic kinetic theory and that is the
model this paper is originally based on, which means that the equations presented in this paper do not depend on the
higher level Atomic Vortex Hypothesis based model. However, during the course of this work it became clear that
viscosity plays a crucial role in our model, which has as consequence that an in-viscous superfluid model is
insufficient to describe the behavior of the medium.}

{
\textstyleNone{{Of course,
}{a}{
}{(viscous)
}{superfluid model can also be described in vector notation and
since this model essentially describes a fluid/gas like medium, we can apply continuum mechanics fluid dynamics vector
calculus methods to re-derive the Maxwell equations from the basic model. As is common practice in continuum mechanics
fluid dynamics vector theory, we can describe its dynamic behavior by working with the medium's flow velocity vector
field}}\footnote{\textstyleNone{ Prieve, D.C., A Course in Fluid Mechanics with Vector Field Theory. Department of
Chemical Engineering, Carnegie Mellon University,
}\url{https://archive.org/details/ACourseInFluidMechanicsWithVectorFieldTheory}\par
}\textstyleNone{{
}\textbf{{v}}{,
with
}\textbf{{v}}{
representing the local average bulk flow velocity. }}}

{
\textstyleNone{{It should be noted that because we use continuum
mechanics, the equations presented in this paper are independent on the detailed description of the constituents of the
medium itself and that there is a lower limit with respect to scale below which the medium can no longer be considered
as a continuum. In that case, the model is no longer applicable, which is a well-known limitation of continuum
mechanics. The Knudson number can be used to estimate this limit. }}}

{
\textstyleNone{{W}{ithin
the fluid dynamics domain, a scalar potential field}{ $\varphi
$}{\textsubscript{f}}{
}{and a vector potential field
}{[D835?][DC00?]}{\textsubscript{f}}{
are generally described for an incompressible fluid (}${\nabla}{\cdot}$\textbf{v}\textsubscript{f }=
0){ with a flow velocity field
}\textbf{{v}}{\textsubscript{f}}{
as follows}}\footnote{ Richardson, S., \& Cornish, A. (1977). Solution of three-dimensional incompressible flow
problems. Journal of Fluid Mechanics, 82(2), 309-319. doi:10.1017/S0022112077000688
\url{https://pdfs.semanticscholar.org/9344/48b028a3a51a7567c2b441b5ca3e49ebb85c.pdf} \par
}\textstyleNone{{ (eq. 17-19):}}}

\begin{equation}
\boldsubformula v_f=\nabla \varphi _f+\nabla \times \boldsubformula A_f
\end{equation}
{
\textstyleNone{{where the velocity potential
}{$\varphi
$}{\textsubscript{f}}{
is a scalar potential field, satisfying the Laplace equation:}}}

\begin{equation}
\nabla ^2\varphi _f=0
\end{equation}
{
\textstyleNone{{and the vorticity potential
}{[D835?][DC00?]}{\textsubscript{f}}{
is a solenoidal (i.e.
}{${\nabla}$.${\cdot}$}\textbf{\textit{{A}}}{\textsubscript{f}}{
= 0) vector potential field satisfying the Poisson equation:}}}

\begin{equation}
\nabla ^2\boldsubformula A_f=-\nabla \times (\nabla \times \boldsubformula A_f)=-\boldsubformula{\omega }_v,
\end{equation}
{
\textstyleNone{{where
}\textbf{{$\omega
$}}{\textsubscript{v}}{=}{${\nabla}$}{{\texttimes}}\textbf{{v}}{\textsubscript{f}}{
 is the }{velocity
}{vorticity field.}}}

{
\textstyleNone{{However, with this definition, the potential fields
are not uniquely defined and the boundary conditions on }{$\varphi
$}{\textsubscript{f 
}}{and
}{[D835?][DC00?]}{\textsubscript{f}}{
depend on the nature of the flow at the boundary of the flow domain and on the topological properties of the flow
domain, respectively.  }}}

{
\textstyleNone{{We can
}{can
}\textit{{attempt}}{
to }{resolve this problem for the general case of a fluid that is
both compressible and rotational by defining a compressible irrotational velocity field
}\textbf{{E}}{\textsubscript{f}}{
for the scalar potential}\textbf{{
}}{$\varphi
$}{\textsubscript{f}}\textbf{{
}}{and an incompressible solenoidal velocity field
}\textbf{{B}}{\textsubscript{f}}{
and associated vorticity field }\textbf{{$\omega
$}}{ for the vector potential
}{[D835?][DC00?]}{\textsubscript{f}}{
using the Helmholtz decomposition and negating the commonly used definition for the velocity potential
}{$\varphi
$}{\textsubscript{f}}{:}}}

\begin{equation}
\boldsubformula v_f=-\nabla \varphi _f+\nabla \times \boldsubformula A_f=\boldsubformula E+\boldsubformula B
\end{equation}
\begin{equation}
\begin{matrix}\hfill \boldsubformula E_f&=&-\nabla \varphi _f\hfill\null \\\hfill \boldsubformula B_f&=&\nabla \times
\boldsubformula A_f\hfill\null \\\hfill \boldsubformula{\omega }&=&\nabla \times \boldsubformula B_f\hfill\null
\end{matrix}
\end{equation}
{
\textstyleNone{{This way, the
}\textbf{\textit{{E}}}{\textsubscript{f}}{
and
}\textbf{\textit{{B}}}{\textsubscript{f}}{
fields
}{describe}{
}{flow velocity fields with a unit of measurement in [m/s]
}{and }{both
the velocity potential and the }{velocity
}{vorticity potential
}{describe
}{fields }{with
a}{ unit of measurement in meters squared per second
[m}{\textsuperscript{2}}{/s].
}{However, the primary vector field
}\textbf{\textit{{F}}}{\textsubscript{f
}}{thus has a
}{unit of measurement in
[m}{\textsuperscript{3}}{/s],
which describes a vector field for a volumetric flow rate or volume velocity. This can be considered as the flow
velocity vector field
}\textbf{{v}}{\textsubscript{f}}{
times a surface
}{S}{
perpendicular to
}\textbf{{v}}{\textsubscript{f}}{
with a surface area proportional to
}\textit{{h}}{\textsuperscript{2}}{
square meters
[m}{\textsuperscript{2}}{],
with
}\textit{{h}}{
the }physical length scale in meters [m]. This results in the zero vector when taking the limit for the length scale
\textit{h} to zero, which is obviously problematic.}}

{
\textstyleNone{So far, we have considered the general mathematical case for the Helmholtz decomposition of any given
vector field \textbf{\textit{F}} as well as its common use in both the electrodynamics and the fluid dynamics domains,
whereby we encountered a number of problems.  In order to resolve these problems and avoid confusion with the various
fields used thus far, let us first introduce a new set of fields
along{ equation
}{(5)}{:}{
}}}

\begin{equation}\label{seq:refText15}
\begin{matrix}\hfill \Pi &=&\nabla \cdot \boldsubformula C\hfill\null \\\hfill \boldsubformula{\Omega }&=&\nabla \times
\boldsubformula C\hfill\null \\\hfill \boldsubformula L&=&-\nabla \Pi \text{ = }-\nabla (\nabla \cdot \boldsubformula
C)\hfill\null \\\hfill \boldsubformula R&=&\nabla \times \boldsubformula{\Omega }\text{ = }\nabla \times (\nabla \times
\boldsubformula C),\hfill\null \end{matrix}
\end{equation}
{
\textstyleNone{{where
}\textbf{\textit{{C}}}{
}{is }{our
primary vector field}{,
}}\href{https://en.wikipedia.org/wiki/Pi_(letter)}{\textstyleNone{\textit{{$\Pi
$}}}}\textstyleNone{{ is
}{the}{ scalar
potential
}{or}{
pressure, }\textbf{\textit{{$\Omega $
}}}}{is the vector potential or angular pressure,
}\textbf{\textit{{L}}}{
is the longitudinal or translational force density and
}\textbf{\textit{{R}}}{
is the rotational or angular force density. Hereby,
}\href{https://en.wikipedia.org/wiki/Pi_(letter)}{\textstyleNone{\textit{{$\Pi
$}}}}\textstyleNone{{
}{and
}\textbf{\textit{{$\Omega
$}}}\textbf{\textit{{
}}}{have a unit of measurement in Pascal [Pa] or Newtons per square
meter
[N/m}{\textsuperscript{2}}{]
and
}\textbf{\textit{{L}}}{
and
}\textbf{\textit{{R}}}{
are in Newtons per cubic meter
[N/m}{\textsuperscript{3}}{].
}\textbf{\textit{{C}}}{
is in Newtons per meter [N/m] or kilograms per second squared
[kg/s}{\textsuperscript{2}}{],
thus representing an as of yet undefined quantity. }{Further down,
we will see that for the medium this unit corresponds to the Ampere, hence the choice for using the symbol
}\textbf{\textit{{C}}}{
}{for ``current''.}}}

{
\textstyleNone{{Let us now
}{consider
}{the 3D generalization of
}{Newton's second law
}{for a
}{substance}{
with a certain mass density}{, expressed in densities or per unit
volume:}}}

\begin{equation}
\boldsubformula f_n=\rho \boldsubformula a=\rho \frac{d\boldsubformula v}{\mathit{dt}}=-{\nabla}\Pi ,
\end{equation}
{
\textstyleNone{{with
}\textbf{\textit{{f}}}{\textsubscript{n}}{
}{the}{ force
density in
[N/m}{\textsuperscript{3}}{],}{
$\rho $ the mass density of the
}{substance}{,
}\textbf{{v}}{
the velocity field,
}\textbf{{a}}{
the acceleration field }{and $\Pi
$}{ the pressure
}{or}{ scalar
}potential field in Pascal [Pa], defined as the divergence of some primary field \textbf{\textit{C}}. Since
\textbf{\textit{C}} should exist according to the Helmholtz decomposition and should have a unit of measurement in
}[kg/s\textsuperscript{2}] or [N/m], we can define \textbf{\textit{C}} as follows:\textstyleNone{  }}

\begin{equation}
\boldsubformula C=\eta \boldsubformula v,
\end{equation}
{
\textstyleNone{{with $\eta
$}}{ the viscosity of the substance in [kg/m-s]. This way, we
obtain} a full 3D generalization of Newton's second law per unit volume, describing not only a longitudinal force
density field \textbf{\textit{L}} but also a rotational or angular force density field \textbf{\textit{R}}: }

\begin{equation}\label{seq:refText18}
\rho \frac{d\boldsubformula v}{\mathit{dt}}=-{\nabla}^2\boldsubformula C=-{\nabla}^2\eta \boldsubformula
v=-(\boldsubformula L+\boldsubformula R).
\end{equation}
This definition also allows us to work with the vector wave equation(3):

\begin{equation}
{\nabla}^2\boldsubformula C\text{ + }k^2\boldsubformula C=\nabla \nabla \cdot \boldsubformula C\text{ - }\nabla \times
\nabla \times \boldsubformula C\text{ + }k^2\boldsubformula C=0.
\end{equation}
{
This is a full 3D vector wave equation, in contrast to the complex wave function that is often used in Quantum
Mechanics. With wave functions there are only two axis, the real and the imaginary, which is simply insufficient to
fully describe phenomena in three dimensions. In other words: current Quantum Mechanics theories lack the required
dimensionality in order to be capable of fully describing the phenomena and are therefore incomplete. }

When we divide equation (18) by mass density $\rho $, we obtain the velocity diffusion equation:

\begin{equation}\label{seq:refText20}
\boldsubformula a=\frac{d\boldsubformula v}{\mathit{dt}}=-{\nabla}^2k\boldsubformula v=-{\nabla}^2\Lambda ,
\end{equation}
with \textbf{\textit{a}} the acceleration field in [m/s\textsuperscript{2}], \textit{k} the diffusivity or kinematic
viscosity, defined by:

\begin{equation}
k=\frac{\eta }{\rho },
\end{equation}
and $\Lambda $ the volumetric acceleration field, defined by: 

\begin{equation}
\Lambda =k\boldsubformula{\mathit{v.}}
\end{equation}
This results in the diffusivity for the medium k having a unit of measurement in meters squared per second
[m\textsuperscript{2}/s] and a value equal to light speed c squared (c\textsuperscript{2}), so there is a per second
[/s] difference in the unit of measurement, suggesting that in our current models the dimensionality of certain
quantities is off by a per second. As we shall see, this has profound consequences for our understanding of physical
reality including the mass-energy equivalence principle. 

This per second difference in units of measurement suggests that in our current models there are a number of problems
involving time derivatives that have not been properly accounted for. When we consider that the solutions of the vector
wave equation are harmonic functions, characterized by sine and cosine functions of time, it becomes clear how these
problems could have arisen. Since the cosine is the time derivative of the sine function and vice versa, there is only
a phase difference between the two. When we consider that all known particles adhere to the wave-particle duality
principle and have characteristic oscillation frequencies that are very high, it becomes clear that the quantum scale
phase differentials between a force acting upon a particle and the resulting (time delayed) acceleration of that
particle are virtually undetectable at the macroscopic level.  

Note that with this diffusion equation, the only units of measurement are the meter and the second, which means that
w\textstyleNone{e have succeeded in separating the dynamics over space and time from the substance (mass density)
that's being diffused over space and time. In other words: with this diffusion equation we have described the quantum
characteristics of spacetime itself. }

Analogous to equation (20), we can also define a second order diffusion equation:

\begin{equation}
\boldsubformula j=\frac{d\boldsubformula a}{\mathit{dt}}=-k{\nabla}^2\boldsubformula a=-k\nabla ^2(-k\nabla
^2\boldsubformula v),
\end{equation}
which we can work out further by multiplying by mass density $\rho $ to define the radiosity or intensity field
\textbf{\textit{I}} in Watts per square meter [W/m\textsuperscript{2}], representing a heat flux density:

\begin{equation}
\rho \boldsubformula j=\rho \frac{d\boldsubformula a}{\mathit{dt}}=-\rho k\nabla ^2\boldsubformula a=-\nabla
^2\boldsubformula I=-\eta \nabla ^2\boldsubformula a=-k\nabla ^2(\boldsubformula L+\boldsubformula R),
\end{equation}
or:

\begin{equation}
\boldsubformula I=k(\boldsubformula L+\boldsubformula R)
\end{equation}
From this, we can derive additional fields analogous to equation (15), which results in fields representing power
density in Watts per cubic meter [W/m\textsuperscript{3}] for the first spatial derivatives and jerk
\textbf{\textit{j}} times mass density in [N/m\textsuperscript{3}{}-s\textsuperscript{3}] for the second spatial
derivatives and thus we find that the spatial derivatives of the intensity field \textbf{\textit{I}} are the time
derivatives of the corresponding spatial derivatives of our primary field \textbf{\textit{C}}.

The process of taking higher order derivatives can be continued indefinitely, whereby for harmonic solutions we end up
with the same results over and over again, resulting in only a phase differential between subsequent results.  

An interesting detail is that the intensity field \textbf{\textit{I}} can also be defined as:

\begin{equation}\label{seq:refText26}
\boldsubformula I=-\kappa \boldsubformula v,
\end{equation}
with $\kappa $ the modulus or elasticity in [Pa] or [kg/m-s\textsuperscript{2}], which has a unit of measurement that
differs by a per second [/s] from the unit of measurement for viscosity \textstyleNone{$\eta $ in [Pa-s] or [kg/m-s].}

This reflects the difference between elastic forces and viscous (shear) forces, namely that the \textstyleNone{ elastic
force is proportional to the amount of deformation, while the viscous one is proportional to the rate of deformation.
So, it appears we can conclude that in physical reality there are no actual static (elastic) forces (at the quantum
level) and that deep down there are only dynamic forces and interactions which are governed by the velocity diffusion
equation(20), whereby what we observe as static forces are in reality the time derivatives of fundamentally viscous
forces.}

This brings us to the mass energy equivalence principle:

\begin{equation}
E=\mathit{mc}^{2,}
\end{equation}
which can now alternatively be formulated by:

\begin{equation}\label{seq:refText28}
L=mk,
\end{equation}
with \textit{L} the angular momentum of a particle in [J/s] or [kg-m\textsuperscript{2}/s] relative to its center of
mass, \textit{m} the mass of the particle and \textit{k} the diffusivity or kinematic viscosity. 

This can be related to the unusual behavior of superfluids such as \textsuperscript{3}He, which spontaneously creates
quantized vortex lines when the container holding the liquid is put into rotation\footnote{  Olli V. Lounasmaa and
Erkki Thuneberg, Vortices in rotating superfluid \textsuperscript{3}He. PNAS July 6, 1999 96 (14) 7760-7767;
\url{https://doi.org/10.1073/pnas.96.14.7760} \par }, thus forming a quantum vortex. This is a hollow core around which
the superfluid flows along an irrotational vortex pattern (i.e. $\nabla \times \boldsubformula v=0$). This flow is
quantized in the sense that the circulation takes on discrete values\footnote{ Whitmore, S C, and Zimmermann, W Jr.,
Observation of Quantized Circulation in Superfluid Helium. United States: N. p., 1968. Web.
doi:10.1103/PhysRev.166.181. \par }. The quantum unit of circulation or quantum circulation constant is \textit{h/m},
with \textit{h} Planck's constant in [J-s] or [kg-m\textsuperscript{2}/s] and \textit{m} the mass of the superfluid
particles in [kg].  Note that Planck{}'s constant has a unit of measurement representing angular momentum.

For the medium, we can equate this quantum circulation constant to \textit{k}, the diffusivity or kinematic viscosity,
which we can now also refer to as the quantum circulation constant, and thus we can compute the mass of an elemental
aether particle along:

\begin{equation}
m_{\mathit{elemental}}=\frac h k,
\end{equation}
which computes to approximately 7.372e-51 kg, about 20 orders of magnitude lighter than the electron. 

When we compute the Compton wavelength for such a particle, we obtain the value of the speed of light c, but with a unit
of measurement in meters [m] rather than velocity [m/s], while it's associated frequency computes to 1 Hertz [Hz].
Since the Compton wavelength of a particle is equal to the wavelength of a photon whose energy is the same as the mass
of that particle along the mass-energy equivalence principle, this puts serious question marks to the mass-energy
equivalence principle in favor of our alternative in equation (28), whereby we conclude that the quantization that is
observed in physics is not a quantization of mass/energy, but one of angular momentum. And since angular momentum is
represented by the magnetic field, we can also conclude that it's the magnetic field that is quantized and that
magnetic field lines are actually irrotational hollow core vortices in a superfluid medium with a circulation equal to
the quantum circulation constant \textit{k}. 

{
Now let us consider \textstyleNone{{the Cauchy momentum equation
without external forces working on the }{fluid:}}}

\begin{equation}
\rho \frac{d\boldsubformula v}{\mathit{dt}}=-\nabla \cdot \boldsubformula P,
\end{equation}
\textstyleNone{{with }} $\boldsubformula P$
\textstyleNone{{ the Cauchy stress tensor, which has a unit of
measurement in
[N/m}{\textsuperscript{2}}{]
or [Pa] and is a central concept in the linear theory of elasticity for continuum solid bodies in static
}}{equilibrium, }when the resultant force and moment on each axis
is equal to zero.{ It can be demonstrated that the components of
the Cauchy stress tensor in every material point in a body satisfy the equilibrium }equations
{and according to the principle of
}\href{https://en.wikipedia.org/wiki/Conservation_of_angular_momentum}{\textstyleInternetlink{conservation of angular
momentum}}{, equilibrium requires that the summation of
}\href{https://en.wikipedia.org/wiki/Torque}{\textstyleInternetlink{moments}}{
with respect to an arbitrary point is zero, which leads to the conclusion that the
}\href{https://en.wikipedia.org/wiki/Stress_(mechanics)#Equilibrium_equations_and_symmetry_of_the_stress_tensor}{\textstyleInternetlink{stress
tensor is symmetric}}{, thus having only six independent stress
components, instead of nine.}

{In our model, we have only four independent stress components,
namely the scalar and vector potentials $\Pi $ and
}\textbf{{$\Omega $}}.

From this momentum equation, the Navier-Stokes equations can be derived, of which the most general one without external
(gravitational) forces is:

\begin{equation}
\rho \frac{D\boldsubformula v}{\mathit{Dt}}=\rho \left(\frac{\partial \boldsubformula v}{\partial t}+\boldsubformula
v\cdot \nabla \boldsubformula v\right)=-\nabla p+\nabla \cdot \left\{\eta \left(\nabla \boldsubformula v+\left(\nabla
\boldsubformula v\right)^T-\frac 2 3\left(\nabla \cdot \boldsubformula v\right)\boldsubformula I\right)+\zeta
\left(\nabla \cdot \boldsubformula v\right)\boldsubformula I\right\},
\end{equation}
\textstyleNone{{with p the pressure, }}\textstyletexhtml{\textbf{I}}
the identity tensor and  $\zeta $  \textstyleNone{{the volume, bulk
or second viscosity. This can be re-written to:}}

\begin{equation}
\rho \frac{\mathit{\delta v}}{\mathit{\delta t}}=-{\nabla}p-\rho \left(v\cdot {\nabla}v\right)+\eta {\nabla}\cdot
\left({\nabla}v+\left({\nabla}v\right)^T\right)+\left(\zeta -\frac{2\eta } 3\right)\left({\nabla}\cdot {\nabla}\cdot
v\right)\mathit{I.}
\end{equation}
\textstyleNone{{This is also a second order equation, whereby
notably for the viscous term }} $\eta {\nabla}{\cdot}\left({\nabla}v+\left({\nabla}v\right)^T\right)$
\textstyleNone{{ the order of the differential operators is
reversed compared to the definition of the second
spa}}\textstyleNone{{tial derivative, the vector Laplace operator,
while for the elastic term, }} $\left(\zeta -\frac{2\eta }
3\right)\left({\nabla}{\cdot}{\nabla}{\cdot}v\right)I${,}\textstyleNone{{
the divergence of the divergence is taken. Also, a separate term is introduced for pressure as well as a convective
term, }} $\rho \left(v{\cdot}{\nabla}v\right)${. All
this}\textstyleNone{{ not only causes the complexity of the
equations to increase dramatically while introducing redundancy in the symmetric stress tensor, it also ignores the
fundamental symmetry between the compressible, irrotational components and the incompressible, solenoidal components as
prescribed by the Helmholtz decomposition.}}

\textstyleNone{{When we compare this with our proposal, we end up
with two fundamentally different approaches:}}

\begin{enumerate}
\item {
\textstyleNone{{A solution that
}{fundamentally}{
only }{has}{
viscosity }{and one fundamental interaction of
Nature,}{ yields harmonic solutions c.q. builds upon deterministic
}{(spherical)
}{harmonics
}{and provides a basis for the observed quantization
}{as well as a
}{3D generalization
of}{ the currently used under-dimensioned wave
functions}{;}}}
\item {
\textstyleNone{{A solution that has both viscosity as well as
elasticity, }{the latter of
}{which builds upon
}{Brownian
}{statistical mechanics and thus requires randomness and is
therefore non-deterministic.  }}}
\end{enumerate}
{\color[rgb]{0.101960786,0.101960786,0.101960786}
\textstyleNone{However, with our solution so far, we have lost the description of elastic behavior and thus our model is
incomplete. This again brings us to the unusual behavior of superfluids, which is currently described with a two-fluid
theory}\footnote{ Russell J. Donnelly, {\textquotedbl}The two-fluid theory and second sound in liquid
helium{\textquotedbl}, Physics Today 62, 34-39 (2009) \url{https://doi.org/10.1063/1.3248499}
\url{https://sites.fas.harvard.edu/~phys191r/References/e1/donnelly2009.pdf} \par }\textstyleNone{. Donnely notes a/o
the following: }}

\begin{enumerate}
\item {\color[rgb]{0.101960786,0.101960786,0.101960786}
\textstyleNone{In superfluid state, liquid helium can flow without friction. A test tube lowered partly into a bath of
helium II will gradually fill by means of a thin film of liquid helium that flows without friction up the tube's outer
wall.}}
\item {\color[rgb]{0.101960786,0.101960786,0.101960786}
\textstyleNone{There is a thermo-mechanical effect. If two containers are connected by a very thin tube that can block
any viscous fluid, an increase in temperature in one container will be accompanied by a rise in pressure, as seen by a
higher liquid level in that container.}}
\item {\color[rgb]{0.101960786,0.101960786,0.101960786}
\textstyleNone{The viscous properties of liquid helium lead to a paradox. The oscillations of a torsion pendulum in
helium II will gradually decay with an apparent viscosity about one-tenth that of air, but if liquid helium is made to
flow through a very fine tube, it will do so with no observable pressure drop---the apparent viscosity is not only
small, it is zero!}}
\end{enumerate}
{\color[rgb]{0.101960786,0.101960786,0.101960786}
\textstyleNone{He also describes ``second sound'', fluctuations of temperature, which according to him ``has turned out
to be an incredibly valuable tool in the study of quantum turbulence'' and provides a condensed summary:}}

{\color[rgb]{0.101960786,0.101960786,0.101960786}
\textstyleNone{\textit{{}``After one of his discussions with London and inspired by the recently discovered effects,
Tisza had the idea that the Bose-condensed fraction of helium II formed a superfluid that could pass through narrow
tubes and thin films withou}\textit{t dissipation. The uncondensed atoms, in contrast, constituted a normal fluid that
was responsible for phenomena such as the damping of pendulums immersed in the fluid. That revolutionary idea demanded
a ``two-fluid'' set of equations of motion and, among other things, predicted not only the existence of ordinary
sound---that is, fluctuations in the density of the fluid---but also fluctuations in entropy or temperature, which were
given the designation ``second sound'' by Russian physicist Lev Landau. By 1938 Tisza's and London's papers had at
least qualitatively explained all the experimental observations available at the time: the viscosity paradox,
frictionless film flow, and the thermo-mechanical effect.''}}}

{\color[rgb]{0.101960786,0.101960786,0.101960786}
\textstyleNone{This leads to the question of whether or not the effects described by the current ``two-fluid'' theory
can also be described by the (spatial derivatives) of the two related fields we have de}\textstyleNone{fined, our
primary field [C] and the intensity field [I], since we already noted that the intensity field [I] does appear to
describe elastic behavior (eq (26)), apart from a per second difference in units of measurement, and that when dealing
with harmonic functions of time, such as those describing elemental particles, it is all too easy to get these time
derivatives mixed up, because there is only a phase differential between the [C] and [I] fields and their respective
spatial derivative fields. }}

{\color[rgb]{0.101960786,0.101960786,0.101960786}
\textstyleNone{Further, since electric current can be associated with both the curl of the magnetic field as well as
with electric resistance and thus dissipation, it seems clear that rather than associating the absence of
dissipation/resistance with the absence of viscosity, this absence should be associated with the absence of vorticity
or turbulence. }}

{\color[rgb]{0.101960786,0.101960786,0.101960786}
\textstyleNone{This brings us to the idea of the ``vortex sponge''}\footnote{ Edward M. Kelly, {\textquotedbl}Maxwell's
Equations as Properties of the Vortex Sponge{\textquotedbl}, American Journal of Physics 31, 785-791 (1963)
\url{https://doi.org/10.1119/1.1969085}  \par }\textstyleNone{,  devised by John Bernoulli in 1736, although in the
shape of vortex tubes rather than ring vortices, which gives rise to elastic behavior of the medium because of momentum
transfer effects arising from the fine-grained vorticity. This idea matches seamlessly to a vortex theory of
atoms}\footnote{ \textcolor{black}{V}an der Laan, G., ``The Vortex Theory of Atoms.'' Master's thesis, Utrecht
University. \url{https://dspace.library.uu.nl/bitstream/handle/1874/260711/Thesis.pdf} \par  }\textstyleNone{, which
was developed after around 1855 a new type of vortex theory emerged, the so-called `vortex sponge theory'. Instead of
viewing atoms as consisting out of small, separate vortices, this type of theory supposed that the ether was completely
filled with tiny vortices. These tiny, close-packed vortices made up large sponge-like structures, which gave this
category of models its name. }}

{\color[rgb]{0.101960786,0.101960786,0.101960786}
\textstyleNone{When we consider the basic idea that particles consist of a number (quantized) vortex rings, we would
then consider these to form such a vortex sponge, especially in the case of crystalline materials such as silicon.
Therefore, we would associate a material substance to such a dynamic vortex sponge rather than considering the aether
itself to be universally filled with tiny vortices.  And since in this view there are no stiff, point-like particles
that bounce onto one another in a random manner, we would also do away with Brownian statistics and consider the
interactions between the vortices to occur along harmonic functions of time.}}

{\color[rgb]{0.101960786,0.101960786,0.101960786}
\textstyleNone{The question then becomes whether or not the static forces we consider at the macroscopic level really
are forces along Newton's third:}}

\begin{equation}
F=ma,
\end{equation}
{\color[rgb]{0.101960786,0.101960786,0.101960786}
\textstyleNone{or are actually time derivatives thereof, yank, along:}}

\begin{equation}
Y=mj,
\end{equation}
{\color[rgb]{0.101960786,0.101960786,0.101960786}
\textstyleNone{with j the jerk, the time derivative of acceleration. The derivative of force with respect to time does
not have a standard term in physics, but the term ``yank'' has recently been proposed in biomechanics}\footnote{ David
C. Lin, Craig P. McGowan, Kyle P. Blum, Lena H. Ting; Yank: the time derivative of force is an important biomechanical
variable in sensorimotor systems. J Exp Biol 15 September 2019; 222 (18): jeb180414. doi:
\url{https://doi.org/10.1242/jeb.180414} \par }\textstyleNone{.}}

\textstyleNone{{This would also offer further insight into what
inertia, resistance to {\textquotedbl}change{\textquotedbl},
}{actually
}{is, because the dynamic viscous forces we have described thus far
are proportional to a rate of
}{deformation}{
and describe }{something dynamic, whereby there is a continuous
flow of mass  along quantized }irrotational{ vortices. In a way,
this can be seen as the opposite of resistance to ``chance'' and could perhaps rather be thought of as conductance of
``change''.}}

{\color[rgb]{0.101960786,0.101960786,0.101960786}
\textstyleNone{Let's illustrate that along the rotating superfluid wherein quantum vortexes are formed. Once a certain
angular speed has been established with the rotating container, a certain number of quantum vortices have formed and
the system is in equilibrium. In that situation, the vortices are irrotational and therefore no vorticity nor
turbulence and thus no resistance nor dissipa}\textstyleNone{tion. In other words: there is a steady-state situation,
which could easily be confused with a {\textquotedbl}static{\textquotedbl} situation, were it not that these the vortex
lines are visible.}}

{\color[rgb]{0.101960786,0.101960786,0.101960786}
\textstyleNone{When we wish to increase the rotation speed of the rotating container, we must exert a
{\textquotedbl}force{\textquotedbl} and thus we introduce turbulence until a new equilibrium is established. This way,
we convert the energy we provide into the rotating superfluid, whereby the steady state situation becomes disturbed and
turbulence is introduced, which results in more quantum vortices forming until eventually the turbulence dies out and a
new equilibrium is established. Thus, from the outside it appears as though the rotating mass in the container resists
change, but in reality it sort of stores {\textquotedbl}change{\textquotedbl} by forming additional vortices until
there is no more turbulence.}}

{\color[rgb]{0.101960786,0.101960786,0.101960786}
\textstyleNone{Now let us look back at equation (20),  the velocity diffusion equation:}}

\begin{equation}
\boldsubformula a=\frac{d\boldsubformula v}{\mathit{dt}}=-{\nabla}^2k\boldsubformula v
\end{equation}
{\color[rgb]{0.101960786,0.101960786,0.101960786}
\textstyleNone{with \textbf{\textit{a}} the acceleration field in [m/s\textsuperscript{2}],
}\textstyleNone{k}\textstyleNone{ the diffusivity, kinematic viscosity or quantum circulation constant. This is an
equation with only meters and seconds and by dividing by velocity we can find that the time derivative operator can be
related to the second spatial derivative by a single constant:}}

\begin{equation}
\frac d{\mathit{dt}}=-k{\nabla}^{2,}
\end{equation}
{\color[rgb]{0.101960786,0.101960786,0.101960786}
\textstyleNone{which suggests space and time are indeed closely related.}}

{\centering\color[rgb]{0.101960786,0.101960786,0.101960786}
\textstyleNone{{}-:-}
\par}

{\color[rgb]{0.101960786,0.101960786,0.101960786}
\textstyleNone{\textbf{Intermezzo}:  my current ``to do'' list, some cut\&pastes from discussions on researchgate:}}

{\color[rgb]{0.101960786,0.101960786,0.101960786}
\textstyleNone{Where we are now is that we can describe both the quantum level as well as the superfluid level (quantum
phenomena at a macroscopic scale) with the same equations, only different parameters like mass density and quantum
circulation constant, whereby we find that fundamentally there are only viscous forces.}}

{\color[rgb]{0.101960786,0.101960786,0.101960786}
\textstyleNone{What we see with superfluids is that when temperature rises (power density increases), elastic behavior
emerges, which is currently described with a two-fluid model.}}

{\color[rgb]{0.101960786,0.101960786,0.101960786}
\textstyleNone{It seems that this effect can be attributed to the formation of some kind of vortex sponge which gives
rise to elastic behavior. And it also seems we can describe the effects this creates within a continuum by the
definition of fields that are derived from the intensity field [I] rather than our primary field [C].}}

{\color[rgb]{0.101960786,0.101960786,0.101960786}
\textstyleNone{The fields that can be defined as the second spatial derivatives of [I] have a unit of measurement
describing the time derivative of force density, which would be yank density.}}

{\color[rgb]{0.101960786,0.101960786,0.101960786}
\textstyleNone{What it appears to come down to is that within current physics Force and Yank have been considered as one
and the same thing, resulting in 3D equations that break the fundamental symmetry demanded by the vector Laplace
operator.}}

{\color[rgb]{0.101960786,0.101960786,0.101960786}
\textstyleNone{So, it seems that there are actually two versions of Newton's law, which have currently been taken
together into one:}}

{\color[rgb]{0.101960786,0.101960786,0.101960786}
\textstyleNone{1) F = m a,}}

{\color[rgb]{0.101960786,0.101960786,0.101960786}
\textstyleNone{2) Y = m j,}}

{\color[rgb]{0.101960786,0.101960786,0.101960786}
\textstyleNone{and the challenge thus comes down to figuring out which one of the two applies where.}}

{\color[rgb]{0.101960786,0.101960786,0.101960786}
\textstyleNone{When we put these quantities in a table:}}

{\color[rgb]{0.101960786,0.101960786,0.101960786}
\textstyleNone{Action:\ \ kg-m[2C6?]2/s.\ \ Momentum mv: kg-m/s.  \ \ Momentum density: \ \ kg/m[2C6?]2-s }}

{\color[rgb]{0.101960786,0.101960786,0.101960786}
\textstyleNone{Energy: \ \ kg-m[2C6?]2/s[2C6?]2\ \ Force  ma: \ \   kg-m/s[2C6?]2  \ \ Force density: 
\ \ kg/m[2C6?]2-s[2C6?]2\ \ }}

{\color[rgb]{0.101960786,0.101960786,0.101960786}
\textstyleNone{Power:  \ \ kg-m[2C6?]2/s[2C6?]3\ \ Yank  mj: \ \   kg-m/s[2C6?]3  \ \ Yank density: 
\ \ kg/m[2C6?]2-s[2C6?]3}}

{\color[rgb]{0.101960786,0.101960786,0.101960786}
\textstyleNone{it also seems that additional fields can be defined to describe action density and its spatial
derivative, momentum density.}}

{\centering\color[rgb]{0.101960786,0.101960786,0.101960786}
\textstyleNone{{}-{}-{}-{}-}
\par}

{\color[rgb]{0.101960786,0.101960786,0.101960786}
\textstyleNone{{\textquotedbl}Sorry, but the root of your special problem is not vector analysis. It is your na\"ive
assumption that you are free in selecting parts of the Navier Stokes equation to handle special
problems.{\textquotedbl}}}

{\color[rgb]{0.101960786,0.101960786,0.101960786}
\textstyleNone{Well, I must confess I was a bit too fast by assuming that because I started from the vector Laplace
operator and all seemed to fit seamlessly, I had solved the puzzle and that the loss of a few independent stress
components was nothing to worry about. So, guilty as charged in that respect.  }}

{\color[rgb]{0.101960786,0.101960786,0.101960786}
\textstyleNone{However, it was not an exercise in selecting parts of Navier Stokes equations that met my needs, it was
an attempt to derive equivalents of Navier Stokes from vector potential theory and to align these with equivalents of
Maxwell and to derive both from one and the same equation, which turned out to represent Newton's third in 3D.}}

{\color[rgb]{0.101960786,0.101960786,0.101960786}
\textstyleNone{Since I was familiar with the scalar and vector potentials used in Maxwell and I found that the terms in
the vector Laplace operator can be written out and define fields that establish a Helmholtz decomposition, I became
convinced that this is the way it should be done. When I searched for usage of a vector potential in fluid dynamics, I
found this paper and not much more: }}

{\color[rgb]{0.101960786,0.101960786,0.101960786}
\url{https://pdfs.semanticscholar.org/9344/48b028a3a51a7567c2b441b5ca3e49ebb85c.pdf}\textstyleNone{ }}

{\color[rgb]{0.101960786,0.101960786,0.101960786}
\textstyleNone{As I wrote in my paper, I attempted to define a primary vector field for the Laplace operator to work on
for these, since that should exist according to the Helmholtz decomposition. It seemed that all I needed to do was
negate the definition for the scalar potential, but then the unit of measurement for the primary field turned out to be
in [m[2C6?]3/s], denoting a volumetric flow velocity, which results in the null vector when taking the limit of the
volume to zero. So that didn't work out very well.}}

{\color[rgb]{0.101960786,0.101960786,0.101960786}
\textstyleNone{After a lot of puzzling, I found a solution that involved viscosity, whereby I found that the kinematic
viscosity nu yielded a value equal to light speed squared for the aether, but a mismatch in units of measurement by a
per second, pointing to problems here and there with time derivatives. When I realized that this constant nu can also
be seen as the quantum circulation constant, I became convinced I'm on the right track and that the thus far mysterious
properties }\textstyleNone{of superfluids (quantum phenomena on a macroscopic scale) offer the key to unlocking the
mysteries of quantum mechanics.}}

{\color[rgb]{0.101960786,0.101960786,0.101960786}
\textstyleNone{{\textquotedbl}The Navier Stokes equations have been derived from momentum conservation. For an
incompressible fluid we get two partial differential equations for density and pressure. For a compressible fluid the
energy balance must be considered, which brings temperature and heat capacity into the game.{\textquotedbl}}}

{\color[rgb]{0.101960786,0.101960786,0.101960786}
\textstyleNone{It is rather interesting that the fields I derived from my primary field [C] do seem to describe an
incompressible fluid (viscous behavior), while we seem to have lost compressibility and that that should bring
temperature and heat capacity into the game.}}

\textstyleNone{{My working hypothesis is that temperature is a
measure of power density and has a unit of measurement in Watts per cubic meter [W/m[2C6?]3]}, but that may not be
correct since Stowe (see below) found a unit in [kg-m/s[2C6?]3]. {
}}

{\color[rgb]{0.101960786,0.101960786,0.101960786}
\textstyleNone{I found a paper regarding superfluids, wherein it is stated that {\textquotedbl}second
sound{\textquotedbl} waves exist in a superfluid, which incorporates the propagation of fluctuations in temperature:}}

{\color[rgb]{0.101960786,0.101960786,0.101960786}
\url{https://sites.fas.harvard.edu/~phys191r/References/e1/donnelly2009.pdf}\textstyleNone{ }}

{\color[rgb]{0.101960786,0.101960786,0.101960786}
\textstyleNone{According to Donnely, this phenomena ``has turned out to be an incredibly valuable tool in the study of
quantum turbulence''.  }}

{\color[rgb]{0.101960786,0.101960786,0.101960786}
\textstyleNone{Thus, we have quite some hints suggesting that elastic behavior, or compressibility, indeed has to do
with the (spatial derivatives of) the intensity field [I] I thus far payed little attention to. I've updated my
overview table and also included another primary field [Q] of which the second spatial derivatives yield momentum
density or mass flux, which I see as another step forward.}}

{\color[rgb]{0.101960786,0.101960786,0.101960786}
\textstyleNone{What I think is an important detail is that the vector Laplace operator is the 3D generalization of the
second spatial derivative, which would be d[2C6?]2/dx[2C6?]2 in 1D. This means that the 3D complexity of the vector
equations we can define with these three vector fields [Q], [C] and [I], such as the vector wave equation, can be
effortlessly reduced to one dimension to describe phenomena like for instance the mechanical behavior of a long rod or
a long thin tube filled with a fluid.  }}

{\color[rgb]{0.101960786,0.101960786,0.101960786}
\textstyleNone{{\textquotedbl}The possible approximations are ``incompressibility'', ``ideal gas'', or even ''perfect
gas'' with a constant heat capacity. Another issue are the boundary conditions inclusive external sources and sinks,
which define the geometry of the considered problem. Finally, the initial values are important.}}

{\color[rgb]{0.101960786,0.101960786,0.101960786}
\textstyleNone{With your approach you stay outside of the terminology used to define Navier Stokes types of
problems.{\textquotedbl}}}

{\color[rgb]{0.101960786,0.101960786,0.101960786}
\textstyleNone{So far, I haven't solved the problem of temperature and black body radiation, but now that I realize the
importance of the intensity field [I] and it's consequence that we have to consider yank rather than force, it seems it
is only a matter of time before we can come full circle.}}

{\color[rgb]{0.101960786,0.101960786,0.101960786}
\textstyleNone{First of all, it is rather interesting that the gas law also involves quantization denoted by n:}}

{\color[rgb]{0.101960786,0.101960786,0.101960786}
\textstyleNone{P V = n Kb T, (eq 1)}}

{\color[rgb]{0.101960786,0.101960786,0.101960786}
\textstyleNone{With T the temperature in Kelvin,}}

{\color[rgb]{0.101960786,0.101960786,0.101960786}
\textstyleNone{P the pressure,}}

{\color[rgb]{0.101960786,0.101960786,0.101960786}
\textstyleNone{V the volume,}}

{\color[rgb]{0.101960786,0.101960786,0.101960786}
\textstyleNone{n the number of quanta,}}

{\color[rgb]{0.101960786,0.101960786,0.101960786}
\textstyleNone{and Kb Boltzmann's constant.}}

{\color[rgb]{0.101960786,0.101960786,0.101960786}
\textstyleNone{Second, I found the work of Paul Stowe very interesting, but very hard to comprehend. On the one hand, he
managed to express all the major constants of nature in terms of just 5 constants and expressed all units of
measurement in just three: mass, length and time, while on the other he managed to write it all down in a manner that I
found very confusing, for instance because he refers to charge q as {\textquotedbl}divergence{\textquotedbl} while
meaning {\textquotedbl}divergence of momentum density{\textquotedbl}:}}

{\color[rgb]{0.101960786,0.101960786,0.101960786}
\url{https://vixra.org/pdf/1310.0237v1.pdf}\textstyleNone{ }}

{\color[rgb]{0.101960786,0.101960786,0.101960786}
\textstyleNone{Nonetheless, valuable insights can be obtained from his work, if only as a starting point for further
considerations. With respect to temperature and the gas law, in his eq. 20 we find a relationship between electrical
charge and Boltzmann's constant: }}

{\color[rgb]{0.101960786,0.101960786,0.101960786}
\textstyleNone{Kb = h/(qc), (eq 2)}}

{\color[rgb]{0.101960786,0.101960786,0.101960786}
\textstyleNone{with q elemental charge and h Planck's constant, which results in the conclusion that the quantization in
the gas law is related to the quantization of the medium, which is governed by the quantum quantization constant nu.
While it is nice that this equation yields the right number, this does not necessarily mean this equation is 100\%
correct as written, but it certainly seems to point in the right direction. }}

{\color[rgb]{0.101960786,0.101960786,0.101960786}
\textstyleNone{Another interesting paper on the subject of black body radiation in relation to aether theory is this one
by C.K. Thornhill, which gives a valuable starting point for deriving Planck's law:}}

{\color[rgb]{0.101960786,0.101960786,0.101960786}
\url{https://etherphysics.net/CKT1.pdf}\textstyleNone{ }}

{\color[rgb]{0.101960786,0.101960786,0.101960786}
\textstyleNone{His main argument:}}

{\color[rgb]{0.101960786,0.101960786,0.101960786}
\textstyleNone{{\textquotedbl}Another argument against the existence of a physical ethereal medium is that Planck's
empirical formula, for the energy distribution in a black-body radiation field, cannot be derived from the kinetic
theory of a gas with Maxwellian statistics. Indeed, it is well-known that kinetic theory and Maxwellian statistics lead
to an energy distribution which is a sum of Wien-type distributions, for a gas mixture with any number of different
kinds of atoms or molecules. But this only establishes the impossibility of so deriving Planck's distribution for a gas
with a finite variety of atoms or molecules. To assert the complete impossibility of so deriving Planck's distribution
it is essential to eliminate the case of a gas with an infinite variety of atoms or molecules, i .e . infinite in a
mathematical sense, but physically, in practice, a very large variety. The burden of the present paper is to show that
this possibility cannot be eliminated, but rather that it permits a far simpler derivation of Planck's energy
distribution than has been given anywhere heretofore.{\textquotedbl}}}

{\color[rgb]{0.101960786,0.101960786,0.101960786}
\textstyleNone{What is interesting, is that he found a relationship between the adiabatic index $\omega $ and the number
of degrees of freedom $\alpha $ of (aether) particles, which leads to the conclusion that $\alpha $ must be equal to 6
and he concludes:}}

{\color[rgb]{0.101960786,0.101960786,0.101960786}
\textstyleNone{{\textquotedbl}Thus, the quest for a gas-like ethereal medium, satisfying Planck's form for the energy
distribution, is directed to an ideal gas formed by an infinite variety of particles, all having six degrees of
freedom.{\textquotedbl}}}

{\color[rgb]{0.101960786,0.101960786,0.101960786}
\textstyleNone{It is this adiabatic index which provides a relationship to heat capacity, since it is also known as the
heat capacity ratio:}}

{\color[rgb]{0.101960786,0.101960786,0.101960786}
\url{https://en.wikipedia.org/wiki/Heat_capacity_ratio}\textstyleNone{ }}

{\color[rgb]{0.101960786,0.101960786,0.101960786}
\textstyleNone{{\textquotedbl}In thermal physics and thermodynamics, the heat capacity ratio, also known as the
adiabatic index, the ratio of specific heats, or Laplace's coefficient, is the ratio of the heat capacity at constant
pressure (C\_P) to heat capacity at constant volume (C\_V).{\textquotedbl}}}

{\color[rgb]{0.101960786,0.101960786,0.101960786}
\textstyleNone{So, while we clearly have not yet cracked the whole nut, it seems to me we are on the right track towards
the formulation of a {\textquotedbl}theory of everything{\textquotedbl}, that holy grail that has thus far proven to be
unreachable, which I'm sure will turn out to be attributable to ignoring the implications of the vector Laplace
operator.}}

{\color[rgb]{0.101960786,0.101960786,0.101960786}
\textstyleNone{Personally, I have no doubt both the weak and strong nuclear forces can be fully accounted for by our
model c.q. electromagnetic forces, once completely worked out, and that the gravitational force also propagates through
the aether, as actually confirmed by the Michelson-Morley experiment, so that we will end up with a model that is much,
much simpler and only has one fundamental interaction of nature. }}

{\color[rgb]{0.101960786,0.101960786,0.101960786}
\textstyleNone{To illustrate the argument that the nuclear forces can be fully accounted for by electromagnetic forces,
I wholeheartedly recommend the experimental work of David LaPoint, who shows this in his laboratory:}}

{\color[rgb]{0.101960786,0.101960786,0.101960786}
\url{https://youtu.be/siMFfNhn6dk}\textstyleNone{  }}

{\color[rgb]{0.101960786,0.101960786,0.101960786}
\textstyleNone{[end intermezzo]}}

{\centering\color[rgb]{0.101960786,0.101960786,0.101960786}
\textstyleNone{{}-:-}
\par}

{\color[rgb]{0.101960786,0.101960786,0.101960786}
\textstyleNone{So far, we have shown that it is possible to derive a complete and mathematically consistent set of
fields from a single equation, the 3D generalization of Newton's second law, by using the LaPlace operator and working
out the terms thereof. With this equation, we can use the vector wave equation, which has harmonic solutions, just like
the wave function currently used in Quantum Mechanics. This makes it possible to extend the current Quantum Mechanical
wave function solutions into full 3D solutions in a manner that maintains the fundamental symmetry of the Helmholtz
decomposition within a framework of uniquely defined fields without gauge freedom. We have also shown that we can
decouple the dynamics of the medium from it's substance, mass density, with the velocity diffusion equation which
reveals that the dynamics of the medium are governed by a single constant }\textstyleNone{k}\textstyleNone{, the
quantum circulation constant. And we have shown that we can take higher order derivatives of these equations over and
over again, resulting in only phase differentials for the resulting vector spherical harmonic solutions.}}

{\color[rgb]{0.101960786,0.101960786,0.101960786}
\textstyleNone{What this comes down to is that we have come to a deeper model of physical reality, which reveals a
number of intricate relationships between various fields defined so far, whereby the quantum circulation constant
}\textstyleNone{k}\textstyleNone{ determines that at the quantum level there is an intricate balance between
translational and angular momentum. This ultimately governs the possible har}\textstyleNone{monic solutions that can
exist in the shape of particles, the oscillating dynamic structures that can be described by the vector spherical
harmonics.  }}

{\color[rgb]{0.101960786,0.101960786,0.101960786}
\textstyleNone{This model offers a new tangible basis for theoretical physics that may eventually very well lead to an
an integrated ``theory of everything'', which is however by no means an easy task. }}

{\color[rgb]{0.101960786,0.101960786,0.101960786}
\textstyleNone{So far, it has proven to be very challenging even to integrate Maxwell's equations with this basis in a
manner that is completely consistent with the current model and it's units of measurement. Maxwell's equations
essentially describe a phenomenological model that is based upon the assumption that some kind of fundamental quantity
called ``charge'' exists, to which a unit of measurement in Coulombs [C] has been assigned. All of the units of
measurement within the electromagnetic domain can be derived from the Coulomb within this model, but there is no
definition of what charge actually is nor what current actually is. Also, there is no explanation for why charge is
considered to be polarized. }}

{\color[rgb]{0.101960786,0.101960786,0.101960786}
\textstyleNone{However, the model presented thus far has as big advantage that it describes a fluid-like medium and thus
we can use fluid dynamics phenomena as analogies in our analysis. }}

{\color[rgb]{0.101960786,0.101960786,0.101960786}
\textstyleNone{Let us start with Ampere's original law to define current density \textbf{\textit{J}}:}}

\begin{equation}\label{seq:refText37}
\boldsubformula J=-\nabla \times \boldsubformula{\mathit{R.}}
\end{equation}
And let us provide an overview of the fields defined thus far, along with their units of measurement:

\begin{flushleft}
\tablefirsthead{}
\tablehead{}
\tabletail{}
\tablelasttail{}
\begin{supertabular}{|m{1.737cm}|m{1.763cm}|m{3.698cm}|m{3.7cm}|m{4.0090003cm}|}
\hline
 &
\textbf{$\Lambda $} =  \textit{k} \textbf{v} &
\textbf{Q} =  $\tau $ \textit{k}\textit{ $\rho $ }\textbf{v} &
\textbf{C} = $\eta $ \textbf{v} =  \textit{k}\textit{ $\rho $ }\textbf{\textit{v}} &
\textbf{I} = $\eta $ \textbf{a} = \textit{k}\textit{ $\rho $ }\textbf{\textit{a}}\textit{ =
}\textit{k}(\textbf{L}+\textbf{R})\\\hline
{\bfseries $\Lambda $, }

{\bfseries Q, }

{\bfseries C, }

{\bfseries I} &
[m\textsuperscript{3}/s\textsuperscript{2}]

 &
[kg/s], [N-s/m], [J-s/m\textsuperscript{2}], [Pa-s-m], [C]

(charge) &
[kg/s\textsuperscript{2}], [N/m]

[J/m\textsuperscript{2}], [Pa-m], [A]

(current, action flux) &
[kg/s\textsuperscript{3}], [N/m-s], [J/m\textsuperscript{2}{}-s], [Pa-m/s], [W/m\textsuperscript{2}] 

(radiosity \textbf{J}\textsubscript{e }, intensity \textbf{I, }

energy flux)\\\hline
\textbf{S,  $\Sigma $}

\href{https://en.wikipedia.org/wiki/Pi_(letter)}{$\Pi $}, \textbf{$\Omega $}

\textbf{T}, \textbf{$\chi $} &
[m\textsuperscript{2}/s\textsuperscript{2}] &
[kg/m-s], [N-s/m\textsuperscript{2}], [J-s/m\textsuperscript{3}], [Pa-s], [V-s]  

(action density,

momentum density flux) &
[kg/m-s\textsuperscript{2}], [N/m\textsuperscript{2}], [J/m\textsuperscript{3}], [Pa], [A/m], [V]

(energy density,

momentum flux, 

force density flux, 

pressure) &
[kg/m-s\textsuperscript{3}], [N/m\textsuperscript{2}{}-s], [J/m\textsuperscript{3}{}-s], [Pa/s],
[W/m\textsuperscript{3}], [K]

(power or heat density, 

force flux,

yank density flux, 

temperature)\\\hline
\textbf{M, $\Lambda $}

\textbf{L}, \textbf{R}

{\bfseries Y, $\Psi $} &
[m/s\textsuperscript{2}]  

(\textbf{a} = d\textbf{v}/dt,

acceleration) &
[kg/m\textsuperscript{2}{}-s], [N-s/m\textsuperscript{3}], [Pa-s/m]  

($\rho $ \textbf{v}, momentum den sity, mass flux) &
[kg/m\textsuperscript{2}{}-s\textsuperscript{2}], [N/m\textsuperscript{3}], [Pa/m], [A/m\textsuperscript{2}],
[C/m\textsuperscript{2}{}-s]

($\rho $ \textbf{a}, force density, charge flux) &
[kg/m\textsuperscript{2}{}-s\textsuperscript{3}], [N/m\textsuperscript{3}{}-s], [J/m\textsuperscript{4}{}-s], [Pa/m-s],
[J/m\textsuperscript{4}{}-s], [W/m\textsuperscript{4}]

($\rho $ \textbf{j}, yank density, current flux)\\\hline
{\bfseries J\textmd{ = curl} R}

(electric current density) &
 &
 &
[kg/m\textsuperscript{3}{}-s\textsuperscript{2}], [N/m\textsuperscript{4}], [Pa/m\textsuperscript{2}],
[A/m\textsuperscript{3}] 

(d\textsuperscript{2}$\rho $/dt\textsuperscript{2}) &
\\\hline
\end{supertabular}
\end{flushleft}
{\raggedleft\color[rgb]{0.101960786,0.101960786,0.101960786}
\textstyleNone{Table 1, overview of fields defined thus far.}
\par}

{
\textstyleNone{{This way, we would think of the electric field as
being described by
}\textbf{\textit{{L}}}{,
the translational force density field, and the magnetic field as being described by
}\textbf{\textit{{R}}}{,
the angular force density field. And thus current would represent vorticity, which aligns pretty well with observations
such as Elmore's non-radiating guided surface wave}}\footnote{ Elmore, G., Introduction to the Propagating Wave on a
Single Conductor, \url{http://www.corridor.biz/FullArticle.pdf}\par
}\textstyleNone{{. From equation
}{(37)}{, this
gives us a }{unit of measurement in kilograms per second square
[kg/s}{\textsuperscript{2}}{]
}{for the Ampere
}{and we can define the Ampere as well as the Coulomb by:}}}

\begin{center}
\tablefirsthead{\centering 1 Ampere = 1 kilogram per second squared. &
\raggedleft\arraybslash (\stepcounter{Text}{\theText})\\}
\tablehead{\centering 1 Ampere = 1 kilogram per second squared. &
\raggedleft\arraybslash (\stepcounter{Text}{\theText})\\}
\tabletail{}
\tablelasttail{}
\begin{supertabular}{m{14.024cm}m{1.578cm}}

\end{supertabular}
\end{center}
\begin{center}
\tablefirsthead{\centering 1 Coulomb = 1 kilogram per second. &
\raggedleft\arraybslash (\stepcounter{Text}{\theText})\\}
\tablehead{\centering 1 Coulomb = 1 kilogram per second. &
\raggedleft\arraybslash (\stepcounter{Text}{\theText})\\}
\tabletail{}
\tablelasttail{}
\begin{supertabular}{m{14.024cm}m{1.578cm}}

\end{supertabular}
\end{center}
{\color[rgb]{0.101960786,0.101960786,0.101960786}
\textstyleNone{We can subsequently define charge density as the divergence of momentum density:}}

\begin{equation}
\rho _q=\nabla \cdot (\rho \boldsubformula v)=\frac 1 k\nabla \cdot (\boldsubformula C)=\frac 1 k\Pi ,
\end{equation}
\textstyleNone{{resulting in a unit of measurement for charge
density}} $\rho _q$ \textstyleNone{{in kilograms per cubic meter
per second
[kg/m}{\textsuperscript{3}}{{}-s]}}\textstyleNone{{,
}{which leads to the conclusion that charge density represents the
time derivative of mass density.}}

\textstyleNone{{With this definition, the charge to mass ratio of a
particle results in a unit of measurement in per second or Hertz [Hz], yielding a characteristic
}{longitudinal
}{oscillation frequency
}{for such a
particle}{. For the electron, this frequency computes to
approximately
175.}{88}{ GHz,
}{which falls within 10\% of the calculated spectral radiance
dE$\nu $/d$\nu $ in the observed cosmic background radiation which peaks at 160.23 GHz and is
}{calculated from
}{a measured CMB temperature of approximately }2.725
{K}}\footnote{\textstyleNone{ Stowe, P. and Mingst, B., The Atomic
Vortex Hypothesis, a Forgotten Path to Unification, }\url{http://vixra.org/abs/1310.0237}\textstyleNone{ }\par
}\textstyleNone{{\textsuperscript{
}}{suggesting a possible connection. }}

\textstyleNone{{This suggestion leads to the idea that even though
we can describe the medium itself as a superfluid, we cannot consider even the vacuum in outer space as devoid from any
particles, }{disturbances or (zero point) energy
}{and }{thus we
can consider it to have}{ a certain charge density $\rho
$}{\textsubscript{q}}{\textsubscript{b}}{\textsubscript{0}}{,
}{a background charge density, which would be
depend}{ing}{
on the material }{or medium we are working with, just like the
permeability and permittivity are. }}

\textstyleNone{{This way, we can
}{defin}{e}{
the electric field
}\textbf{\textit{{E}}}{
as follows}}:

\begin{equation}
\boldsubformula E=\frac 1{\rho _q}\boldsubformula L=k\frac{\boldsubformula L}{\boldsubformula{\Pi }},
\end{equation}
with \textbf{\textit{L}} as defined in equation (15) and
\textstyleNone{{$\rho
$}{\textsubscript{q}}}\textstyleNone{{\textsubscript{
}}}the charge density, resulting in a unit of measurement for the electric field \textbf{\textit{E}} in meters per
second [m/s]\textstyleNone{. }Coulomb's law then becomes:

\begin{equation}
\boldsubformula F=q\boldsubformula E=\frac q{\rho _q}\boldsubformula L=qk\frac{\boldsubformula L}{\boldsubformula{\Pi
}}.
\end{equation}
The electric (scalar) potential $\varphi $ can subsequently be defined as:

\begin{equation}
\varphi =\frac 1{\rho _q}\Pi =k=\frac{\eta }{\rho },
\end{equation}
with $\Pi $ the scalar pressure in Pascal [Pa] as defined in equation (15), yielding a unit of measurement in meters
squared per second [m\textsuperscript{2}/s] for the scalar electric potential $\varphi $ and thus we can
\textstyleNone{define the Volt as:}

\begin{center}
\tablefirsthead{\centering 1 Volt = 1 square meter per second. &
\raggedleft\arraybslash (\stepcounter{Text}{\theText})\\}
\tablehead{\centering 1 Volt = 1 square meter per second. &
\raggedleft\arraybslash (\stepcounter{Text}{\theText})\\}
\tabletail{}
\tablelasttail{}
\begin{supertabular}{m{14.024cm}m{1.578cm}}

\end{supertabular}
\end{center}
We can now also work out the unit of measurement for permittivity $\varepsilon $, which has an SI unit in
[C\textsuperscript{2}/N-m\textsuperscript{2}]. By substitution we find that this results in a unit of measurement in
kilograms per cubic meter [kg/m\textsuperscript{3}] and we can equate the mass density of the medium $\rho $ to its
permittivity:

\begin{equation}
\rho =\epsilon 
\end{equation}
For the magnetic field, we start out at the unit of measurement for permeability $\mu $, which is defined in SI units as
Newtons per Ampere squared [N/A\textsuperscript{2}]. By substitution we find that this corresponds
[m-s\textsuperscript{2}/kg], the inverse of the modulus/elasticity in [Pa] or [kg/m-s\textsuperscript{2}]. The latter
differs by a per second to the unit of measurement for viscosity $\eta $ in [Pa-s] or [kg/m-s], the same difference we
encountered earlier and which led us to conclude that in our current models the dimensionality of certain quantities is
off by a per second. Therefore, we define the \textit{value} of viscosity $\eta $ but not its unit of measurement by:

\begin{equation}
\eta =\frac 1{\mu }.
\end{equation}
We can now define the magnetic field strength:

\begin{equation}
\boldsubformula H=\boldsubformula R,
\end{equation}
with \textbf{\textit{R}} the angular force density in Newton per cubic meter [N/m\textsuperscript{3}], resulting in a
unit of measurement for the magnetic field strength in Ampere per meter squared [A/m\textsuperscript{2}], which differs
from the SI definition which is in Ampere per meter [A/m]. 

The magnetic flux density then becomes:

\begin{equation}
\boldsubformula B=\mu \boldsubformula H=\mu \boldsubformula R,
\end{equation}
and has a unit of measurement in per meter [/m].

The magnetic (vector) potential \textbf{\textit{A}} can subsequently be defined as:

\begin{equation}
\boldsubformula A=\Omega ,
\end{equation}
with \textbf{\textit{$\Omega $}} the angular vector pressure in Pascal [Pa].

{
\textstyleNone{This leaves us with a problem in the dimensionality of the Lorentz force, however, which is not easily
resolved in a satisfactory manner, although dimensionally, we can resolve the problem by defining the Lorentz force
as:}}

\begin{equation}
\boldsubformula F_L=q\lambda \boldsubformula{(v\times B)}=mc\boldsubformula{(v\times \boldsubformula B)},
\end{equation}
whereby $\lambda $ is the wavelength of the particle along $\lambda $=c/f. With f=q/m we then obtain q$\lambda $=mc.

This brings us in the situation whereby we have obtained a fluid dynamics medium model that is capable of bridging the
gap between the Quantum Mechanic and macroscopic worlds in a deterministic manner, but leaves us with open questions
around the detailed nature of the Coulomb and Lorentz forces, especially in relation to the nature of charged particles
and their mass/charge ratios. 

However, it is clear that the irrotational vortex plays a dominant role in magnetics and these can also form closed loop
rings, which explains why magnetic field lines are always closed. This suggests that toroidal ring models like
Parson's\footnote{Alfred L. Parson, {\textquotedbl}A Magneton Theory of the Structure of the Atom{\textquotedbl},
Smithsonian Miscellaneous Collection, Pub 2371, 80pp (Nov 1915).  \par
https://ia802702.us.archive.org/35/items/amagnetontheory00parsgoog/amagnetontheory00parsgoog.pdf\par }\textsuperscript{
} can be integrated with our model, especially because solid spherical harmonics can be expressed as series of toroidal
harmonics and vice versa\footnote{ Matt Majic, Eric C. Le Ru, ``Relationships between solid spherical and toroidal
harmonics'', \href{https://arxiv.org/abs/1802.03484}{arXiv:1802.03484}. https://arxiv.org/abs/1802.03484\par
}\textsuperscript{ }and it is known that the solutions to the vector wave equation are the spherical harmonics.

When we assume that particles can indeed be considered as consisting of a number of closed loop hollow core vortex
rings, then the physics of the vortex ring can also be expected to provide further insight in the nature of the Lorentz
force working on charged particles. It is for example known that a vortex ring moves forward with its own self-induced
velocity v\footnote{ SULLIVAN, I., NIEMELA, J., HERSHBERGER, R., BOLSTER, D., \& DONNELLY, R. (2008). Dynamics of thin
vortex rings. Journal of Fluid Mechanics, 609, 319-347. doi:10.1017/S0022112008002292 
\url{https://www.researchgate.net/publication/232025984} \par }. And since a vortex ring has two axis of rotation,
poloidal and toroidal, this could also offer an explanation for the existence of the polarization currently attributed
to charge.  

{
\textstyleNone{{Either way,
s}{ince all
}{our}{ fields
are uniquely defined as solutions of the vector Laplace equation,
}{we can establish that with deriving all fields from equation
}{(18)}{, we
have eliminated ``gauge freedom'' and since we know these equations can be transformed using the Galilean coordinate
transform, we have also eliminated the need for the Lorentz transform
}{and are thus no longer bound to the universal speed
limit}{.}}}

{
\textstyleNone{{With this application of the fundamental theorem of
vector calculus, we have thus come to a revised version of the Maxwell equations that not only promises to resolve all
of the problems that have been found over the years, we also obtain a model that is easy to interpret and can be easily
simulated and visualized with finite-difference time-domain methods (FTDT) as well. }}}

{
\textstyleNone{{Now let us consider the difference between the
definition we found for
}\textbf{{E}}{
and the corresponding definition in Maxwell's equations:}}}

\begin{equation}
\boldsubformula E_m=-\nabla \varphi _m-\frac{\partial \boldsubformula A_m}{\partial t},
\end{equation}
{
\textstyleNone{{When considered from the presented perspective, this
is what breaks the fundamental result of Helmholtz' decomposition, namely the decomposition into a rotation free
translational component and a divergence free rotational component, since
}\textbf{{A}}{\textsubscript{m}}{
is not rotation free and therefore neither is its time derivative.  }}}

{
\textstyleNone{{When taking the curl on both sides of this equation,
we obtain the Maxwell-Faraday equation, representing Faraday's law of induction:}}}

\begin{equation}
\nabla \times \boldsubformula E_m=-\frac{\partial \boldsubformula B_m}{\partial t},
\end{equation}
{
\textstyleNone{{Faraday's law of induction is a basic law of
electromagnetism predicting how a magnetic field will interact with an electric circuit to produce an electromotive
force (EMF), which is thus a law that applies at the macroscopic level. It is clear that this law should not be
entangled with a model for the medium and therefore our revision should be preferred.  }}}

{\centering
\textstyleNone{\textbf{{Discussion and Conclusions}}}
\par}

{
\textstyleNone{{We have shown that the terms in the Laplace operator
can be written out to define a complete and mathematically consistent whole of four closely related vector fields which
by definition form solutions to the vector Laplace equation, a result that has }tremendous consequences for both the
analytical analysis of the electromagnetic field as well as fluid dynamics vector }\textstyleNone{theory, such as
weather forecasting, oceanography and mechanical engineering. The symmetry between the fields thus defined is
fundamental and has been mathematically proven to be correct, so it is vital to maintain this fundamental symmetry in
our physics equations.  }}

{\color[rgb]{0.101960786,0.101960786,0.101960786}
\textstyleNone{We have also shown that we can decouple the dynamics of the medium from it's substance, mass density,
with the velocity diffusion equation which reveals that the dynamics of the medium are governed by a single constant
}\textstyleNone{k}\textstyleNone{, the quantum circulation constant. And we have shown that we can take higher order
derivatives of these equations over and over again, resulting in only phase differentials for the resulting vector
spherical harmonic solutions.}}

{
\textstyleNone{{Revising Maxwell equations by deriving directly from
a superfluid medium model using the Laplace operator, we have called upon vector theory for an ideal, compressible,
viscous Newtonian
}{super}{fluid
that has led to equations which are known to be mathematically consistent, are known to be free of singularities and
are invariant to the Galilean transform as well. This results in an integrated model which has only three fundamental
units of measurement: mass, length and time and also explains what ``charge'' is: a compression/decompression
oscillation of ``charged'' particles. }}}

{
\textstyleNone{{As is known from fluid dynamics, these revised
Maxwell equations predict three types of wave phenomena, which we can easily relate to the observed phenomena:}}}

\begin{enumerate}
\item {
\textstyleNone{{Longitudinal pressure waves, Tesla's superluminal
waves}}\footnote{\textstyleNone{ Tesla, N. (1900). Art of transmitting electrical energy through the natural mediums.
US patent No. 787,412. In this patent, Tesla disclosed a wireless system by which he was able to transmit electric
energy along the earth's surface and he established that the current of his transmitter passed over the earth's surface
with a mean velocity 471,240 kilometers per second, about $\pi $/2 times the speed of light.}\par
}\textstyleNone{{ c.q. the super luminal longitudinal dielectric
mode, which he found to propagate at a speed of }471,240 kilometers
{per second, within 0.1\% of $\pi $/2 times the speed of light. The
factor $\pi $/2 coincides with the situation whereby the theoretical reactance of a shorted lossless transmission line
goes to infinity}}\footnote{ Knight, David. (2016). The self-resonance and self-capacitance of solenoid coils:
applicable theory, models and calculation methods.. 10.13140/RG.2.1.1472.0887. 
\url{https://www.researchgate.net/publication/301824613} \par
}\textstyleNone{{ (eq 1.2) and thus does not support an
electromagnetic wave propagation mode; }}}
\end{enumerate}
\begin{enumerate}
\item {
\textstyleNone{{{}``Transverse'' ``water'' surface waves, occurring
at the boundary of two media with different densities such as the metal surfaces of an antenna and air, aka the ``near
field'', }{Elmore's
}{non-radiating surface waves that
}{have }{been
shown to be guidable along a
completel}{y}{
unshielded conductor}}\footnote{ Elmore, G., Introduction to the Propagating Wave on a Single Conductor,
\url{http://www.corridor.biz/FullArticle.pdf}\par
}\textstyleNone{{;}}}
\end{enumerate}
\begin{enumerate}
\item {
\textstyleNone{{Vortices and/or vortex rings, the ``far field'',
which is known to be quantized and to incorporate a thus far mysterious mixture of ``particle'' and ``wave'' properties
}{aka
``photons''}{, the so called ``wave particle duality'' principle.
}}}
\end{enumerate}
{
\textstyleNone{{Even though the actual wave equations for these
three wave types still need to be derived, we can already conclude these to exist and predict a number of their
characteristics, because of the integration of the electromagnetic domain with the fluid dynamics domain. The latter
has a tremendous advantage, namely that dynamic phenomena known to occur in fluids and gasses can be considered to also
occur in the medium.}}}

{\centering
\textstyleNone{\textbf{{Further Research}}}
\par}

{
\textstyleNone{\textbf{{Theoretical}}}}

{
\textstyleNone{{While the revised Maxwell equations presented in
this paper describe the motions of the medium accurately in principle, the actual wave equations for the three
predicted wave types still need to be derived and worked out. This is particularly complicated for the ``transverse''
``water'' surface wave, because of the fact that in current fluid dynamics theory the potential fields have not been
defined along the Helmholtz decomposition defined by the vector Laplacian as we proposed, which leads to non-uniquely
defined fields and associated }}\textstyleNone{{problems with
boundary conditions. In order to derive a wave equation for the ``transverse'' surface wave, the incompressibility
constraint would have to be removed from the Saint-Venant equations}}\footnote{
\textcolor[rgb]{0.09411765,0.09411765,0.09019608}{Farge, M., \& Sadourny, R. (1989). Wave-vortex dynamics in rotating
shallow
water.}\textstyleappleconvertedspace{\textcolor[rgb]{0.09411765,0.09411765,0.09019608}{~}}\textit{\textcolor[rgb]{0.09411765,0.09411765,0.09019608}{Journal
of Fluid
Mechanics,}}\textstyleappleconvertedspace{\textcolor[rgb]{0.09411765,0.09411765,0.09019608}{~}}\textit{\textcolor[rgb]{0.09411765,0.09411765,0.09019608}{206}}\textcolor[rgb]{0.09411765,0.09411765,0.09019608}{,
433-462. doi:10.1017/S0022112089002351 }\par }\textstyleNone{{ and
these would subsequently need to be fully worked out using vector calculus methods. }}}

{
\textstyleNone{{Furthermore, we have also argued that Faraday's law
should not be entangled with the model for the medium, which leaves us without revised equations for Faraday's law of
induction. This leads to the question of why a DC current trough a wire loop results in a magnetic field, but the
magnetic field of a permanent magnet does not induce a current in a wire wound around it. A similar question arises
when a (neodymium) magnet is used as an electrode in an electrolysis experiment, which results in a vortex becoming
visible in the electrolyte above the magnet.}}}

{
\textstyleNone{{It is expected the answers to these questions as
well as Faraday's law of induction can be worked out by considering the physics of the irrotational vortex, given that
we found that the current density is actually one and the same thing as the vorticity of the medium, apart from a
constant. In the absence of external forces, a vortex evolves fairly quickly toward the irrotational flow pattern,
where the flow velocity
}\textbf{{v}}{
is inversely proportional to the distance r. The fluid motion in a vortex creates a dynamic pressure that is lowest in
the core region, closest to the axis, and increases as one moves away from it. It is the gradient of this pressure that
forces the fluid to follow a curved path around the axis and it is this pressure gradient that is directly related to
the velocity potential $\Phi
$}{\textsubscript{fd}}{
c.q. the velocity field component
}\textbf{{E}}{\textsubscript{fd}}{.}}}

{
\textstyleNone{\textbf{{Practical}}}}

{
\textstyleNone{{The revised Maxwell equations presented in this
paper open the possibilities of further considerations and research into the properties of the dielectric and
gravitational fields and associated wave phenomena. Because both of these fields are considered as one and the same
within the above presented revised Maxwell paradigm, a wide range of possible applications become conceivable, some of
which are hardly imaginable from within the current paradigm and/or are highly speculative while others are more
straightforward.}}}

{
\textstyleNone{\textit{{Superluminal communication}}}}

{
\textstyleNone{{This is the most direct application of the theory
presented in this paper, which is supported by a number of sources mentioned in the abstract, the oldest of which dates
back to 1834, some theoretical methods}}\footnote{\textstyleNone{ Shore, G.M.. (1998). `Faster than light' photons in
gravitational fields --- Causality, anomalies and horizons. Nuclear Physics B. 460. 379-394.
10.1016/0550-3213(95)00646-X. }\url{https://www.researchgate.net/publication/223392267}\textstyleInternetlink{ }\par
}\textstyleNone{{\textsuperscript{,}}}\footnote{\textstyleNone{
Nanni, Luca. (2019). ``Evanescent waves and superluminal behavior of matter''. International Journal of Modern Physics
A. 34. 1950141. 10.1142/S0217751X19501410.
}\url{https://www.researchgate.net/publication/335719399}\textstyleInternetlink{ }\par
}\textstyleNone{{\textsuperscript{,}}}\footnote{\textstyleNone{
Pav\v{s}i\v{c}, Matej. (2011). Extra Time Like Dimensions, Superluminal Motion, and Dark Matter.
}\url{https://www.researchgate.net/publication/51946961}\textstyleInternetlink{ }\par
}\textstyleNone{{\textsuperscript{,}}}\footnote{ Aginian, M.A. \&
Arutunian, S. \& Lazareva, E.G. \& Margaryan, A.V. (2018). Superluminal Synchotron Radiation. Resource-Efficient
Technologies. 19-25. 10.18799/24056537/2018/4/216. \url{https://www.researchgate.net/publication/329985374}  \par
}\textstyleNone{{ as well as some preliminary experimental work by
the author}}\footnote{ Lammertink, A.H..  Did I actually measure a superluminous signal thus disproving the relativity
theory? \par
\url{https://www.researchgate.net/post/Did-I-actually-measure-a-superluminous-signal-thus-disproving-the-relativity-theory}
\par  }\textstyleNone{{. There is active and ongoing experimental
research in this area.}}}

{
\textstyleNone{\textit{{Experiments regarding gravitational effects,
such as aimed at obtaining thrust.}}}}

{
\textstyleNone{{The Biefeld-Brown effect is an electrical phenomenon
that has been the subject of extensive research involving charging an asymmetric capacitor to high voltages and the
effect is commonly attributed to corona discharges which occur only at the sharp electrode, which causes an imbalance
in the number of positive and negative ions created in comparison to when a symmetric capacitor is used. }}}

{
\textstyleNone{{However, according to a
report}}\footnote{\textstyleNone{ Bahder, Thomas; Fazi, Christian (June 2003). Force on an Asymmetric Capacitor. U.S.
Army Research Laboratory -- via Defense Technical Information Center.
}\url{http://www.dtic.mil/docs/citations/ADA416740}\textstyleHyperlinkii{ }\textstyleNone{ }\par
}\textstyleNone{{ by researchers from the Army Research Laboratory
(ARL), the effects of ion wind was at least three orders of magnitude too small to account for the observed force on
the asymmetric capacitor in the air. Instead, they proposed that the Biefeld--Brown effect may be better explained
using ion drift instead of ion wind. This was later confirmed by researchers from the Technical University of
Liberec}}\footnote{\textstyleNone{ Mal\'ik, M. \& Primas, J. \& Kopeck\'y, V\'aclav \& Svoboda, M.. (2013). Calculation
and measurement of a neutral air flow velocity impacting a high voltage capacitor with asymmetrical electrodes. AIP
Advances. 4. 10.1063/1.4864181. }\par }\textstyleNone{{. }}}

{
\textstyleNone{{If this is correct, then the need for an asymmetric
capacitor raises the question if the resulting diverging electric field can indeed be used to obtain thrust by working
on an electrically neutral dielectric, in this case a dielectric consisting of air and net neutral ions, and how this
results in a net force acting upon the capacitor plates. It is known that a dielectric is always drawn from a region of
weak field toward a region of stronger field. It can be shown that for small objects the force is proportional to the
gradient of the square of the electric field, because the induced polarization charges are proportional to the fields
and for given charges the forces are proportional to the field as well. There will be a net force only if the square of
the field is changing from point to point, so the force is proportional to the gradient of the square of the
field}}\footnote{ Feynman, R. (1963, 2006, 2010). The Feynman Lectures on Physics Vol. II Ch. 10: Dielectrics. (Fig.
10--8.) \url{https://www.feynmanlectures.caltech.edu/II_10.html\#Ch10-F8}  \par
}\textstyleNone{{.}}}

{
\textstyleNone{{Another line of research in this regard has to do
with the gravitational force itself, which }{can be speculated
to}{ be caused by longitudinal dielectric flux, which causes a
pushing and not a pulling force. This is supported by Van Flandern}}\footnote{ Van Flandern, T (1998). The speed of
gravity ? What the experiments say. Physics Letters A. 250 (1--3): 1--11.
\url{http://www.ldolphin.org/vanFlandern/gravityspeed.html} \par
}\textstyleNone{{, who determined that with a purely central
pulling force and a finite speed of gravity, the forces in a two-body system no longer point toward the center of mass,
which would make orbits unstable. The fact alone that a central pulling gravity force requires a practically infinite
speed makes clear that pulling gravity models are untenable and recourse must be taken to a Lesagian type of pushing
gravity model.}}\textstyleNone{{ The longitudinal dielectric flux
which would
}{thus}{
describe gravity is probably caused by cosmic (microwave) background radiation. If this naturally occurring flux had an
arbitrary frequency spectrum, superconductors would reflect this flux and would thus shield gravity, which does not
happen.}}}

{
\textstyleNone{{However, acceleration fields outside a rotating
superconductor were found}}\footnote{\textstyleNone{ Tajmar, Martin \& Plesescu, Florin \& Marhold, Klaus \& de Matos,
C.. (2006). Experimental Detection of the Gravitomagnetic London Moment. }\par
}\textstyleNone{{\textsuperscript{,}}}\footnote{\textstyleNone{
Tajmar, Martin \& Plesescu, Florin \& Seifert, Bernhard \& Marhold, Klaus. (2007). Measurement of Gravitomagnetic and
Acceleration Fields Around Rotating Superconductors. ChemInform. 38. 10.1002/chin.200734232. }\par
}\textstyleNone{{, which are referred to as Gravitomagnetic
effects, and also anomalous acceleration signals, anomalous gyroscope signals and Cooper pair mass excess were found in
experiments with rotating superconductors}}\footnote{\textstyleNone{ De Matos, C. \& Christian, Beck. (2009). Possible
Measurable Effects of Dark Energy in Rotating Superconductors. Advances in Astronomy. 2009. 10.1155/2009/931920.
}}\textstyleNone{{. }}}

{
\textstyleNone{{It can be speculated that the relation Stowe and
Mingst found between the characteristic oscillation frequency of the electron and the cosmic microwave background
radiation is what causes the spectrum of the gravitational flux and that this is related to the characteristic
oscillation frequencies of the electron, neutron and proton as well. If that is the case, then the incoming flux would
resonate with the oscillating particles within the material at these specific frequencies, which would therefore not be
blocked/reflected but would be absorbed/re-emitted along Huygens' principle.}}}

{
\textstyleNone{{It can further be speculated that when objects are
rotated, their ``clock'', the characteristic oscillation frequency of the elemental particles making up the material,
would be influenced, causing them to deviate from the specific frequencies they otherwise operate at. It is conceivable
that this would result in a condition whereby superconductors would indeed reflect the naturally occurring
gravitational flux, which could explain this anomaly.}}}

{\centering
\textstyleNone{\textbf{{Acknowledgments}}}
\par}

{
\textstyleNone{This work would not have been possible without the groundbreaking work of Paul Stowe and Barry Mingst,
who succeeded in integrating the gravitational domain with the electromagnetic domain within a single superfluid based
model. It is this integration that resolves the classic problems associated with aether based theories, namely that
because the gravitational field was considered to be separate from the electromagnetic domain, the movements of
planetary bodies would necessarily result in measurable disturbances in the medium. When no such disturbances were
found in the Michelson-Morley experiment, the aether hypothesis was considered as having been disproven. But because
the gravitational force is now considered to }\textstyleNone{be a force caused by longitudinal dielectric waves, which
propagate trough the medium, this argument no longer applies. And therewith there is no longer any reason to disregard
an aether based theory as a basis for theoretical physics.}}

{
\textstyleNone{This work would also not have been possible without the work of Eric Dollard, N6KPH, who replicated a lot
of Tesla's experiments in the 1980's. It is his demonstrations and analysis of Tesla's work that enabled the very
consideration of an aether based theory as an alternative to the current theoretical model.}}
\end{document}
